\documentclass[Analysis-3]{subfiles}

\begin{document}
\chapter*{Lecture 27} %Set chapter name
\addcontentsline{toc}{chapter}{Lecture 27} %Set chapter title
\setcounter{chapter}{27} %Set chapter counter
\setcounter{section}{0}

\section{Conservative Vector Fields}

Let $S \subseteq \R^3$ be a oriented surface, then we call the orientation vector field $\vec{n} : S \to \R^3$ as the \textbf{normal vector field}. 

\begin{Eg}{}{}
    For $\mathbb{S}^{n-1} := \{ x \in \R^n \mid \| x \| = 1 \}$, then 
    \[
        \vec{n}_1(x) = x \ \forall \, x \in \mathbb{S}^{n-1} \quad \mbox{ and } \quad \vec{n}_2(x) = -x \ \forall \, x \in \mathbb{S}^{n-1}
    \]
    are the two normal vector fields on the sphere. 
\end{Eg}

\begin{Eg}{}{}
    Let $\mathcal{G}(f) = \{ (x,y,f(x,y)) \mid (x,y) \in \mathcal{O}_2\}$ where $f : \mathcal{O}_2 \to \R$ is a $\mathscr{C}^1$ function. Then a parametrisation of the surface $\mathcal{G}(f)$ is given by the function 
    \begin{align*}
        \vec{r} : \mathcal{O}_2 &\to \R^3 \\ 
        (x,y) &\mapsto (x,y,f(x,y))
    \end{align*}
    Then we have 
    \[
        \vec{r}_x \times \vec{r}_y = (-f_x, -f_y, 1)    
    \]
    Then a normal vector field is given by 
    \[
        \vec{n}(x,y) = \frac{\vec{r}_x \times \vec{r}_y}{\|\vec{r}_x \times \vec{r}_y \|}    
    \]
    Unless otherwise mentioned this will be our standard orientation of the normal vector field. 
\end{Eg}

Usually computation of $\int_S \vec{F} \cdot \dd \vec{S}$ is complicated, let us look at some exmaples to gain more familiarity. 

\begin{Eg}{}{}
    Consider the vector field $\vec{F}(x,y,z) = (x,y,z)$ on $S = \mathrm{im} \, r$, where 
    \[
        \vec{r}(x,y) = (\cos x, \sin x, y) \quad 0 \leq x \leq \frac{\pi}{2}, \, 0 \leq y \leq 1    
    \]
    Then $\vec{r}_x \times \vec{r}_y = (\cos x, \sin x, 0)$, so $\vec{n}(x,y) = (\cos x, \sin x, 0)$ is a normal vector field. 
    \begin{align*}
        \int_S \vec{F} \cdot \dd \vec{S} &= \int_S \vec{F} \cdot \vec{n} \dd s \\ 
        &= \int_0^1 \int_0^{\frac{\pi}{2}} \vec{F}(\vec{r}(x,y)) \cdot (\vec{r}_x \times \vec{r}_y) \, \dd A \\ 
        &= \int_0^1 \int_0^{\frac{\pi}{2}} \left\langle (\cos x, \sin x , y) , (\cos x, \sin x, 0)\right\rangle \dd A \\ 
        &= \int_0^1 \int_{0}^{\frac{\pi}{2}} \dd A \\ 
        &= \frac{\pi}{2} 
    \end{align*}
\end{Eg}

We already know that $\int_{\mathcal{C}} \nabla f \cdot \dd r = f(B) - f(A)$, now a natural question that arises is 

\textbf{Question:} Given $\vec{F}$ does there exists $f$ a scalar field such that $\nabla f = \vec{F}$ ?  

\begin{Def}{Conservative Vector Field}{}
    A vector field $\vec{F}$ on $\mathcal{O}_n$ is called \textbf{conservative} if there exists a scalar field $f \in \mathscr{C}^1(\mathcal{O}_n)$ such that $\nabla f = \vec{F}$, then $f$ is called the \textbf{potential function}. 
\end{Def}

\begin{Thm}{}{}\label{thm1:27}
    Let $\vec{F}$ be a vector field over $\mathcal{O}_n$, the following are equivalent: 
    \begin{enumerate}
        \item $\vec{F}$ is conservative.
        \item $\int_{\mathcal{C}}\vec{F} \cdot \dd r = 0$, for all closed and piecewise smooth curve $\mathcal{C}$.
        \item $\int_{\mathcal{C}_1} \vec{F} \cdot \dd r = \int_{\mathcal{C}_2} \vec{F} \cdot \dd r$, for all curves $\mathcal{C}_1$ and $\mathcal{C}_2$ with same initial and end points.
    \end{enumerate}
\end{Thm}

\textbf{Question:} Given a vector field $\vec{F}$, can we conclude $\vec{F}$ is conservative? (NO!)

We will give a general picture for the most common case, when $n = 3$. Let $\vec{F} = (P,Q,R)$ where $P,Q,R$ are scalar fields. Now if $\vec{F} = \nabla f$ for some scalar field $f$, then we would have 
\begin{align}\label{eq1:27}
    &f_x \equiv \pdv{f}{x} = P \\ 
    &f_y \equiv \pdv{f}{y} = Q \\ 
    &f_z \equiv \pdv{f}{z} = R
\end{align}
Then we can define \textbf{curl} of a vector field 
\[
    \vec{\nabla} \times \vec{F} := \begin{vmatrix}
        \hat{i} & \hat{j} & \hat{k} \\ 
        \pdv{}{x} & \pdv{}{y} & \pdv{}{z} \\ 
        P & Q & R
    \end{vmatrix}    
\]
Then expanding this out and using the relations $\ref{eq1:27}$ and others we get that $\vec{\nabla} \times \vec{F} = \mathbf{0}$. Thus we have proved that if $\vec{F}$ is conservative then $\vec{\nabla} \times \vec{F} = \mathbf{0}$.

\ 

\textbf{Remark.} Thus a necessary condition for a vector field to be conservative is that, its curl should be the zero vector field. 

\ 

\begin{Eg}{}{}
    Let $\vec{F}(x,y) = (y-3,x+2) = (P,Q)$ (say), then $\pdv{P}{y} = \pdv{Q}{x} = 1$. Let $f$ be a possible potential function, then 
    \[
        \pdv{f}{x} = y-3 \quad \mbox{ and } \quad \pdv{f}{y} = x+2     
    \] 
    Then by \textbf{Fundamental Theorem of Calculus} (assuming domain is convex) we get 
    \[
        f(x,y) = xy - 3x + g(y)    
    \] 
    But then using $\pdv{f}{y} = x+2$ we get 
    \[
        x + g'(y) = \pdv{f}{y} = x+2 \Rightarrow g'(y) = 2     
    \]
    Therefore taking $f(x,y) = xy - 3x + 2y$ gives us a potential function for the vector field $\vec{F}$. 
\end{Eg}


\textbf{Remark.} This approach works for all $\vec{F}$ such that $\vec{\nabla} \times \vec{F} = \mathbf{0}$ and the domain is convex.   

\begin{Eg}{}{}
    Let $\vec{F}(x,y) = \left( \frac{-y}{x^2+y^2}, \frac{x}{x^2+y^2} \right) = (P,Q)$ (say) on $\R^2 \setminus \{\mathbf{0}\}$. Then we have $\pdv{P}{y} = \pdv{Q}{x}$, but we will show that $\vec{F}$ is not conservative. Consider the curve 
    \[
        \mathcal{C} : \gamma(t) = ( \cos t, \sin t), \quad 0 \leq t \leq 2\pi    
    \]
    then 
    \begin{align*}
        \int_{\mathcal{C}} \vec{F} \cdot \dd r &= \int_0^{2\pi} \left\langle \vec{F}(\gamma(t)),\gamma'(t) \right\rangle \dd t \\ 
        &= \int_0^{2\pi} \left\langle (-\sin t, \cos t), (-\sin t, \cos t) \right\rangle \dd t \\ 
        &= \int_0^{2\pi} \dd t \\ 
        &= 2\pi
    \end{align*}
    But $\mathcal{C}$ is clearly a closed curve, hence by theorem $\ref{thm1:27}$ we must have $\int_{\mathcal{C}} \vec{F} \cdot \dd r = 0$. (Contradiction!)
\end{Eg}

\section{Green's Theorem}

\begin{Def}{Simply Connected Domain}{}
    Let $\mathcal{D}$ be a open and connected set. Let $\mathcal{C}$ be a simple and closed curve if $\mathcal{C}$ can be shrunk continuously to a point inside $\mathcal{D}$, then we say $\mathcal{D}$ is \textbf{simply connected}.
\end{Def}

\begin{Thm}{Green's Theorem}{}\label{thm2:27}
    Let $\mathcal{R} \subseteq \R^2 $ be a simply connected domain with boundary curve $\mathcal{C}$ where parametrisation is taken in anti-clockwise direction. Let $\vec{F} = (P,Q)$ be a $\mathscr{C}^1$ vector field on $\mathcal{R}$, then 
\[
    \int_{\mathcal{C}} \vec{F} \cdot \dd r := \int_{\mathcal{C}} P \, \dd x + Q \, \dd y = \int_{\mathcal{R}} \left( \pdv{Q}{x} - \pdv{P}{y}\right) \dd A 
\]
\end{Thm}

What happens when $\mathcal{R}$ is not simply connected? 
\begin{align*}
    \int_{\mathcal{C}} P \dd x + Q \dd y &= \int_{\tilde{\mathcal{C}}_1} P \dd x + Q \dd y + \int_{\tilde{\mathcal{C}}_2} P \dd x + Q \dd y \\ 
    &= \int_{\mathcal{R}_1} (Q_x - P_y) \dd A + \int_{\mathcal{R}_2} (Q_x - P_y) \dd A \\ 
    &= \int_{\mathcal{R}} (Q_x - P_y) \dd A
\end{align*}


\end{document}
