\documentclass[Analysis-3]{subfiles}
\usepackage{style}


\begin{document}
\chapter*{Lecture 12} %Set chapter name
\addcontentsline{toc}{chapter}{Lecture 12} %Set chapter title
\setcounter{chapter}{12} %Set chapter counter
\setcounter{section}{0}

\section{Hessian Matrix}

\begin{Def}{Hessian}
    . Suppose $f : \mathcal{O}_n \to \mathbb{R}$ is $C^2$ at $a \in \mathcal{O}_n $. The $\textbf{Hessian of $f$ at $a$}$ is defined by, \[ H_f(a) = \left[ \frac{\partial^2 f}{\partial x_i \partial x_j}(a) \right]_{n \times n} \]  
\end{Def}

\[ \therefore H_f = \begin{bmatrix}
    \frac{\partial^2 f}{\partial {x_1}^2} & \frac{\partial^2 f}{\partial x_1 \partial x_2} & \ldots & \frac{\partial^2 f}{\partial x_1 \partial x_n} \\ \\
    \frac{\partial^2 f}{\partial x_2 \partial x_1} & \frac{\partial^2f}{\partial {x_2}^2} & \ldots & \frac{\partial^2f}{\partial x_2 \partial x_n} \\ \\
    \vdots & \vdots & \ddots & \vdots \\ \\
    \frac{\partial^2f}{\partial x_n \partial x_1} & \frac{\partial^2f}{\partial x_n \partial x_2} & \ldots & \frac{\partial^2 f}{\partial {x_n}^2}
\end{bmatrix} \]


$\bullet$ $H_f = {H_f}^T \hspace{5mm} (\text{i.e. symmetric})$ 

\begin{Eg}{}
    .$f : \mathbb{R}^2 \to \mathbb{R}$ be a function defined by $f(x,y) = \sin^2 x + x^2y + y^2$ \\

    Then, \[Df = \begin{bmatrix}
        \sin2x + 2xy & x^2 + 2y 
    \end{bmatrix} :  \mathbb{R}^2 \to \mathbb{R} \hspace{5mm} \text{is linear.} \]
    Gradient = \[\bigtriangledown f = \langle \sin2x + 2xy , x^2 + 2y \rangle \in \mathbb{R}^2 \] 


\[ H_f = \begin{bmatrix}
    f_{XX} & f_{XY} \\
    f_{XY} & f_{YY}
\end{bmatrix} = \begin{bmatrix}
    2( \cos2x + y) & 2x \\
    2x & 2
\end{bmatrix} \hspace{5mm} [ \because  f \in C_2] \]

\end{Eg}

\begin{notnBox}
   Given $A \in M_n(\mathbb{R})$ , $ \hspace{5mm} A = \left[a_{ij}\right] n \times n $
 \[ \forall x \in \mathbb{R}^n , Q_A(x) = \underbrace{x^tAx}_{\begin{bmatrix}
    x_1 & x_2 & \ldots & x_n
 \end{bmatrix} \begin{bmatrix}
    &&\\
    & a_{ij} & \\
    &&
 \end{bmatrix} \begin{bmatrix}
    x_1 \\
    x_2 \\
    \vdots \\
    x_n
 \end{bmatrix}} \left( = \langle Ax, x \rangle_{\mathbb{R}^n} \right) \]
   
\[ \implies Q_A(x) = \sum_{i,j = 1}^{n} a_{ij} \bar{x_i} x_j  \]

\end{notnBox}


\begin{Def}{Quadretic Form}
    .A fuction $f : \mathbb{R}^n \to \mathbb{R}$ is a $\textbf{Quadretic Form}$ if $f(x) = Q_A(x) \hspace{3mm} \forall x $ for some symmetric $A \in M_n(\mathbb{R})$
\end{Def}

$\bullet$ A Quadretic Form is a homogeneous polynomial of degree 2

$\vspace{3mm}$

$\bullet$ $p(x,y) = a_1 x^2 + a_2 y^2 + a_{12} xy$
Hence, \[ A = \begin{bmatrix}
    a_1 & \frac{1}{2} a_{12} \\
    \frac{1}{2} a_{12} & a_2
\end{bmatrix} \hspace{5mm} \implies p = Q_A \]

\begin{Def}{Positive Definite, Negative Definite, Semi Definite}
    . A symmetric matrix $A \in M_n(\mathbb{R})$ is $\textbf{Positive Definite}$ if \[ \langle Ax, x \rangle > 0 \hspace{3mm} \forall x \in \mathbb{R}^n \backslash \{0\} \]
    \vspace{2mm}
    A symmetric matrix $A \in M_n(\mathbb{R})$ is $\textbf{Negative Definite}$ if \[ \langle Ax, x \rangle < 0 \hspace{3mm} \forall x \in \mathbb{R}^n \backslash \{0\} \]
    A symmetric matrix $A \in M_n(\mathbb{R})$ is $\textbf{Semi Definite}$ if \[ \langle Ax, x \rangle \geq 0 \hspace{3mm} \forall x \in \mathbb{R}^n \backslash \{0\} \]
\end{Def}

\begin{Eg}{}
    . \begin{enumerate}
        \item  $I_n$ is positive definite as $\langle I_nx, x \rangle = {\|x \|}^2 > 0 \hspace{5mm} \forall x \in \mathbb{R}^n \backslash \{0\}$ 
        \item $A = B^TB$ for some $B \in M_n{\mathbb{R}}$
         \begin{align*}
            \langle Ax,x \rangle & = \langle B^TBx,x \rangle \\
            &= x^TB^TBx \\
            &= {(Bx)}^T Bx = {\| Bx \|}^2 \hspace{5mm} \forall x \in \mathbb{R}^n \backslash \{0\} 
         \end{align*}   
         Hence, \[ \langle Ax,x \rangle \geq 0 \]

         Now, if $ \langle Ax,x \rangle = 0 $ , then  $x$ is kernel of $B$. \\
         Now, if $A$ is positive definite , there is no such $x$ , hence, columns of $B$ are linearly independent. \\

         In general $A$ is $\textbf{Positive Semi Definite}$
         ( $"\Longleftarrow"$ also holds )

         \item $\begin{bmatrix}
            1 & 0 \\
            0 & -1
         \end{bmatrix} \longrightarrow Q_A = {x_1}^2 - {x_2}^2 \longrightarrow \textbf{indefinite}$.
         \item  $\begin{bmatrix}
            1 & 0 \\
            0 & 0
         \end{bmatrix} \longrightarrow Q_A = {x_1}^2 \geq 0 \longrightarrow \textbf{Positive Semi Definite}$ 

    \end{enumerate}
\end{Eg}

$\bullet \hspace{2mm}$ Now, 
\begin{align*} & \langle Ah, h \rangle = \| Ah\| \| h \| \cos \theta > 0 \\\\
   & \implies \boxed{0 \leq \theta \leq \frac{\pi}{2}}
\end{align*}  


$\bullet \hspace{2mm} $ In general it is difficult to classify positive definite $n \times n$ matrices . But not for $n = 2$ .

\begin{Thm}{}
    . Let $A = \begin{bmatrix}
        a & b \\
        b & c
    \end{bmatrix} \in M_2(\mathbb{R}) $ be symmetric. Then :-

    \begin{enumerate}
        \item $A$ is $\textbf{Positive Definite} \Leftrightarrow a > 0 \hspace{1mm} \text{and} \hspace{1mm} ac -b^2 >0 $ 
        \item $A$ is $\textbf{Negative Definite} \Leftrightarrow a < 0 \hspace{1mm} \text{and} \hspace{1mm} ac -b^2 >0 $ 
        \item $A$ is $\textbf{Indefinite} \Leftrightarrow ac -b^2 <0 $ 
    \end{enumerate}
\end{Thm}

$\textbf{Proof :-}$ \[ \langle, h \rangle = h^TAh\]
Now , scaling $h$ is okay as sign would not be changed. \\

Pick, $X = (x_1 , x_2) \in \mathbb{R}^2 \backslash \{(0,0)\} \hspace{5mm}$ [Let, $x_2 \neq 0$] \\
WLOG, $X = (x,1) \hspace{3mm} x \in \mathbb{R} \hspace{3mm}$ [Scaling] \\
\[\therefore \langle Ax, x \rangle = ax^2 + 2bx + c > 0 \hspace{3mm} \forall x \in \mathbb{R}\]

Now, if $x_2 = 0$, WLOG, $X = \begin{bmatrix}
    1 \\
    0
\end{bmatrix} \hspace{3mm}$ (Scaling) \\

Then, \[ \langle Ax, x \rangle = a \]

\begin{align*}
    \therefore \hspace{1mm} & \hspace{1mm} \text{A is Positive Definite} \\
    \implies & \Leftrightarrow a > 0 \hspace{1mm} \text{and} \hspace{1mm} ax^2 + bx+c > 0 \hspace{3mm} \forall x \in \mathbb{R} \\
    \implies & \Leftrightarrow a > 0 \hspace{1mm} \text{and} \hspace{1mm} (2b)^2 - 4ac < 0 \\
    \implies & \Leftrightarrow a > 0 \hspace{1mm} \text{and} \hspace{1mm} ac - b^2 > 0
\end{align*}

By same method, 
\begin{align*}
    \therefore \hspace{1mm} & \hspace{1mm} \text{A is Negative Definite} \\
    \implies & \Leftrightarrow a < 0 \hspace{1mm} \text{and} \hspace{1mm} ax^2 + bx+c < 0 \hspace{3mm} \forall x \in \mathbb{R} \\
    \implies & \Leftrightarrow a < 0 \hspace{1mm} \text{and} \hspace{1mm} (2b)^2 - 4ac < 0 \\
    \implies & \Leftrightarrow a < 0 \hspace{1mm} \text{and} \hspace{1mm} ac - b^2 > 0
\end{align*}

Now, 
\begin{align*}
    \therefore \hspace{1mm} & \hspace{1mm} \text{A is Indefinite} \\
    \implies & \Leftrightarrow ax^2 + bx+c < 0 \hspace{3mm} \text{for some} \hspace{1mm} x \in \mathbb{R} \hspace{1mm} \text{and} \hspace{1mm} ax^2 + bx+c > 0 \hspace{3mm} \text{for some} \hspace{1mm} x \in \mathbb{R} \\
    \implies & \Leftrightarrow (2b)^2 - 4ac > 0 \\
    \implies & \Leftrightarrow ac - b^2 < 0
\end{align*}

\begin{LemN}{}
    . Let, $a \in \mathcal{O}_n , A(x) = \begin{bmatrix}
        a_1(x) & a_2(x) \\
        a_2(x) & a_3(x)
    \end{bmatrix}$ \\
    Suppose, $A$ is continuous at $a \hspace{3mm} \forall x \in \mathcal{O}_n \hspace{3mm}$ [i.e. $a_i's $ are continuous at $a$] .\\
    If $A$ is Positive Definite at $a$ , then $A$ is Positive Definite in a nbd of $a$.
\end{LemN}

\textbf{Proof :- } \begin{align*}
    & A(a) \hspace{1mm} \text{is Positive Definite} \\
    \implies & a_1(a) > 0 \hspace{1mm} \text{and} \hspace{1mm} a_1(a) a_3(a) - {a_2}^2(a) > 0 \\
    \implies & \exists B_{\epsilon} (a) \hspace{1mm} \text{such that both are positive}. 
\end{align*} 

We are done . $\Qed$
















\end{document}