\documentclass[Analysis-3]{subfiles}
\usepackage{style}


\begin{document}
\chapter*{Lecture 11} %Set chapter name
\addcontentsline{toc}{chapter}{Lecture 11} %Set chapter title
\setcounter{chapter}{11} %Set chapter counter
\setcounter{section}{0}

\section{Recall}

\[\begin{tikzcd}
	{\mathcal{O}_n} &&& {\mathcal{O}_m} &&& {\mathbb{R}^p}
	\arrow["f"', from=1-1, to=1-4]
	\arrow["g"', from=1-4, to=1-7]
	\arrow["{g \circ f}", curve={height=-30pt}, from=1-1, to=1-7]
\end{tikzcd}\]


where f and g are both differentiable functions.

\[ J_{g \circ f}(a) = J_{g}(f(a)) . J_{f}(a) \hspace{1cm} [ \text{Chain Rule}]  \]

Now, comparing the $(i,j)^{th}$ element , we get :-
\[ \frac{\partial (g \circ f)_{i}}{\partial x_j}(a) = \sum_{k = 1}^{m}   \frac{\partial g_{i}}{\partial y_k}(b) . \frac{\partial f_{k}}{\partial x_j}(a)  \hspace{1cm} \left( \text{Where $b = f(a)$} \right)\]

This may be written in a slightly more suggestive sense :-

% \[\begin{tikzcd}
%	{\frac{\partial (g \circ f)_{i}}{\partial x_j} = \sum_{k = 1}^{m}   \frac{\partial g_{i}}{\partial y_k} . \frac{\partial y_{k}}{\partial x_j}} \\
%	{} \\
%	{y_k = f_k \left( x_1 , x_2 , \ldots, x_n \right) \\ z_k = g_k \left( y_1 , y_2, \ldots , y_m \right)} && {} & {\boxed{\frac{\partial z_{i}}{\partial x_j} = \sum_{k = 1}^{m}   \frac{\partial z_{i}}{\partial y_k} . \frac{\partial y_{k}}{\partial x_j}}} \\
%	&&& {}
%	\arrow[shift right=5, curve={height=30pt}, from=3-1, to=1-1]
%	\arrow[from=3-1, to=3-4]
%\end{tikzcd}\]




\begin{Eg}{}
	. Let, $f(x,y,z) = xy^{2}z$ and $x=t, y=e^t, z= 1+t$ \\
	Then, \begin{align*}
		f(x,y,z) &= t(e^t)^2(1+t)\\
		&= (t+t^2)e^{2t}
	\end{align*}

Hence, \begin{align*}
	\frac{df}{dt} &= \frac{d}{dt}[(t+t^2)e^{2t}] \\
	&= (1+2t)e^{2t} + 2(t+t^2)e^{2t}\\
	&= (2t^2+4t+1)e^{2t} 
\end{align*}

Also, \begin{align*}
	\frac{df}{dt} &= \frac{\partial f}{\partial x} \frac{dx}{dt} + \frac{\partial f}{\partial y} \frac{dy}{dt} + \frac{\partial f}{\partial z} \frac{dz}{dt} \\
	&= y^2z . 1 + 2xyz.e^t + xy^2.1 \\
	&= e^{2t}(1+t) + 2t(1+t)e^t.e^t + te^{2t} \\
	&= e^{2t}[1+t+2t+2t^2+t] \\
	&= (2t^2+4t+1)e^{2t}
\end{align*}

\end{Eg}


\section{Polar Coordinate}

\begin{Def}{Laplacian}
	. $f : \mathcal{O}_n \to \mathbb{R}$ be a function. Then the Laplacian of f is defined by :- \[ \Delta f = \sum_{i = 1}^{n} \frac{\partial^2 f}{ \partial {x_i}^2}   \]
\end{Def}

Now, \begin{align*}
	\Delta f &= \sum_{i = 1}^{n} \frac{\partial^2 f}{ \partial {x_i}^2} \\
	&= \langle \frac{\partial}{\partial x_1}, \ldots, \frac{\partial}{\partial x_n}  \rangle . \langle \frac{\partial f}{\partial x_1}, \ldots, \frac{\partial f}{\partial x_n}  \rangle \\
	&= \bigtriangledown . \bigtriangledown f 
\end{align*}

Hence, \[\boxed{\Delta f = \bigtriangledown . \bigtriangledown f}\]


Let, $x = r\cos \theta$ and $y= r\sin \theta$ and $f : \mathbb{R}^2 \to \mathbb{R} \hspace{5mm} ( f \in C^2) $

Hence, 
\[\begin{matrix}
	\frac{\partial x}{\partial r} = \cos \theta , & \frac{\partial x}{\partial \theta} = -r\sin \theta \\
	\frac{\partial y}{\partial r} = \sin \theta , & \frac{\partial y}{\partial \theta} = r\cos \theta
\end{matrix}\]

 \begin{align*}
	\therefore \hspace{1cm}	& \frac{\partial f}{\partial r } = \frac{\partial f}{\partial x} \frac{\partial x}{\partial r} + \frac{\partial f}{\partial y} \frac{\partial y}{\partial r} \\
	 \implies& \frac{\partial f}{\partial r } = \frac{\partial f}{\partial x} \cos \theta + \frac{\partial f}{\partial y} \sin \theta \\
	\implies & \frac{\partial^2 f}{\partial r^2 } = \frac{\partial}{\partial r} \left[\frac{\partial f}{\partial x} \cos \theta + \frac{\partial f}{\partial y} \sin \theta \right] \\
	 \implies & \frac{\partial^2 f}{\partial r^2 } = \cos \theta \left[\frac{\partial}{\partial r} \frac{\partial f}{\partial x}\right] + \sin \theta \left[\frac{\partial}{\partial r} \frac{\partial f}{\partial y} \right] \\
	 \implies & \frac{\partial^2 f}{\partial r^2 } = \cos \theta \left[\frac{\partial^2 f}{\partial x^2} \frac{\partial x}{\partial r} + \frac{\partial^2 f}{\partial y \partial x} \frac{\partial y}{\partial r} \right] + \sin \theta \left[\frac{\partial^2 f}{\partial x \partial y} \frac{\partial x}{\partial r} + \frac{\partial^2 f}{\partial y^2} \frac{\partial y}{\partial r} \right] \\ 
	 \implies & \frac{\partial^2 f}{\partial r^2 } = \cos \theta \left[\frac{\partial^2 f}{\partial x^2} \cos \theta + \frac{\partial^2 f}{\partial y \partial x} \sin \theta \right] + \sin \theta \left[\frac{\partial^2 f}{\partial x \partial y} \cos \theta + \frac{\partial^2 f}{\partial y^2} \sin \theta \right] \\
	 \implies & \frac{\partial^2 f}{\partial r^2 } = \cos \theta \left[\cos \theta f_{XX} + \sin \theta f_{XY} \right] + \sin \theta \left[\cos \theta f_{XY} + \sin \theta f_{YY} \right] \\
	 \implies & \boxed{\frac{\partial^2 f}{\partial r^2 } = \cos^2 \theta f_{XX} + \sin^2 \theta f_{YY} + \sin 2 \theta f_{XY}}
\end{align*}


Similarly, \[ \boxed{\frac{\partial^2 f}{\partial \theta^2 } = -r \left( \cos \theta f_X + \sin \theta f_Y \right) + \left( r^2\sin^2 \theta f_{XX} + r^2\cos^2 \theta f_{YY} - r^2\sin 2 \theta f_{XY} \right)} \]

Hence, we can show that :- 
\[ \boxed{\Delta f = f_{XX} + f_{YY} = \frac{\partial^2 f}{\partial r^2} + \frac{1}{r} \frac{\partial f}{\partial r} + \frac{1}{r^2} \frac{\partial^2 f}{\partial \theta^2}} \]

\begin{Eg}{}
	. Let, $z = z(u,v)$ where , \[ u(x,y) = x^2y \text{and} v(x,y) = 3x + 2y \]
	Hence, \[\begin{matrix}
		\frac{\partial u}{\partial x} = 2xy , & \frac{\partial u}{\partial y} = x^2 \\
		\frac{\partial v}{\partial x} = 3 , & \frac{\partial v}{\partial \theta} = 2
	\end{matrix}\]

\begin{align*}
	\therefore \hspace{1cm}	& \frac{\partial z}{\partial x } = \frac{\partial z}{\partial u} \frac{\partial u}{\partial x} + \frac{\partial z}{\partial v} \frac{\partial v}{\partial x} \\
	\implies & \frac{\partial z}{\partial x } = 2xy \frac{\partial z}{\partial u} + 3 \frac{\partial z}{\partial v}\\
    \implies & \frac{\partial^2 z}{\partial x^2 } = \frac{\partial}{\partial x} \left[ 2xy \frac{\partial z}{\partial u} + 3 \frac{\partial z}{\partial v} \right] \\
	\implies & \frac{\partial^2 z}{\partial x^2 } = 2y \frac{\partial z}{\partial u} + 2xy \frac{\partial}{\partial x} \left[\frac{\partial z}{\partial u}\right]  + 3\frac{\partial}{\partial x} \left[\frac{\partial z}{\partial v}\right] \\
	\implies & \frac{\partial^2 z}{\partial x^2 } = 2y \frac{\partial z}{\partial u} + 2xy \left(\frac{\partial}{\partial u} \left[\frac{\partial z}{\partial u}\right] \frac{\partial u}{\partial x} + \frac{\partial}{\partial v} \left[\frac{\partial z}{\partial u}\right] \frac{\partial v}{\partial x}\right)  + 3\left( \frac{\partial}{\partial u} \left[\frac{\partial z}{\partial v}\right] \frac{\partial u}{\partial x} + \frac{\partial}{\partial v} \left[\frac{\partial z}{\partial v}\right] \frac{\partial v}{\partial x} \right) \\
	\implies & \frac{\partial^2 z}{\partial x^2 } = 2y \frac{\partial z}{\partial u} + 2xy \left(\frac{\partial^2 z}{\partial u^2} \frac{\partial u}{ \partial x} + \frac{\partial^2 z}{\partial v \partial u} \frac{\partial v}{\partial x}\right)  + 3\left( \frac{\partial^2 z}{\partial u \partial v} \frac{\partial u}{ \partial x} + \frac{\partial^2 z}{\partial v^2} \frac{\partial v}{\partial x}\right) \\
	\implies & \frac{\partial^2 z}{\partial x^2 } = 2y \frac{\partial z}{\partial u} + 2xy \left(2xy \frac{\partial^2 z}{\partial u^2}  + 3 \frac{\partial^2 z}{\partial v \partial u}\right)  + 3\left( \frac{\partial^2 z}{2xy \partial u \partial v}  + 3 \frac{\partial^2 z}{\partial v^2} \right) \\ 
	\implies & \boxed{\frac{\partial^2 z}{\partial x^2 } = 2y z_u + 4x^2y^2z_{uu}+ 6xy z_{uv} + 6xy z_{vu} + 9z_{vv}}
\end{align*}


\end{Eg}


$\bullet$ $\textbf{Homework :-}$

Find $z_{yy} , z_{yx} , z_{xy}$ and check if $z_{xy} = z_{yx}$.

\section{Extrema}

\begin{Def}{Extrema}
	. Let, $a$ is a interior point of $S \subseteq \mathbb{R}^n$ and $f : S \to \mathbb{R}$ be a function . $f$ attains a local maximum at $a$ if $\exists a \in \mathcal{O}_n \subseteq S$ such that $f(a) \geq f(x) \hspace{3mm} \forall x \in \mathcal{O}_n$. Similarly, $f$ attains a local minimum at $a$ if $\exists a \in \mathcal{O}_n \subseteq S$ such that $f(a) \leq f(x) \hspace{3mm} \forall x \in \mathcal{O}_n$. Now , either local maximum or local minimum is called $\textbf{Extremum}$ (In plural $\textbf{Extrema}$) .   
\end{Def}

\begin{Def}{Critical Point / Stationary Point}
	Let, $f : S (\subseteq \mathbb{R}^n) \to \mathbb{R}$ be a function and $a \in \mathcal{O}_n \subseteq S$. We say that $a$ is a $\textbf{critical point}$ or $\textbf{stationary point}$
	 if \[ \left(\bigtriangledown f\right)(a) = 0 \hspace{5mm} \Leftrightarrow f_{X_i}(a) = 0 \hspace{3mm} \forall i = 1(1)n \]
\end{Def}

\begin{Thm}{}
	. Let, $f: \mathcal{O}_n \to \mathbb{R}$ is differentiable at $a \in \mathcal{O}_n$ . If $a$ is a local extremum, then \[ \left(\bigtriangledown f\right)(a) = 0 \]
\end{Thm}

$\textbf{Proof:-}$ $\hspace{1cm}$ Fix $i \in \{ 1,2, \ldots, n \}$

\begin{clmBox}
	$\textbf{Claim :-} \hspace{5mm} f_{X_i}(a) = 0$ 
\end{clmBox}

Set, $\phi_i : (a_i - \epsilon , a_i + \epsilon) \to \mathbb{R}$ defined by $\phi_i(t) = f(a_1, \ldots,a_{i-1},t, a_{i+1}, \ldots, a_n)$

Hence, \[ \frac{d \phi_i}{dt}(a_i) = f_{X_i}(a) \]

Also, $a_i$ is a local extremum of $\phi_i$

Hence, \[ \frac{d \phi_i}{dt}(a_i) = 0 \hspace{3mm} \implies \hspace{3mm} f_{X_i}(a) = 0 \hspace{5cm} \Qed \]


$\bullet \hspace{2mm} \boxed{Q}$ What is 2nd order total derivative ?

$\boxed{Ans}$ Hessian Matrix.








\end{document}

























