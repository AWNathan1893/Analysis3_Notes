\documentclass[Analysis-3]{subfiles}

\begin{document}
\chapter*{Lecture 28} %Set chapter name
\addcontentsline{toc}{chapter}{Lecture 28} %Set chapter title
\setcounter{chapter}{28} %Set chapter counter
\setcounter{section}{0}

\newcommand{\inti}[3]{\int_{#1} {#2} \hspace{0.1cm} \dd {#3}}
%The content

\section{Green's Theorem}
\begin{Thm}{$\R^2$ version of Green's Theorem}{}
    Let $\mathcal{R} \subseteq \R^2 $ be a simply connected domain with boundary curve $\mathcal{C}$ where parametrisation is taken in anti-clockwise direction. Let $\vec{F} = (P,Q)$ be a $\mathscr{C}^1$ vector field on $\mathcal{R}$, then 
\[
    \int_{\mathcal{C}} \vec{F} \cdot \dd r := \int_{\mathcal{C}} P \, \dd x + Q \, \dd y = \int_{\mathcal{R}} \left( \pdv{Q}{x} - \pdv{P}{y}\right) \dd A 
\]
\end{Thm}

\textit{Proof}.(for Simple region)

\begin{wrapfigure}{r}{0.50\textwidth}
    \centering
    \includegraphics[width=.78\linewidth]{figures/lec-28.1.png}
    \caption{A simple region}
\end{wrapfigure}

\vspace{0.2cm}

 Let, $\mathcal{R} = \qty{(x,y) | a\le x\le b, \varphi_1(x) \le y \le \varphi_2(x)}$ be a simple region. Here $\mathcal{C} = c_1 \cup v_2 \cup c_2 \cup v_2$ is the curve bounding the region along anti-clockwise direction (as shown in the picture). 

 \begin{align*}
    -\int_{\mathcal{R}} \pdv{P}{y} \dd A &= -\int_{a}^{b} \int_{\varphi_1(x)}^{\varphi_2(x)} \pdv{P}{y} \dd y \dd x \\
    &= - \int_{a}^{b} (P(x,\varphi_2(x))-P(x,\varphi_1(x))) \dd x \\
 \end{align*}

 Now, $c_1,c_2,v_1,v_2$ can be explicitly written as, 
 \begin{align*}
    v_1 &= \qty{(a,t) | \varphi_1(a) \le t \le \varphi_2(a)} \\
    c_1 &= \qty{(x,\varphi_1(x)) | a \le x \le b} \\
    v_2 &= \qty{(a,t) | \varphi_1(b) \le t \le \varphi_2(b)} \\
    c_2 &= \qty{(x,\varphi_2(x)) | a \le x \le b}
 \end{align*}

 Now,

 \begin{align*}
    \int_{v_1} P \dd x &= \int_{v_1} P(x(t),y(t)) \dv{x(t)}{t} \dd t = 0 \\
    \int_{c_1} P \dd x &= \int_{a}^{b} P(t,\varphi_1(t)) \dd t \\
    \int_{c_2} P \dd x &= \int_{a}^{b} P(t,\varphi_2(t)) \dd t \\
    \implies \int_{\mathcal{C}} P\dd x &= -\int_{\mathcal{R}} \pdv{P}{y} \dd A
 \end{align*}


 By similar mechanism we can show $\int_{\mathcal{C}} Q \dd y = \int_{\mathcal{R}} \pdv{Q}{y} \dd A$. The resst follows from here. $\hfill \blacksquare$
\pagebreak
\begin{Eg}{}{}
    Let, $\mathcal{C}$ be the bounday of $[0,1]^2$ .i.e $\partial [0,1]\times[0,1] = \mathcal{C}$. Evaluate

    \[\int_{\mathcal{C}} \langle x^2-y^2,2xy \rangle\]
    \textit{Solution.} We can decompose $\mathcal{C} = c_1 \cup c_2 \cup c_3 \cup c_4$ (as in the following picture)

    \begin{wrapfigure}{r}{0.25\textwidth}
        \centering
        \includegraphics[width=.78\linewidth]{figures/lec-28.2.png}
        \caption{$\partial( [0,1]^2)$}
    \end{wrapfigure}
    
    Let, $P(x,y) = x^2 - y^2, Q(x,y) = 2xy$. Then the integral, 
    
    \begin{align*}
        \int_c P \dd x + Q \dd y &= \iint_{[0,1]^2} (2y + 2y) \dd A \hspace{0.2cm} \text{(Green's Theorem)}\\
        &= \int_{0}^1 \int_0^1 4y \hspace{0.1cm} \dd y \dd x \\
        &= 2
    \end{align*}
    
    If we try to calculate the integral manually then also we will end up getting same result.
    
    $\hfill \blacksquare$
\end{Eg}

\begin{tcolorbox}
    \textbf{Area of a closed Region.} Let, $\mathcal{R}$(simply connected) be a closed region and $\mathcal{C} = \partial{\mathcal{R}}$ be the curve enclosing the region. Using Green's Theorem we get ,

   \[\text{Area}(\mathcal{R}) = \int_{\mathcal{R}} \dd A = \int_{\mathcal{C}} x \dd y = \int_{\mathcal{C}} -y \dd x = \int_{\mathcal{C}} \frac{x \dd y -y \dd x}{2} \]

\end{tcolorbox}

\begin{Eg}{Area inside the ellipse: $\frac{x^2}{a^2} + \frac{y^2}{b^2} = 1$}{}
    
\textit{Solution.} Parametrisation of ellipse $x = a \cos t , y = b \sin t$ where $t \in [0,2\pi)$. Using the above application of Green's Theorem we can write, 
\[\text{Area} = \int_c x \dd y = ab \int_{0}^{2\pi} \cos^2t \hspace{0.1cm}\dd t = \pi ab \]
$\hfill \blacksquare$
\end{Eg}

\begin{Thm}{Independence of path}{}
   Let $\vec{F}$ be a $C^1$ vector field on $\R^2$ such that $\int_c \vec{F} \cdot \,d \vec{r}$ is independent of path. Then $\vec{F}$ is conservative over an open and simply connected domain.
\end{Thm}

\textit{Proof.} Let, $\mathcal{D}$ be an open and connected domain. $\vec{F} = \inp{P}{Q}$ is defined over $\mathcal{D}$. Also let, $P_0 = \inp{x_0}{y_0}$ be a fixed point in the domain $\mathcal{D}$ and $P_1 = \inp{x}{y} \in \mathcal{D}$ be a variable point. $\mathcal{C}$ be a smooth curve joining $P_0$ and $P_1$. Define 

\[\varphi(x,y) = \inti{\mathcal{C}}{\vec{F} \cdot }{\vec{r}}\]

Since, $\mathcal{D}$ is open set so we must get an open ball centered at $P_1$ contained in $\mathcal{D}$. Take a point $P_1'= \inp{x_1}{y}$ inside that open ball such taht $x_1 < x$. Let, $c_1$ be a  smooth curve from $P_0$ to $P_1$ and $c_2$ be a line segment from $P_1'$ to $P_1$. So, $c_1 \cup c_2$ defines a smooth curve from $P_0$ to $P_1$.
\pagebreak

\[\begin{tikzcd}
	&& {P_1'(x_1,y)} && {P_1(x,y)} \\
	\\
	{P_0(x_0,y_0)}
	\arrow["{c_1}", curve={height=-18pt}, from=3-1, to=1-3]
	\arrow["{c_2}", from=1-3, to=1-5]
	\arrow["{\mathcal{C}}"', curve={height=18pt}, from=3-1, to=1-5]
\end{tikzcd}\]

As $\inti{\mathcal{C}}{\vec{F} \cdot }{\vec{r}}$ is path independent We can write, \[
    \varphi(x,y) = \inti{c_1}{\vec{F} \cdot }{\vec{r}} + \inti{c_2}{\vec{F} \cdot }{\vec{r}}
\]

Now we take the partial derivative of both sides of this equation with respect to $x$. The first integral does not depend on the variable $x$ since $c_1$ is the path from $P_0(x_0,y_0,z_0)$ to $P'_1(x_1,y,z)$ and so partial differentiating this line integral with respect to $x$ is zero.

\begin{align*}
    \pdv{\varphi}{x} &=\pdv{x} \left( \inti{c_1}{\vec{F} \cdot }{\vec{r}} + \inti{c_2}{\vec{F} \cdot }{\vec{r}} \right)  \\
    &= \underbrace{\pdv{x} \left( \inti{c_1}{\vec{F} \cdot }{\vec{r}}\right)}_{= 0}  + \pdv{x} \left(\inti{c_2}{\vec{F} \cdot }{\vec{r}} \right) 
\end{align*}

Now,$c_2$ can be parametrized as $r(t) = \inp{t}{y}$ where $t \in [x_1,x]$. So,

\begin{align*}
    \pdv{x} \left(\inti{c_2}{\vec{F} \cdot }{\vec{r}}\right) &= \pdv{x} \left( \int_{x_1}^x \inp{P(t,y)}{Q(t,y)} \cdot \inp{1}{0} \,d t \right) \\ 
    &= \pdv{x} \left( \int_{x_1}^x P(t,y) \,d t \right) \\
    &= P(x,y) \hspace{0.2cm} \text{[Fundamental Theorem of calculus]}
\end{align*}

Similarly we can show that, $\pdv{\varphi}{y} = Q(x,y)$. And hence,  $ \nabla \cdot \varphi = \vec{F(x,y)}$. We can define $\varphi$ as the potential of $\vec{f}$. $\hfill \blacksquare$

\begin{Thm}{}{}
    Let, $\mathcal{D}$ be a simply connected domain in $\R^2$ and $\vec{F}$ is a $C^1$ vector field on $\mathcal{D}$. Then $\vec{F}$ is conservative iff $\curl \vec{F} = 0$ on $\mathcal{D}$.
\end{Thm}

\textit{Proof.} ($\Rightarrow$) This direction is trivialy easy.

\vspace{0.2cm}

($\Leftarrow$) From Green's Theorem we can say that $ \inti{c}{\vec{F} \cdot }{\vec{r}} = 0$ over all closed curve $c$. For any two point $p_0,p_1 \in \mathcal{D}$ if $\gamma_1, \gamma_2 : [0,1] \to \mathcal{D}$ are two smmoth curves joining $p_0$ and $p_1$. (i.e $\gamma_1(0) = \gamma_2 (0) = p_0$ and $\gamma_1(1) = \gamma_2(1) = p_1$) then $\gamma_1 \cup \gamma_2(1-t)$ is a closed curve. So, $ \inti{\gamma_1}{\vec{F} \cdot }{\vec{r}}  = \inti{\gamma_2}{\vec{F} \cdot }{\vec{r}}$. Which means the integral is path independent. Using the previous theorem we can say, $\vec{F}$ is conservative on $\mathcal{D}$. $\hfill \blacksquare$

\begin{Def}{Divergence of a vector field}{}
    Given a vector field $\vec{F} = (f_1,\cdots, f_n) : \R^n \to \R^n $ , the ``Divergence'' of $\vec{F}$ is,

    \[\text{div}(F) = \sum_{i=1}^{n} \pdv{f_i}{x_i} \equiv \div \vec{F}\]
\end{Def}

\end{document}
