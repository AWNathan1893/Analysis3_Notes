\documentclass[../Analysis-3]{subfiles}

\begin{document}
\chapter*{Lecture 4} %Set chapter name
\addcontentsline{toc}{chapter}{Lecture 4} %Set chapter title
\setcounter{chapter}{4} %Set chapter counter
\setcounter{section}{0}

\section{More examples of Continuous maps}

\begin{Eg}{}{}
  Consider the function
  \[ f(x,y) =
    \begin{cases}
      \frac{xy}{\sqrt{x^2+y^2}}, \, (x,y) \neq (0,0) \\
      0, \, (x,y) = (0,0)
    \end{cases} \]

  $ f $ is clearly continuous on $ \R^2 \setminus \{(0,0)\} $ so we need only check the limit at $ (0,0) $. We have,
  \[ 0 \leq \abs{\frac{xy}{\sqrt{x^2+y^2}}} \leq \frac 12 \frac{x^2+y^2}{\sqrt{x^2+y^2}} = \frac 12 \norm{(x,y)} \]
  By the squeeze theorem, we get $ \displaystyle \lim_{(x,y) \to (0,0)}f(x,y) = 0 $, and hence, $ f $ is continuous on $ \R^2 $.
\end{Eg}

\textbf{Exercise.} Show that any linear map is continuous. (Hint: Use that the norm is continuous.)

\begin{Eg}{}{}
  Let $ D = \{(x,y) \in \R^2 \mid y \neq 0\} = \Pi_2^{-1}(\R \setminus \{0\})$. Consider the function
  \begin{align*}
    f : D  & \to \R            \\
    f(x,y) & = x \sin \frac 1y
  \end{align*}
  Clearly $ f $ is continuous on $ D $. We also have,
  \[ 0 \leq \abs{x \sin \frac 1y} \leq \norm{(x,y)} \]
  and hence, by the Squeeze theorem, $ \displaystyle \lim_{(x,y) \to (0,0)}f(x,y) = 0 $

  So, we can extend $ f $ to $ (0,0) $ continuously by defining it to be 0.
\end{Eg}

\section{Uniform Continuity}

\begin{Def}{Uniform Continuity}{}
  Let $ f : \mathcal{O}_n \to \R^m $, where $ \mathcal{O}_n $ is open in $ \R^n $. $ f $ is uniformly continuous if for all $ \veps > 0 $, there is $ \delta > 0 $ such that
  \begin{alignat*}{2}
    f(x) \in B_\veps(f(a))     & \quad            &  & \forall \, x \in B_\delta(a) \cap \cal O, \quad \forall \, a \in \cal O \\
                               & \Big\Updownarrow                                                                              \\
    \norm{f(x) - f(a)} < \veps & \quad            &  & \forall \, \norm{x - a} < \delta, \, a,x \in \cal O
  \end{alignat*}
\end{Def}

\textbf{Exercise.} Show that uniform continuity implies continuity.

\section{Derivatives}

\begin{notnBox}{}{}
  \begin{enumerate}[label = (\arabic*)]
    \item $ {\cal O}_n $ denotes an open set in $ \R^n $, and we omit the $ n $ if it is clear from context.
    \item A function $ f:\R^n \to \R^m $ has components $ f = (f_1, f_2, \dots, f_m) $
  \end{enumerate}
\end{notnBox}
\ssk

Recall the notion of derivative in $ \R $: \\
Let $ f : {\cal O}_1 \to \R $ be a function and $ a \in {\cal O}_1 $. Then $ f $ is differentiable at $ a $ if there is a real number $ \alpha (= f'(a)) $ such that
\[ \lim_{h \to 0}\frac{f(a+h)-f(a)}{h} = \alpha \label{star}\tag{1} \]

This clearly cannot be carried over verbatim to functions of several variables; we can't divide by a vector! To get a reasonable definition, we note that \eqref{star} is equivalent to,
\[ \lim_{h \to 0}\frac{f(a+h)-f(a)-\alpha h}{h} = 0 \]

Now, $ h \mapsto \alpha h $ is a linear map and we already know $ \{\text{linear maps on } \R\} \longleftrightarrow \R $. So the derivative $ f'(a) $ is really a linear map! This leads to the following definition.

\begin{Def}{Derivative in several variables}{}
  Let $ f : {\cal O}_n \to \R^m $ and $ a \in {\cal O}_n $. $ f $ is differentiable at $ a $ if there is a linear map $ L : \R^n \to \R^m $ such that
  \begin{alignat*}{2}
    \lim_{h \to 0} & \frac{f(a+h)-f(a)-Lh}{\norm{h}}            &  & = 0 \\
                   & \qquad \qquad \,\,\Big \Updownarrow                 \\
    \lim_{h \to 0} & \frac{\norm{f(a+h)-f(a)-Lh}}{\norm{h}}     &  & = 0 \\
                   & \qquad \qquad \,\, \Big \Updownarrow                \\
    \lim_{x \to a} & \frac{\norm{f(x)-f(a)-L(x-a)}}{\norm{x-a}} &  & = 0 \\
  \end{alignat*}
\end{Def}

\end{document}