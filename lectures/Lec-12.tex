\documentclass[../Analysis-3]{subfiles}
\usepackage{style}


\begin{document}
\chapter*{Lecture 12} %Set chapter name
\addcontentsline{toc}{chapter}{Lecture 12} %Set chapter title
\setcounter{chapter}{12} %Set chapter counter
\setcounter{section}{0}

\section{Hessian Matrix}

\begin{Def}{Hessian}{}
    Suppose $f : \Op{n} \to \R$ is $C^2$ at $a \in \Op{n} $. The $\textbf{Hessian of $f$ at $a$}$ is defined by, \[ H_f(a) = \left[ \pdv{f}{x_i}{x_j}\left( a \right) \right]_{n \times n} \]
\end{Def}

\[ \therefore H_{f} = \begin{bmatrix}
        \pdv[2]{f}{x_1}   & \pdv{f}{x_1}{x_2} & \ldots & \pdv{f}{x_1}{x_n} \\
        \pdv{f}{x_2}{x_1} & \pdv[2]{f}{x_2}   & \ldots & \pdv{f}{x_2}{x_n} \\
        \vdots            & \vdots            & \ddots & \vdots            \\
        \pdv{f}{x_n}{x_1} & \pdv{f}{x_n}{x_2} & \ldots & \pdv[2]{f}{x_n}
    \end{bmatrix} \]

$\bullet$ $H_f = {H_f}^T \quad (\text{i.e. symmetric})$

\begin{Eg}{}{}
    Let, $f : \R^2 \to \R$ be a function defined by $f(x,y) = \sin^2 x + x^2y + y^2$. Then,
    \[Df = \begin{bmatrix}
            \sin2x + 2xy & x^2 + 2y
        \end{bmatrix} :  \R^2 \to \R \quad \text{is linear.} \]
    Gradient = \[\nabla f = \left\langle \sin2x + 2xy, x^2 + 2y \right\rangle \in \R^2 \]

    \[ H_f = \begin{bmatrix}
            f_{xx} & f_{xy} \\
            f_{xy} & f_{yy}
        \end{bmatrix} = \begin{bmatrix}
            2( \cos2x + y) & 2x \\
            2x             & 2
        \end{bmatrix} \quad [ \because  f \in C_2] \]

\end{Eg}

\begin{notnBox}
    Given $A \in M_n(\R)$, $ \quad A = \left[a_{ij}\right] n \times n $
    \[ \forall\ x \in \R^n, Q_A(x) = \underbrace{x^tAx}_{\begin{bmatrix}
                x_1 & x_2 & \ldots & x_n
            \end{bmatrix} \begin{bmatrix}
                 &        & \\
                 & a_{ij} & \\
                 &        &
            \end{bmatrix} \begin{bmatrix}
                x_1    \\
                x_2    \\
                \vdots \\
                x_n
            \end{bmatrix}} \left( = {\left\langle Ax, x \right\rangle}_{\R^n} \right) \]

    \[ \implies Q_A(x) = \sum_{i,j = 1}^{n} a_{ij} \bar{x_i} x_j  \]
\end{notnBox}


\begin{Def}{Quadratic Form}{}
    A function $f : \R^n \to \R$ is a $\textbf{Quadratic Form}$ if $f(x) = Q_A(x) \ \forall\ x $ for some symmetric $A \in M_n(\R)$
\end{Def}

$\bullet$ A Quadratic Form is a homogeneous polynomial of degree 2

\vspace{3mm}

$\bullet$ $p(x,y) = a_1 x^2 + a_2 y^2 + a_{12} xy$
Hence, \[ A = \begin{bmatrix}
        a_1                & \frac{1}{2} a_{12} \\
        \frac{1}{2} a_{12} & a_2
    \end{bmatrix} \quad \implies p = Q_A \]

\begin{Def}{Positive Definite, Negative Definite, Semi Definite}{}
    A symmetric matrix $A \in M_n(\R)$ is \textbf{Positive Definite} if \[ \left\langle Ax, x \right\rangle > 0 \ \forall\ x \in \R^n \backslash \{0\} \]
    \vspace{2mm}
    A symmetric matrix $A \in M_n(\R)$ is \textbf{Negative Definite} if \[ \left\langle Ax, x \right\rangle < 0 \ \forall\ x \in \R^n \backslash \{0\} \]
    A symmetric matrix $A \in M_n(\R)$ is \textbf{Semi Definite} if \[ \left\langle Ax, x \right\rangle \geq 0 \ \forall\ x \in \R^n \backslash \{0\} \]
\end{Def}

\begin{Eg}{}{}
    \begin{enumerate}
        \item  $I_n$ is positive definite as $\left\langle I_nx, x \right\rangle = {\|x \|}^2 > 0 \quad \forall\ x \in \R^n \backslash \{0\}$
        \item $A = B^TB$ for some $B \in M_n{\R}$
              \begin{align*}
                  \left\langle Ax,x \right\rangle & = \left\langle B^TBx,x \right\rangle                                    \\
                                                  & = x^TB^TBx                                                              \\
                                                  & = {(Bx)}^T Bx = {\| Bx \|}^2 \quad \forall\ x \in \R^n \backslash \{0\}
              \end{align*}
              Hence, \[ \left\langle Ax,x \right\rangle \geq 0 \]

              Now, if $ \left\langle Ax,x \right\rangle = 0 $, then  $x$ is kernel of $B$. \\
              Now, if $A$ is positive definite, there is no such $x$, hence, columns of $B$ are linearly independent. \\
              In general, $A$ is \textbf{Positive Semi Definite} (``$\Longleftarrow$'' also holds)

        \item $\begin{bmatrix}
                      1 & 0  \\
                      0 & -1
                  \end{bmatrix} \longrightarrow Q_A = {x_1}^2 - {x_2}^2 \longrightarrow $, i.e., \textbf{indefinite}
        \item  $\begin{bmatrix}
                      1 & 0 \\
                      0 & 0
                  \end{bmatrix} \longrightarrow Q_A = {x_1}^2 \geq 0 \longrightarrow $, i.e., \textbf{Positive Semi Definite}

    \end{enumerate}
\end{Eg}

$\bullet \hspace{2mm}$ Now,
\begin{align*}
     & \left\langle Ah, h \right\rangle = \norm{Ah}\norm{h}\cos{\theta} > 0 \\
     & \implies \boxed{0 \leq \theta \leq \frac{\pi}{2}}
\end{align*}

$\bullet \hspace{2mm} $ In general it is difficult to classify positive definite $n \times n$ matrices. But not for $n = 2$.

\begin{Thm}{}{}
    Let $A = \begin{bmatrix}
            a & b \\
            b & c
        \end{bmatrix} \in M_2(\R) $ be symmetric. Then,
    \begin{enumerate}
        \item $A$ is \textbf{Positive Definite} $ \Longleftrightarrow a > 0 \text{ and } ac -b^2 >0 $
        \item $A$ is \textbf{Negative Definite} $ \Longleftrightarrow a < 0 \text{ and } ac -b^2 >0 $
        \item $A$ is \textbf{Indefinite} $ \Longleftrightarrow ac -b^2 <0 $
    \end{enumerate}
\end{Thm}

\begin{proof}
    We have, \[ \left\langle A, h \right\rangle = h^TAh \]
    Now, scaling $h$ is okay as sign would not be changed. So, pick ${\bf x} = (x_1, x_2) \in \R^2 \backslash \{(0,0)\} $ with $x_2 \neq 0$.
    Without loss of generality, take ${\bf x} = (x,1) \ x \in \R$ (Scaling) \\
    \[ \therefore \left\langle Ax, x \right\rangle = ax^2 + 2bx + c > 0 \ \forall\ x \in \R \]
    Now, if $x_2 = 0$ again, Without loss of generality assume ${\bf x} = \begin{bmatrix}
            1 \\
            0
        \end{bmatrix}$ (Scaling). Then, \[ \left\langle Ax, x \right\rangle = a \]

    \begin{align*}
        \therefore          & \ A \text{ is Positive Definite}                         \\
        \Longleftrightarrow & \ a > 0 \text{ and } ax^2 + bx+c > 0 \ \forall\ x \in \R \\
        \Longleftrightarrow & \ a > 0 \text{ and } (2b)^2 - 4ac < 0                    \\
        \Longleftrightarrow & \ a > 0 \text{ and } ac - b^2 > 0
    \end{align*}

    Similarly,
    \begin{align*}
        \therefore          & \ A \text{ is Negative Definite}                         \\
        \Longleftrightarrow & \ a < 0 \text{ and } ax^2 + bx+c < 0 \ \forall\ x \in \R \\
        \Longleftrightarrow & \ a < 0 \text{ and } (2b)^2 - 4ac < 0                    \\
        \Longleftrightarrow & \ a < 0 \text{ and } ac - b^2 > 0
    \end{align*}

    And,
    \begin{align*}
        \therefore          & \ A \text{ is Indefinite}                                                                              \\
        \Longleftrightarrow & \ ax^2 + bx+c < 0 \ \text{for some } x \in \R \text{ and } ax^2 + bx+c > 0 \ \text{for some } x \in \R \\
        \Longleftrightarrow & \ (2b)^2 - 4ac > 0                                                                                     \\
        \Longleftrightarrow & \ ac - b^2 < 0
    \end{align*}

    \textbf{Exercise.} Prove No (2) and No (3) .
\end{proof}

\begin{Lem}{}{}
    Let, $a \in \Op{n}, A(x) = \begin{bmatrix}
            a_1(x) & a_2(x) \\
            a_2(x) & a_3(x)
        \end{bmatrix}$
    Suppose, $A$ is continuous at $a \ \forall\ x \in \Op{n}$ (i.e., $a_i$'s are continuous at $a$). Then, $A$ is Positive Definite at $a \implies A$ is Positive Definite in a neighborhood of $a$.
\end{Lem}

\begin{proof}
    \begin{align*}
                 & A(a) \hspace{1mm} \text{is Positive Definite}                                    \\
        \implies & a_1(a) > 0 \hspace{1mm} \text{ and } \hspace{1mm} a_1(a) a_3(a) - {a_2}^2(a) > 0 \\
        \implies & \exists\ B_{\epsilon} (a) \hspace{1mm} \text{such that both are positive}.
    \end{align*} We are done!
\end{proof}





\end{document}