\documentclass[../Analysis-3.tex]{subfiles}

\begin{document}
\chapter*{Lecture 12} %Set chapter name
\addcontentsline{toc}{chapter}{Lecture 12} %Set chapter title
\setcounter{chapter}{12} %Set chapter counter
\setcounter{section}{0}

\section{Hessian Matrix}

\begin{Def}{Hessian}{hessian}
  Suppose $f : \Op{n} \to \R$ is $C^2$ at $a \in \Op{n} $. The $\textbf{Hessian of $f$ at $a$}$ is defined by, \[ H_f(a) = \left( \pdv{f}{x_i}{x_j}\left( a \right) \right)_{n \times n} \]
\end{Def}

\[ H_{f} = \begin{pmatrix}
    \pdv[2]{f}{x_1}   & \pdv{f}{x_1}{x_2} & \ldots & \pdv{f}{x_1}{x_n} \\[5pt]
    \pdv{f}{x_2}{x_1} & \pdv[2]{f}{x_2}   & \ldots & \pdv{f}{x_2}{x_n} \\
    \vdots            & \vdots            & \ddots & \vdots            \\
    \pdv{f}{x_n}{x_1} & \pdv{f}{x_n}{x_2} & \ldots & \pdv[2]{f}{x_n}
  \end{pmatrix} \]

Note that for any $ f \in C^n $, its Hessian is a symmetric matrix, i.e., $H_f = {H_f}^t$.

\begin{Eg}{}{}
  Let, $f : \R^2 \to \R$ be a function defined by $f(x,y) = \sin^2 x + x^2y + y^2$. Then,
  \[Df = \begin{pmatrix}
      \sin2x + 2xy & x^2 + 2y
    \end{pmatrix} :  \R^2 \to \R \quad \text{is linear.} \]
  Gradient, \[\grad f = \left\langle \sin2x + 2xy, x^2 + 2y \right\rangle \in \R^2 \]

  \[ H_f = \begin{pmatrix}
      f_{xx} & f_{xy} \\
      f_{xy} & f_{yy}
    \end{pmatrix} = \begin{pmatrix}
      2( \cos2x + y) & 2x \\
      2x             & 2
    \end{pmatrix} \quad [ \because  f \in C^2] \]

\end{Eg}

\textbf{Now we will introduce some notation.} Given $A = \left( a_{ij} \right)_{n\times n} \in M_n(\R)$ and $ x \in \R^n $ we denote $ Q_A(x) $  by,

\begin{align*}
  Q_A(x) = x^tAx = {\left\langle Ax,x \right\rangle}_{\R^n}
         & = \begin{pmatrix}
               x_1 & x_2 & \ldots & x_n
             \end{pmatrix} \begin{pmatrix}
                              &   & \\
                              & A & \\
                              &   &
                           \end{pmatrix} \begin{pmatrix}
                                           x_1 & x_2 & \ldots & x_n
                                         \end{pmatrix} ^t \\
  Q_A(x) & = \sum_{i,j = 1}^{n} a_{ij} x_i x_j
\end{align*}



\begin{Def}{Quadratic Form}{quadform}
  A function $f : \R^n \to \R$ is called a $\textbf{Quadratic Form}$ if $f(x) = Q_A(x) \ \forall\ x $ and for some symmetric $A \in M_n(\R)$
\end{Def}

A Quadratic Form is a homogeneous polynomial of degree 2.

If $p(x,y) = a_1 x^2 + a_2 y^2 + a_{12} xy$
is a bivariate homogeneous polynomial of degree 2 then taking, $A = \begin{pmatrix}
    a_1                & \frac{1}{2} a_{12} \\
    \frac{1}{2} a_{12} & a_2
  \end{pmatrix}$ we get $p = Q_A$.

\begin{Def}{Positive Definite, Negative Definite, Semi Definite}{}
  \begin{itemize}
    \item A symmetric matrix $A \in M_n(\R)$ is called \textbf{Positive Definite} if \[ \left\langle Ax, x \right\rangle > 0 \ \forall\ x \in \R^n \backslash \{0\} \]
    \item A symmetric matrix $A \in M_n(\R)$ is called \textbf{Negative Definite} if \[ \left\langle Ax, x \right\rangle < 0 \ \forall\ x \in \R^n \backslash \{0\} \]
    \item A symmetric matrix $A \in M_n(\R)$ is called \textbf{Semi Definite} if \[ \left\langle Ax, x \right\rangle \geq 0 \ \forall\ x \in \R^n \backslash \{0\} \]
  \end{itemize}
\end{Def}

\begin{Eg}{}{}
  \begin{enumerate}
    \item  $I_n$ is positive definite as $\left\langle I_nx, x \right\rangle = {\|x \|}^2 > 0 \quad \forall\ x \in \R^n \backslash \{0\}$
    \item $A = B^tB$ for some $B \in M_n{\R}$
          \begin{align*}
            \left\langle Ax,x \right\rangle
             & = \left\langle B^tBx,x \right\rangle                                    \\
             & = x^tB^tBx                                                              \\
             & = {(Bx)}^t Bx = {\| Bx \|}^2 \quad \forall\ x \in \R^n \backslash \{0\}
          \end{align*}
          Hence, \[ \left\langle Ax,x \right\rangle \geq 0 \]

          Now, if $ \left\langle Ax,x \right\rangle = 0 $, then  $x$ is kernel of $B$. \\
          Now, if $A$ is positive definite, there is no such $x$, hence, columns of $B$ are linearly independent. \\
          In general, $A$ is \textbf{Positive Semi Definite} (``$\Longleftarrow$'' also holds)

    \item $\begin{pmatrix}
              1 & 0  \\
              0 & -1
            \end{pmatrix} \longrightarrow Q_A = {x_1}^2 - {x_2}^2 \longrightarrow $, i.e., \textbf{indefinite}
    \item  $\begin{pmatrix}
              1 & 0 \\
              0 & 0
            \end{pmatrix} \longrightarrow Q_A = {x_1}^2 \geq 0 \longrightarrow $, i.e., \textbf{Positive Semi Definite}

  \end{enumerate}
\end{Eg}

For a positive definite matrix $ A $,
\begin{align*}
   & \left\langle Ah, h \right\rangle = \norm{Ah}\norm{h}\cos{\theta} > 0 \\
   & \implies \boxed{0 \leq \theta \leq \frac{\pi}{2}}
\end{align*}

\

In general, it is difficult to classify positive definite $n \times n$ matrices. But not for $n = 2$.

\begin{Thm}{}{}
  Let $A = \begin{pmatrix}
      a & b \\
      b & c
    \end{pmatrix} \in M_2(\R) $ be symmetric. Then,
  \begin{enumerate}[label=(\roman*)]
    \item $A$ is \textbf{Positive Definite} $ \Longleftrightarrow a > 0 \text{ and } ac -b^2 >0 $ \label{charpd}
    \item $A$ is \textbf{Negative Definite} $ \Longleftrightarrow a < 0 \text{ and } ac -b^2 >0 $ \label{charnd}
    \item $A$ is \textbf{Indefinite} $ \Longleftrightarrow ac -b^2 <0 $ \label{charind}
  \end{enumerate}
\end{Thm}

\begin{proof}
  We have, \[ \inp{A}{h} = h^tAh \]
  Now, scaling $h$ is okay as sign would not be changed. So, pick ${\bf x} = (x_1, x_2) \in \R^2 \backslash \{(0,0)\} $ with $x_2 \neq 0$.
  Without loss of generality, take ${\bf x} = (x,1) \ x \in \R$ (Scaling) \\
  \[ \therefore \inp{A\bf{x}}{\bf{x}} = ax^2 + 2bx + c > 0 \ \forall\ x \in \R \]
  Now, if $x_2 = 0$ again, Without loss of generality assume ${\bf x} = \begin{pmatrix}
      1 \\
      0
    \end{pmatrix}$ (Scaling). Then, \[ \inp{A\bf{x}}{\bf{x}} = a \]
  So,
  \begin{align*}
                        & \ A \text{ is Positive Definite}                         \\
    \Longleftrightarrow & \ a > 0 \text{ and } ax^2 + bx+c > 0 \ \forall\ x \in \R \\
    \Longleftrightarrow & \ a > 0 \text{ and } (2b)^2 - 4ac < 0                    \\
    \Longleftrightarrow & \ a > 0 \text{ and } ac - b^2 > 0
  \end{align*}

  Similarly,
  \begin{align*}
                        & \ A \text{ is Negative Definite}                         \\
    \Longleftrightarrow & \ a < 0 \text{ and } ax^2 + bx+c < 0 \ \forall\ x \in \R \\
    \Longleftrightarrow & \ a < 0 \text{ and } (2b)^2 - 4ac < 0                    \\
    \Longleftrightarrow & \ a < 0 \text{ and } ac - b^2 > 0
  \end{align*}

  And,
  \begin{align*}
                        & \ A \text{ is Indefinite}                                                                              \\
    \Longleftrightarrow & \ ax^2 + bx+c < 0 \ \text{for some } x \in \R \text{ and } ax^2 + bx+c > 0 \ \text{for some } x \in \R \\
    \Longleftrightarrow & \ (2b)^2 - 4ac > 0                                                                                     \\
    \Longleftrightarrow & \ ac - b^2 < 0
  \end{align*}
\end{proof}

\begin{Lem}{}{hessian:nbd}
  Let, $a \in \Op{n},\ A(x) = \begin{pmatrix}
      a_1(x) & a_2(x) \\
      a_2(x) & a_3(x)
    \end{pmatrix}$. Suppose, $A$ is continuous at $a \ \forall\ x \in \Op{n}$ (i.e., $a_i$'s are continuous at $a$). Then, $A$ is Positive Definite at $a \implies A$ is Positive Definite in a neighborhood of $a$.
\end{Lem}

\begin{proof}
  $ A(a) $ is Positive Definite, i.e., $ a_1(a) > 0 $ and $ a_1(a) a_3(a) - {a_2}^2(a) > 0 $. As $ a_1(x) $ and $ a_1(x)a_3(x) - a_2^2(x) $ are polynomial of continuous functions, we can find an $ \epsilon > 0 $ such that both are positive in $ B_{\epsilon}(a) $.
\end{proof}


\end{document}