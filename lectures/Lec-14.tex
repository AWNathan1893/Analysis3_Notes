\documentclass[a4paper]{article} 
\usepackage[utf8]{inputenc}%remove
\usepackage{../style}%remove
%\documentclass[../Analysis-3.tex]{subfiles}

\begin{document}
%\chapter*{Lecture 14} %Set chapter name
%\addcontentsline{toc}{chapter}{Lecture 14} %Set chapter title
%\setcounter{chapter}{14} %Set chapter counter
%\setcounter{section}{0}

\section{Compact subsets of $\mathbb{R}^n$}

\begin{Def}{Compact Subset}{}
  A subset $K \subseteq \mathbb{R}^n$ is said to be compact if every sequence $\{x_n\} \subseteq K$ has a subsequence $\{x_{n_k}\}$ that is convergent to some $x \in K$.
\end{Def}

This is known as the Bolzano-Weierstrass Property.

Observe that a compact subset of $\mathbb{R}^n$ is always closed. To see this, note that every sequence $\{x_n\} \subseteq K$, where $K$ is a compact subset of $\mathbb{R}^n$ that converges to some $x \in \mathbb{R}^n$ has a convergent subsequence $\{x_{n_k}\}$ that converges to the same $x$. Since $K$ is compact, we can say that $x \in K$. So, the convergent sequence $\{x_n\}$ converges to a point in $K$. Hence, $K$ is closed.



\end{document}