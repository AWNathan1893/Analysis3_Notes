\documentclass[../Analysis-3.tex]{subfiles}

\begin{document}
\chapter*{Lecture 7} %Set chapter name
\addcontentsline{toc}{chapter}{Lecture 7} %Set chapter title
\setcounter{chapter}{7} %Set chapter counter
\setcounter{section}{0}
\setcounter{equation}{0}
\setcounter{figure}{0}


\section{Schwarz Theorem}

In the previous lecture, we discussed the notion of partial derivatives. In general, the partial derivatives depend on the order of differentiation. However, using Clairaut's Theorem, we found a necessary when $f_{xy} = f_{yx}$. Now, we conclude that discussion by the following result.

\begin{Thm}{Schwarz}{7:1}
  Let $(a,b) \in \mathcal{O}_2$ and $f:\mathcal{O}_2 \to \R$. Suppose $f_x$, $f_y$, and $f_{xy}$ exist on $\mathcal{O}_2$. If $f_{xy}$ is continuous at $(a,b)$, then $f_{yx}$ exists in a neighbourbood of $(a,b)$ and $f_{xy}(a,b) = f_{yx}(a,b)$.
\end{Thm}

\begin{proof}
  Just as before, we take $(a,b) = (0,0)$. From the proof of Clairaut's Theorem, we have $F(h,k) = f_{xy}(c_1, c_2)$ for some $0 < c_1< h$ and $0 < c_2 < k$. By continuity of $f_{xy}$ at $(0,0)$, for any $\epsilon >0$ there exists $h_\epsilon, k_\epsilon > 0$, such that
  \[
    \mid f_{xy}(u,v)- f_{xy}(0,0)\mid < \epsilon \;\;\forall \; (u,v) \in [0,h_\epsilon]\times[0,k_\epsilon]
  \]
  
  But then $\mid F(h,k) - f_{xy}(0,0) \mid < \epsilon \;\;\forall \; (u,v) \in [0,h_\epsilon]\times[0,k_\epsilon]$, that is, $F$ is continuous at $(0,0)$ with limit $f_{xy}(0,0)$ at $(0,0)$.

  As $f_y(h,0)$ exists for $h$ sufficiently small, $\lim_{k \to 0} F(h,k)$ exists for $h$ sufficiently small. Thus, by continuity of $F$ at $(0,0)$,
  \[\lim_{h \to 0}\lim_{k \to 0} F(h,k) \text{  exists and is equal to  }\lim_{(h,k) \to (0,0)} F(h,k)\]
  Thus, $f_{yx}(0,0)$ exists and is equal to $f_{xy}(0,0)$
\end{proof}

\msk

This is a slightly more useful version of Clairaut's Theorem. However, in many applications (say, partial differential equations), we work with $C^2$ or even $C^\infty$ functions, in which case both of these hold automatically.

\msk

\textbf{Exercise:} Formulate and prove a similar result for higher order derivatives. In particular, provide a sufficient condition for $f: \mathcal{O}_n \to \R$ so that
\[\frac{\partial^n f}{\partial x_{i_1} \partial x_{i_2} \dots \partial x_{i_m}} = \frac{\partial^n f}{\partial x_{\sigma(i_1)} \partial x_{\sigma(i_2)} \dots \partial x_{\sigma(i_m)}}\]
over $\mathcal{O}_n$ for any permutation $\sigma$ of the elements $\{i_1,i_2,\dots,i_n\}$.

\section{Partial and Total Derivatives}

We will now see that the partial derivatives provide an effective way of proving the existence and computing the total derivatives of a function $f: \mathcal{O}_n \to \R^m$. In this lecture and the next, we will develop the relations between partial and total derivatives by a series of results.

\begin{Def}{Jacobian Matrix}{}
  For a function $f = (f_1, f_2, \dots, f_m): \mathcal{O}_n \to \R^m$, if all the partial derivatives $\pdv{f_i}{x_j}$ at $a \in \mathcal{O}_n$, we define the Jacobian of the function at $a$ by the $m \times n$ matrix,
  \[J_f(a) = \left( \pdv{f_i}{x_j}\/(a) \right)_{m \times n}\]
\end{Def}

\begin{Thm}{}{7:2}
  Consider a function $f = (f_1, f_2, \dots, f_m): \mathcal{O}_n \to \R^m$ differentiable at $a \in \mathcal{O}_n$. Then all the partial derivatives $\pdv{f_i}{x_j}$ exist at $a$. In particular, for $f$ differentiable at $a$, we have,
  \[(Df)(a) = J_f(a) = \left( \pdv{f_i}{x_j} \/ (a)\right)_{m \times n}\]
\end{Thm}

\begin{proof}
  Without loss of generality, we take $m = 1$, and let $a = (a_1, a_2, \dots, a_n)$. Fix an arbitrary index $i \in \{1,2,\dots,n \}$. We define $\eta_i : [a_i - \epsilon,a_i + \epsilon] \to \R^n$, defined by
  \[\eta_i(t) = (a_1, \dots, a_{i-1}, t, a_{t+1}, \dots, a_n) = a + (t-a_i)e_i \]

  As $\mathcal{O}_n$ is open and $\eta_i$ is continuous, we can find $\epsilon$ small such that $f([[a_i - \epsilon,a_i + \epsilon]]) \subseteq \mathcal{O}_n \subseteq \R^n$. Evidently, $\eta_i$ is differentiable and $(D\eta_i) = [0,\dots,1,\dots,0]^t = e_i^t$ over $[a_i - \epsilon,a_i + \epsilon]$. Now, by the definition of partial derivatives, \(D(f\circ\eta_i)(a_i) = f_{x_i} (a)\).

  Again, by chain rule, as $f$ is differentiable at $a$, $D(f\circ\eta_i)(a_i) = f_{x_i}(a)$ exists, and
  \begin{align*}
    D(f\circ\eta_i)(a_i) & = Df(\eta_i(a_i)) \cdot D\eta_i (a_i) \\
    \implies f_{x_i} (a) & = Df(a) \cdot e_i^t = [Df(a)]_i
  \end{align*}

  As the index $i$ was arbitrary to begin with, this completes the proof.
\end{proof}

This theorem proves that differentiability of a function implies the existence of its partial derivatives, and gives the form of the derivative in the standard basis. But it is often quite elaborate and laborious to prove that a function is differentiable, whereas computation of the partial derivatives is much more straightforward. In the next lecture, we formulate a sufficient condition for differentiability of a function based on its partial derivatives.


\end{document}
