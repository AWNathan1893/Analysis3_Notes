\documentclass[../Analysis-3]{subfiles}

\begin{document}
\chapter*{Lecture 1} %Set chapter name
\addcontentsline{toc}{chapter}{Lecture 1} %Set chapter title
\setcounter{chapter}{1} %Set chapter counter
\setcounter{section}{0}

\section{Introduction}
We will talk about $ n $-variable calculus, that is, calculus on $ \R^n $. Recall the the following:
\begin{enumerate}[label = $\bullet$]
      \item The setting is,
            \[ \R^n = \underbrace{\R \times \R \cdots \times \R}_{n \text{times}} = \{x = (x_1, \dots, x_n) \mid x_i \in \R \, \forall \, i = 1,2, \dots, n\} \]
      \item Analysis on $ \R $ consisted of ideas like open sets, compact sets, convergence, limits, differentiability, integrability etc.
      \item $ \R^n $ is a $ n $-dimensional inner product space over $ \R $, with the standard orthonormal basis $ \{e_j\}_{j=1}^n $
\end{enumerate}

Extending the analytic ideas to $ \R^n $ exploiting the algebraic structure is the matter of this course, which further gives way to differential geometry.

\section{Review: $ \R^n $ as a vector space}

\begin{enumerate}[label = (\roman*)]
      \item The standard orthonormal basis of $ \R^n $ is $ \{e_i\}_{i=1}^n $.
      \item For all $ x \in \R^n $, there is a unique representation
            \[ x = \sum_{i=1}^n x_i e_i, \, x_i \in \R \]
            Thus we identify $ x $ with the \textit{coordinates} $ (x_1,x_2, \dots, x_n) $.
      \item Euclidean inner product on $ \R^n $:

            For $ x = (x_1, \dots, x_n), y = (y_1, \dots, y_n) \in \R^n $, we define
            \[ \inp xy = \sum_{i=1}^{n} x_iy_i \]
\end{enumerate}

\section{Linear Functions}

In doing analysis on $ \R $, the main motive was to study functions $ f: \R \to \R $ and their properties, namely continuity, differentiability, integrability etc. We now wish to do the same for functions $ f: \R^n \to \R^m $ for arbitrary natural numbers $ n,m $.

Two easy examples of such functions are:
\begin{enumerate}[label = (\roman*)]
      \item Constant maps
            \begin{align*}
                  f: \R^n                     & \to \R^m           \\
                  \forall \, x \in \R^n, f(x) & = a, \, a \in \R^m
            \end{align*}

      \item Linear maps

            A function $ L: \R^n \to \R^m $ is linear if for all $ \alpha \in \R, x,y \in \R^n $, $ L(\alpha x + y) = \alpha L(x) + L(y) $.
\end{enumerate}

It turns out that linear maps are useful in understanding most other `\textit{nice}' functions, and so we now look at these in more detail.
\ssk

Let $ L $ be a linear map from $ \R^n $ to $ \R^m $. Consider the domain $ \{\alpha x + y \mid \alpha \in \R \} $, that is, the line through $ y $ in the direction of $ x $. The image under $ L $ is,
\[ \{ \alpha Lx + Ly \mid \alpha \in \R\} \]
which is the line through $ Ly $ in the direction of $ Lx $. Hence, $ L $ maps lines to lines.

\textbf{Exercise.} Is the converse also true?

\subsection*{Matrix representation of a linear map}

Let $ L: \R \to \R $ be a linear map. Then,
\[ L(x) = x L(1) \quad \forall \, x \in \R \]
Therefore,
\[ \mathcal L (\R, \R) := \{\text{set of all linear maps from } \R \text{ to } \R\} \leftrightarrow \R \]

Now consider the general case; let $ L: \R^n \to \R^m $ be a linear map. If we fix the bases $ \{e_j\}_{j=1}^n $ of $ \R^n $ and $ \{e_i\}_{i=1}^m $ of $ \R^m $, $ L $ is determined uniquely by the equations
\[ Le_j = \sum_{i=1}^{m} a_{ij}e_i \]
and hence,
\[ L \leftrightarrow (a_{ij})_{m \times n} \in M_{m,n}(\R) \]

\section{Analytic ideas in $ \R^n $}

We have the Euclidean norm on $ \R^n $ defined by,
\[ \norm{x} = \left(\sum_{i=1}^{n}x_i^2\right)^{\frac 12} \quad \forall \, x \in \R^n \]
This induces the metric given as,
\[ d(x,y) = \norm{x-y} = \left(\sum_{i=1}^{n}(x_i-y_i)^2\right)^{\frac 12} \quad \forall \, x,y \in \R^n \]

\begin{Thm}{Cauchy-Schwarz Inequality}{}
      For all $ x,y \in \R^n $, \[ \inp xy \leq \norm{x} \norm{y} \]
\end{Thm}
\begin{proof}
      Consider $ x,y \in \R^n $. We have,
      \[ \sum_{i=1}^{n}\sum_{j=1}^{n} (x_iy_j - x_jy_i)^2 \geq 0 \]
      But the left hand side is, after expanding,
      \[ \sum_{i,j = 1}^{n} x_i^2y_j^2 + \sum_{i,j = 1}^{n} x_j^2y_i^2 - 2 \sum_{i,j=1}^{n} x_ix_j y_iy_j = 2 \norm{x}^2\norm{y}^2 - 2\inp xy^2 \]
      which gives the desired inequality.
\end{proof}

\textbf{Note:} The proof shows that equality holds only if there is $ \lambda \in \R $ such that for all $ i $, either $ x_i = \lambda y_i $ or $ y_i = \lambda x_i $.
\msk

Recall the triangle inequality for $ \R $, for all $ x,y \in \R $
\[ \abs{x+y} \leq \abs{x} + \abs{y} \]

\begin{Thm}{Triangle inequality for $ \R^n $}{}
      For all $ x,y \in \R^n $,
      \[ \norm{x+y} \leq \norm{x} + \norm{y} \]
\end{Thm}
\begin{proof}
      We have,
      \begin{align*}
            \norm{x+y}^2 & = \inp{x+y}{x+y}                                                                 \\
                         & = \norm{x}^2 + 2\inp xy + \norm{y}^2                                             \\
                         & \leq \norm{x}^2 + \norm{y}^2 +2 \norm{x}\norm{y} \tag{Cauchy Schwarz inequality} \\
            \implies \norm{x+y} \leq \norm{x} + \norm{y}
      \end{align*}
      which is the desired inequality.
\end{proof}
\msk

The following is a technical result, which can be thought of as an analogue of the Lipschitz condition for linear maps on $ \R^n $. It hints towards continuity of linear maps, and we will see that it is indeed so later.
\ssk

\begin{Thm}{}{LinLip}
      Let $ L: \R^n \to \R^m $ be a linear map. There is $ M > 0 $ such that
      \[ \norm{Lx} \leq M \norm{x} \quad \forall \, x \in \R^n \]
\end{Thm}
\begin{proof}
      We have, for $ x = \sum_{i=1}^{n}x_ie_i $,
      \begin{align*}
            \norm{Lx}          & = \norm{\sum_{i=1}^{n} x_i Le_i}                                                                  \\
                               & \leq \sum_{i=1}^{n} |x_i| \norm{Le_i} \tag{Triangle inequality}                                   \\
            \implies \norm{Lx} & \leq \norm{x} \left(\sum_{i=1}^{n}\norm{Le_i}^2\right)^{\frac 12} \tag{Cauchy Schwarz inequality}
      \end{align*}
      Taking $ M = \left(\sum_{i=1}^{n}\norm{Le_i}^2\right)^{\frac 12} $, we get the result.
\end{proof}

\end{document}