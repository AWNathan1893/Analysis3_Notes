\documentclass[../Analysis-3]{subfiles}

\begin{document}
\chapter*{Lecture 5} %Set chapter name
\addcontentsline{toc}{chapter}{Lecture 5} %Set chapter title
\setcounter{chapter}{5} %Set chapter counter
\setcounter{section}{0}

We have figured out that the reasonable definition of the derivative of a map $ f: \Op{n} \to \R^m $ at a point $ a \in \Op{n} $ is $ Df(a) = L $ where
\[ \lim_{h \to 0}  \frac{f(a+h)-f(a)-Lh}{\norm{h}} = 0 \]
We now proceed to perform a series of reductions that will actually answer the question of \\ 
differentiability of functions and also provide techniques to compute the derivative.

\section{Matrix representation of the derivative}

\begin{Thm}{}{}
  Let $ f: \Op{n} \to \R^m $ and $ a \in \Op{n} $. Then, $ f $ is differentiable at $ a $ iff $ f_i $ is differentiable at $ a $ for all $ i $. In that case, we have
  \[ Df(a) =
    \begin{bmatrix}
      Df_1(a) \\
      Df_2(a) \\
      \vdots  \\
      Df_m(a)
    \end{bmatrix} \]
\end{Thm}

\begin{proof}
  Assume that $ f $ is differentiable at $ a $ and let $ L = Df(a) $. For all $ i = 1,2,\dots,m $ we set $ L_i = \Pi_i \circ L $. Further, let $ \tilde{f}_i(h) = f_i(a+h)-f_i(a)-L_ih $, so that
  \[ f(a+h)-f(a)-Lh = (\tilde{f}_1(h), \tilde{f}_2(h), \dots, \tilde{f}_m(h)) \]
  But then, $ \abs{\tilde{f}_i(h)} \leq \norm{f(a+h)-f(a)-Lh} $ which proves that $ f_i $ is differentiable at $ a $ and has derivative $ Df_i(a) = L_i $, for all $ i $.
  \msk
  
  Assume that $ f_i $ is differentiable at $ a $ for all $ i $ and define
  \[ L =
    \begin{bmatrix}
      Df_1(a) \\
      Df_2(a) \\
      \vdots  \\
      Df_m(a)
    \end{bmatrix} \]
  Let $ \Pi_i(f(a+h)-f(a)-Lh) = f_i(a+h)-f_i(a)-Df_i(a)h =  \tilde{f}_i(h) $, so that
  \begin{align*}
    \lim_{h \to 0} \frac{\tilde{f}_i(h)}{\norm{h}}          & = 0 \quad \forall \, i \\
    \implies \lim_{h \to 0} \frac{f(a+h)-f(a)-Lh}{\norm{h}} & = 0
  \end{align*}
  which proves that $ f $ is differentiable at $ a $ and has derivative $ Df(a) = L $.
\end{proof}

\begin{noteBox}
  \begin{enumerate}[label = $\bullet$]
    \item Because of this theorem, it is enough to study differentiability of maps $ f: \Op{n} \to \R $.
    \item Let $ \gamma: \Op{1} \to \R^n $. Then, $ \gamma $ is differentiable at $ a \in \Op{1} $ iff $ \gamma_i $ is differentiable at $ a $ for all $ i $ and in that case,
          \[ D\gamma(a) = \gamma'(a) =
            \begin{bmatrix}
              \gamma_1'(a) \\
              \gamma_2'(a) \\
              \vdots       \\
              \gamma_n'(a)
            \end{bmatrix} \]
          This is the notion of velocity vector of a curve in $ \R^n $ from elementary calculus.
  \end{enumerate}
\end{noteBox}

\section{Further properties of differentiable functions}
\begin{Thm}{Differentiability implies continuity}{}
  Let $ f: \Op{n} \to \R^m $. If $ f $ is differentiable at $ a \in \Op{n} $, $ f $ is continuous at $ a $.
\end{Thm}

\begin{proof}
  Let $ f $ be differentiable at $ a $. Then, for all $ x \in \Op{n} $,
  \[ 0 \leq \norm{f(x) - f(a)} \leq \norm{f(x) - f(a) - Df(a)(x-a)} + \norm{Df(a)(x-a)} \]
  Taking the limit as $ x \to a $, the first term on the right goes to 0 by the definition of $ Df(a) $ and the second term goes to zero as any linear map is continuous. Hence, by the squeeze theorem,
  \[ \lim_{x \to a}\norm{f(x)-f(a)} = 0  \]
  which proves that $ f $ is continuous at $ a $.
\end{proof}

\begin{Thm}{Chain rule}{}
  Consider maps $ f,g $ such that
  \[\begin{tikzcd}
      {\Op{n}} &&& {\Op{m}} &&& {\R^p}
      \arrow["f", from=1-1, to=1-4]
      \arrow["g", from=1-4, to=1-7]
      \arrow["{g \circ f}"', curve={height=30pt}, from=1-1, to=1-7]
    \end{tikzcd}\]
  Assume that $ f $ is differentiable at $ a \in \Op{n} $ and $ g $ is differentiable at $ f(a) \in \Op{m} $. Then $ g \circ f $ is differentiable at $ a $ and further
  \[ \underbrace{D(g \circ f)(a)}_{\R^n \to \R^p} = \underbrace{Dg(f(a))}_{\R^m \to \R^p} \cdot \underbrace{Df(a)}_{\R^n \to \R^m} \]
\end{Thm}

\begin{proof}
  Let $ A = Df(a) $ and $ B = Dg(b) $ where $ b = f(a) $. For $ x,y $ in sufficiently small neighbourhoods of $ a,b $ respectively, we consider the maps
  \begin{align*}
    r_f(x) & = f(x) - f(a) - A(x-a)      \\
    r_g(y) & = g(y) - g(b) - B(y-b)      \\
    r(x)   & = g(f(x)) - g(b) - BA(x-a)
  \end{align*}
  By definition of the derivative,
  \[ \lim_{x \to a}\frac{r_f(x)}{\norm{x-a}}=0 \qquad \lim_{y \to b}\frac{r_g(y)}{\norm{y-b}}=0\]
  We wish to prove that
  \[ \lim_{x \to a}\frac{r(x)}{\norm{x-a}}=0 \]
  We have,
  \begin{align*}
    r(x)          & = g(f(x)) - g(b) - BA(x-a)                \\
                  & = g(f(x)) - g(b) + B(r_f(x) - f(x)+f(a))  \\
                  & = Br_f(x) + g(f(x)) - g(b) - B(f(x)-f(a)) \\
    \implies r(x) & = Br_f(x) + r_g(f(x))
  \end{align*}
  Now,
  \[ \lim_{x \to a}\frac{Br_f(x)}{\norm{x-a}} = B \left(\lim_{x \to a}\frac{Br_f(x)}{\norm{x-a}}\right) = 0 \]\ssk
  
  The other term requires some more analysis. Fix $ \veps > 0 $. There is $ \delta > 0 $ such that $ \norm{r_g(y)} < \veps \norm{y-b} $ for all $ y $ with $ 0 < \norm{y-b} < \delta $. By continuity of $ f $ at $ a $, there is $ \delta_1 > 0 $ such that $ \norm{f(x) - f(a)} < \delta $ for all $ x $ with $ 0 < \norm{x-a} < \delta_1 $. Hence,
  \[ 0 < \norm{x-a} < \delta_1 \implies \norm{r_g(f(x))} < \veps\norm{f(x)-f(a)} \]
  and so
  \[ \lim_{x \to a}\frac{r_g(f(x))}{\norm{f(x)-f(a)}} = 0 \]
  We now note,
  \[ \norm{f(x)-f(a)} = \norm{r_f(x)+A(x-a)} \leq \norm{r_f(x)} + M_A\norm{x-a} \]
  where we get $ M_A > 0 $ by Theorem \ref{th:LinLip}. Hence, we finally have
  \[ \lim_{x \to a}\frac{r(x)}{\norm{x-a}} = \lim_{x \to a}\frac{r_g(f(x))}{\norm{f(x)-f(a)}} \frac{\norm{f(x)-f(a)}}{\norm{x-a}} = 0 \]
  which completes the proof.
  
\end{proof}

\end{document}