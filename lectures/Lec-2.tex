\documentclass[../Analysis-3]{subfiles}

\begin{document}
\chapter*{Lecture 2} %Set chapter name
\addcontentsline{toc}{chapter}{Lecture 2} %Set chapter title
\setcounter{chapter}{2} %Set chapter counter
\setcounter{section}{0}

\section{Distance and Topology in $ \R^n $}

Using the inner product on $ \R^n $, we get the Euclidean distance
\[ d(x,y) = \norm{x-y} = \sqrt{\inp{x-y}{x-y}} = \sqrt{\sum_{i=1}^{n} (x_i-y_i)^2} \]

We wish to extend the notions of $ (\R, \abs{\cdot}) $ to $ (\R^n, \norm{\cdot}) $. We already know from Lecture 1 that the triangle inequality holds,
\[ \norm{x+y} \leq \norm{x} + \norm{y} \]

\begin{Def}{Open balls}{}
  The open ball centered at $ a \in \R^n $ of radius $ r $ is,
  \[ B_r(a) = \{x \in \R^n \mid \norm{x-a} < r \} \]
\end{Def}

\textbf{Exercise.} Show that open balls are convex sets.
\msk

\begin{Def}{Open sets}{}
  A set $ \mathcal O \subseteq \R^n $ is open if $ \mathcal O = \phi $ or for all $ x \in \mathcal O $, there is $ r > 0 $ such that $ B_{r}(x) \subseteq \mathcal O $.
\end{Def}

\begin{Eg}{}{}
  \begin{enumerate}[label = (\roman*)]
    \item Any open ball is open.
    \item We define open boxes in $ \R^n $ to be the subsets of the form $ \prod_{i=1}^{n}(a_i,b_i) $. Any open box is open.
  \end{enumerate}
\end{Eg}

\begin{Def}{Convergence of Sequences}{}
  Let $ \{x_m\}_{m \in \N} \subseteq \R^n $ and $ x \in \R^n $. We say $ x_m \longrightarrow x $ if for all $ \veps > 0, $ there is $ N \in \N $ such that
  \begin{alignat*}{2}
         & \norm{x_m - x}           < \veps \, &  & \forall \, m \geq N \\
    \iff & d(x_m, x)           < \veps \,      &  & \forall \, m \geq N \\
    \iff & x_m \in B_\veps(x)  \,              &  & \forall \, m \geq N
  \end{alignat*}
\end{Def}

\textbf{Exercise.} Show that the limit of a sequence in $ \R^n $ is unique whenever it exists.

\begin{Def}{Limit points}{}
  We define the deleted $ \veps $-neighbourhood of $ a \in \R^n $ to be $ D_\veps(a) = B_\veps(a) \setminus \{a\} $. The point $ a $ is a limit point of $ S \subseteq \R^n $ if for all $ \veps > 0, D_\veps(a) \cap S \neq \phi $. If we do net delete $ a $, we get isolated points.
\end{Def}

\begin{Def}{Projections}{}
  For all $ i \in \{1,2, \dots, n\} $ we define the maps
  \begin{align*}
    \Pi_i : \R^n          & \to \R      \\
    x = (x_1, \dots, x_n) & \mapsto x_i
  \end{align*}
  $ \Pi_i $ is called the projection onto the $ i $th coordinate.
\end{Def}

\begin{Thm}{}{}
  Let $ \{x_m\}_{m \in \N} \cup \{x\} \subseteq \R^n $. Then,
  \begin{align*}
    x_m             & \longrightarrow x                                              \\
    \iff \Pi_i(x_m) & \longrightarrow \Pi_i(x) \, \forall \, i \in \{1,2, \dots, n\}
  \end{align*}
\end{Thm}
\begin{proof}
  Assume $ x_m \longrightarrow x $. Now, for all $ j \in \{1,2, \dots, n\} $,
  \begin{align*}
    \norm{x_m-x}^2 = \sum_{i=1}^{n} \abs{\Pi_i(x_m) - \Pi_i(x)}^2 & \geq \abs{\Pi_j(x_m) - \Pi_j(x)}^2 \\
    \implies \abs{\Pi_j(x_m) - \Pi_j(x)}                          & \longrightarrow 0                  \\
    \implies \Pi_j(x_m)                                           & \longrightarrow \Pi_j(x)
  \end{align*}

  Now assume $ \Pi_j(x_m) \longrightarrow \Pi_j(x) $ for all $ j \in \{1,2, \dots, n\}
  $. Then,
  \begin{align*}
    \abs{\Pi_j(x_m) - \Pi_j(x)}                           & \longrightarrow 0, \, \forall \, j \in \{1,2, \dots, n\} \\
    \implies \sum_{i=1}^{n} \abs{\Pi_j(x_m) - \Pi_j(x)}^2 & \longrightarrow 0                                        \\
    \implies \norm{x_m - x}^2                             & \longrightarrow 0                                        \\
    \implies x_m                                          & \longrightarrow x
  \end{align*}
\end{proof}

\begin{Def}{Closed sets}{}
  A set $ C \subseteq \R^n $ is closed if $ \R^n \setminus C $ is open.
\end{Def}

\textbf{Exercise.} Show that a set $ C \subseteq \R^n $ is closed iff $ \forall \, \{x_m\}_{m \in \N} \subseteq C $ with $ x_m \longrightarrow x $ for some $ x \in \R^n $, we have $ x \in C $.
\msk

\textbf{Exercise.} Show that:
\begin{enumerate}[label = (\arabic*)]
  \item Arbitrary union of open sets is open.
  \item Finite intersection of open sets is open.
  \item Arbitrary intersection of closed sets is closed.
  \item Finite union of closed sets is closed.
  \item Any finite subset of $ \R^n $ is closed.
\end{enumerate}

\begin{Def}{Interior of a set}{}
  Let $ \phi \neq S \subseteq \R^n $. The interior of $ S $ is,
  \[ \operatorname{Int}(S) = \{a \in S \mid \exists \, r > 0, B_r(a) \subseteq S\} \]
\end{Def}
\textbf{Exercise.} Show that:
\begin{enumerate}[label = (\arabic*)]
  \item For any nonempty set $ S \subseteq \R^n $, $ \operatorname{Int}(S) $ is open.
  \item A set $ S $ is open iff $ \operatorname{Int}(S) = S $.
\end{enumerate}

\begin{Def}{Exterior of a set}{}
  Let $ \phi \neq S \subseteq \R^n $. The exterior of $ S $ is,
  \[ \operatorname{Ext}(S) = \{a \in \R^n \mid \exists \, r > 0, B_r(a) \cap S = \phi \} \]
\end{Def}
\textbf{Exercise.} Show that $ \operatorname{Ext}(S) = \operatorname{Int}(\R^n \setminus S) $.

\begin{Eg}{}{}
  For $ S = [0,2] \setminus \{1\} = [0,1) \cup (1,2] $, $ 1 \notin \operatorname{Ext}(S) $.
\end{Eg}

\begin{Def}{Boundary of a set}{}
  Let $ \phi \neq S \subseteq \R^n $. The boundary of $ S $ is,
  \[ \partial S = \{a \in \R^n \mid \forall \, r > 0, B_r(a) \cap S \neq \phi \text{ and } B_r(a) \cap (\R^n \setminus S) \neq \phi \} \]
\end{Def}

\begin{Eg}{}{}
  For $ S = [0,1) \cup (1,2] \cup \{5\} $, $ \partial S = \{0,1,2,5\} $ but the set of limit points is $ \{0,1,2\} $.
\end{Eg}
\textbf{Exercise.} Show that:
\begin{enumerate}[label = (\arabic*)]
  \item $ S $ is open iff $ S \cap \partial S = \phi $.
  \item $ S $ is closed iff $ S \supseteq \partial S $.
  \item $ S $ is closed iff $ S = \overline{S} =: S \cup \partial S = S \cup \{\text{Limit points of } S\} $
  \item $ \overline{S} = \operatorname{Int}(S) \sqcup \partial S $. This gives the partition $ \R^n = \operatorname{Int}(S) \sqcup \partial S \sqcup \operatorname{Ext}(S) $
  \item $ \partial S $ is closed.
  \item Let $ \{\mathcal O_i\}_{i=1}^n \subseteq \mathcal P(\R) $ and define $ \mathcal O = \prod_{i=1}^{n} \mathcal O_i $. If $ \mathcal{O}_i $'s are open (closed), $ \mathcal O $ is open (closed).
\end{enumerate}
\pagebreak

\section{Limits and Continuity}
Recall the notion of limit in $ \R $:
\ssk

Suppose $ f: (a,b)\setminus \{c\} \to \R $ is a function. We say $ \displaystyle\lim_{x \to c}f $ exists if there is $ \alpha \in \R $ such that $ \forall \, \veps > 0, \, \exists \, \delta > 0 $ such that $ x \in D_{\delta}(c) \implies \abs{f(x) - \alpha} < \veps $, that is, $ f(D_\delta(c)) \subseteq B_\veps(\alpha) $; in such a case we say that $ \displaystyle \lim_{x \to c}f = \alpha $.

We now extend this to $ \R^n $.
\msk

\begin{Def}{Limits in $ \R^n $}{}
  Let $ S \subseteq \R^n, a \in \{\text{Limit points of } S\} $ and, $ f: S \setminus \{a\} \to \R^m $. We say that $ \displaystyle \lim_{x \to a}f = b $ if for all $ \veps > 0 $, there is $ \delta > 0 $ such that $ f(x) \in B_\veps(b) $ for all $ x \in D_\delta(a) \cap S $.

  In other words, $ \displaystyle \lim_{x \to a}f = b $ if for all $ \veps > 0 $, there is $ \delta > 0 $ such that
  \[ \norm{f(x) - b} < \veps \quad \forall \, x \in S, 0 < \norm{x - a} < \delta \]
\end{Def}
\msk

\begin{Def}{Continuity in $ \R^n $}{}
  Let $ S \subseteq \R^n, a \in S $ and, $ f: S \to \R^m $. We say that $ f $ is continuous at $ a $ if for all $ \veps > 0 $, there is $ \delta > 0 $ such that $ f(x) \in B_\veps(f(a)) $ for all $ x \in B_\delta(a) \cap S $, that is, $ \norm{f(x) - f(a)} < \veps $ for all $ x \in S $ with $ \norm{x - a} < \delta $.
\end{Def}

\textbf{Note:} Any function defined on $ S $ is vacuously continuous at an isolated point $ a $ by our definition.

\end{document}