\documentclass[../Analysis-3]{subfiles}

\begin{document}
\chapter*{Lecture 8} %Set chapter name
\addcontentsline{toc}{chapter}{Lecture 8} %Set chapter title
\setcounter{chapter}{8} %Set chapter counter
\setcounter{section}{0}

\section{A kind of converse of the [above theorem]}

\begin{Thm}{Final Reduction}{}\label{thm:finalred}
  Let $ f :\Op{n} \to \R^m $ and $ a \in \Op{n} $. Suppose, $ \pdv{f_i}{x_j} $ exists on $ \Op{n} $ and continuous at $ a \in \Op{n} $ for all $ i\in [m], j\in [n] $. Then,
  \begin{align*}
    Df(a) = J_{f}(a) = \begin{bmatrix}
                         \pdv{f_i}{x_j}\left( a \right)
                       \end{bmatrix}
  \end{align*}
\end{Thm}

\begin{proof}
  Without loss of generality, take $ a = (0,\ldots,0) \in \Op{n} $ and $ m = 1 $.

    [
      \textbf{Back Calculation:}

      We already ``know", $ L = \begin{bmatrix}
          f_{x_1}(0) & \cdots & f_{x_n}(0)
        \end{bmatrix} $ and $ Lh = \sum_{i=1}^{n}h_i\pdv{f}{x_i}\left( 0 \right)\ \forall\ h \in \R^n $
    ]

  \begin{clmBox}
    \begin{align*}
      \frac{1}{\norm{h}}\left| f(h)-f(0)-Lh \right| \to 0 \text{ as } h \to 0
    \end{align*}
  \end{clmBox}

  \begin{proof}
    Clearly, $ \frac{1}{\norm{h}}\left| f(h)-f(0)-Lh \right| = \frac{1}{\norm{h}}\left| f(h)-f(0)-\sum_{i=1}^{n}h_i\pdv{f}{x_i}\left( 0 \right) \right|\ \forall\ i \in [n]$. Define, $ \hat{h}_i = (h_1, \ldots, h_i, \underbrace{0,\ldots,0}_{n-r}) $ and $ \hat{h}_0  = 0 $. Then,
    \begin{align*}
      f(h)-f(0) & = \left( f\left(\hat{h}_1\right) - f\left(\hat{h}_0\right) \right) + \left( f\left(\hat{h}_2\right) - f\left(\hat{h}_1\right) \right) + \cdots + \left( f\left(\hat{h}_{n}\right) - f\left(\hat{h}_{n-1}\right) \right) \\
                & =\sum_{i=1}^{n} \left( f\left(\hat{h}_{i}\right) - f\left(\hat{h}_{i-1}\right) \right)
    \end{align*}
    which implies, \begin{align*}
      \left| f(h)-f(0)-\sum_{i=1}^{n}h_i\pdv{f}{x_i}\left( 0 \right) \right| & = \left| \sum_{i=1}^{n} \left( f\left(\hat{h}_{i}\right) - f\left(\hat{h}_{i-1}\right) - h_i\pdv{f}{x_i}\left( 0 \right) \right) \right|                                \\
                                                                             & = \left| \sum_{i=1}^{n} \underbrace{h_i\pdv{f}{x_i}\left( \hat{h}_{i-1} + c_ie_i \right)}_{\text{By MVT, as explained below}} - h_i\pdv{f}{x_i}\left( 0 \right) \right|
    \end{align*}

    For a fixed $ h $ fix $ i \in [n] $. Consider the map,
    \[\begin{tikzcd}
        t &&& {\hat{h}_{i-1} + te_i} &&& {f\left( \hat{h}_{i-1} + te_i \right)} \\
        {(h_i-\epsilon, h_i+\epsilon)} &&& {\R^n} &&& \R
        \arrow[maps to, from=1-1, to=1-4]
        \arrow[maps to, from=1-4, to=1-7]
        \arrow["{\eta_i}", curve={height=-30pt}, maps to, from=1-1, to=1-7]
        \arrow[from=2-1, to=2-4]
        \arrow[from=2-4, to=2-7]
        \arrow["{\eta_i}"', curve={height=30pt}, from=2-1, to=2-7]
      \end{tikzcd}\]
    Clearly, $ \eta_i $ is differentiable on $ (0,h_i) $ and continuous on $ [0,h_i] $. So, MVT gives \begin{align*}
      \left( \eta_i(h_i) - \eta_i(0) \right) = f_{x_i}(c_i)h_i \qquad \Big(=\eta_i'(c_i)h_i \Big) \tag{for some $ c_i\in (0,h_i) $}
    \end{align*}

    Now, observe that, as $ h \to 0, \hat{h}_{i-1}+c_ie_i \to 0 $ which in turn yields, $ f_{x_i}\left( \hat{h}_{i-1}+c_ie_i \right) \to f_{x_i}(0) $. Therefore,
    \begin{align*}
      \frac{1}{\norm{h}}\left| f(h)-f(0)-Lh \right| & = \left| \sum_{i=1}^{n} h_i\pdv{f}{x_i}\left( \hat{h}_{i-1} + c_ie_i \right) - h_i\pdv{f}{x_i}\left( 0 \right) \right|                                          \\
                                                    & = \frac{1}{\norm{h}}\sum_{i=1}^{n} \left| h_i \right|\left| \pdv{f}{x_i}\left( \hat{h}_{i-1} + c_ie_i \right) - \pdv{f}{x_i}\left( 0 \right) \right|            \\
                                                    & \leq \sum_{i=1}^{n} \left| \pdv{f}{x_i}\left( \hat{h}_{i-1} + c_ie_i \right) - \pdv{f}{x_i}\left( 0 \right) \right|  \tag{as $\norm{h} \geq |h_i|\ \forall\ i$} \\
                                                    & \longrightarrow 0 \text{ as } h \to 0
    \end{align*}
    Hence, we get the claim!
  \end{proof}
  And, this completes the proof.
\end{proof}

\textbf{Remark.}
\begin{enumerate}
  \item With Theorem (\ref{thm:finalred}) computation of derivative is much easier when we are in favorable situation.
  \item We have the following: \begin{enumerate}[label = (\roman*)]
          \item If $ f $ is differentiable at $ a $ then $ \pdv{f_i}{x_j}\left( a \right) $ exists for all $ i, j $ and $ Df(a) = J_{f}(a) $
          \item If $ \pdv{f_i}{x_j} $ is continuous at $ a $ then $ f $ is differentiable and $ Df(a) = J_{f}(a) $.
        \end{enumerate}
        $ \therefore $ the gap between (i) and (ii) $ \equiv $ Continuity of partials, which is removable.
\end{enumerate}

\begin{Eg}{Differentiable but discontinuous}{}
  Take, \begin{align*}
    f(x,y) = \begin{cases}
               \left( x^2+y^2 \right)\sin{\left( \frac{1}{\sqrt{x^2+y^2}} \right)}, & (x,y) \not= 0 \\
               0,                                                                   & (x,y) = 0
             \end{cases}
  \end{align*}

  Then, \begin{align*}
    \abs{f(x,y) - f(0,0)} & = \abs{x^2+y^2}\abs{\sin{\left( \frac{1}{\sqrt{x^2+y^2}} \right)}} \\
                          & \leq (x^2+y^2) = \norm{(x,y)}^2
  \end{align*}
  $ \implies f $ is continuous at $ (0,0) $.

  The partials of $ f $, \begin{align*}
    f_x(x,y) & = 2x\sin{\left( \frac{1}{\sqrt{x^2+y^2}} \right)} - \frac{x}{\sqrt{x^2+y^2}}\cos{\left( \frac{1}{\sqrt{x^2+y^2}} \right)} \\
    f_y(x,y) & = 2y\sin{\left( \frac{1}{\sqrt{x^2+y^2}} \right)} - \frac{y}{\sqrt{x^2+y^2}}\cos{\left( \frac{1}{\sqrt{x^2+y^2}} \right)}
  \end{align*}
  So their limits along the respective axis, \begin{align*}
    f_x(0,0) & = \lim_{t \to 0} \frac{f(t,0) - f(0,0)}{t} = 0 \\
    f_y(0,0) & = \lim_{t \to 0} \frac{f(0,t) - f(0,0)}{t} = 0
  \end{align*}
  And, \begin{align*}
    \frac{1}{\sqrt{h^2+k^2}}\abs{f(h,k) - f(0,0) - L\begin{bmatrix} h \\ k \end{bmatrix}} & = \sqrt{h^2+k^2}\abs{\sin{\left( \frac{1}{\sqrt{h^2+h^2}} \right)}} \leq \norm{(h,k)}
  \end{align*}
  which shows that, $ f $ is differentiable at $ (0,0) $ and $ Df(0,0) = \begin{bmatrix} 0 & 0 \end{bmatrix} $. But, $ f_x $ and $ f_y $ are not continuous at $ (0,0) $!
\end{Eg}

\begin{Eg}{Homework}{}
  Take, \begin{align*}
    f(x,y) = \begin{cases}
               x^{\frac{4}{3}}\sin{\left( \frac{y}{x} \right)}, & x \not= 0 \\
               0,                                               & x = 0
             \end{cases}
  \end{align*}
  \begin{itemize}
    \item Show that, \begin{enumerate}
            \item $ f $ is differentiable on $ \R^2 $.
            \item $ f_x $ and $ f_y $ exist and continuous on $ \Op{2} = \left\{ (x,y)\in\R^2 : x \not= 0 \right\} $.
            \item $ f_x $ is not continuous at $ (0,y) $ for all $ y \not= 0 $.
          \end{enumerate}
    \item Discuss the nature of continuity of $ f $ at the origin.
  \end{itemize}
\end{Eg}

\begin{Eg}{}{}
  Let, $ f:\R^3 \to \R^4 $ be defined by, \[ f(x,y) = \left( x+2y+3z, xyz, \cos{x}, \sin{x} \right) \]
  Then, the jacobian at $ (x,y,z) $  \begin{align*}
    J_{f}(x,y,z) = \begin{bmatrix}
                     1             & 2  & 3  \\
                     yz            & zx & xy \\
                     -\sin{x}      & 0  & 0  \\
                     \quad \cos{x} & 0  & 0
                   \end{bmatrix}
  \end{align*}
  which has every entry continuous, thus \[J_{f}(x,y,z) = Df(x,y,z)\]
\end{Eg}

\end{document}