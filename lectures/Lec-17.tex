\documentclass[../Analysis-3.tex]{subfiles}
% \myexternaldocument{Lec-14}

\begin{document}
\chapter*{Lecture 17} %Set chapter name
\addcontentsline{toc}{chapter}{Lecture 17} %Set chapter title
\setcounter{chapter}{17} %Set chapter counter
\setcounter{section}{0}

\section{Properties of Riemann-Darboux Integration}

In the previous lecture, we introduced the Riemann-Darboux Integral. In this lecture, we will explore some important properties of this integral, starting with a characterization.

\begin{Thm}{Classification of Riemann Integrable Functions}{}
  Let $ f \in \mathscr{B}(B^n) $. Then $ f \in \mathscr{R}(B^n) $ if and only if for every $ \epsilon > 0 $, there exists a partition $ P \in \mathscr{P}(B^n) $ of $ B^n $ such that
  \[  (0 \leq) \ U(f,P) - L(f,P) < \epsilon  \]
\end{Thm}

\begin{proof}
  \begin{itemize}
    \item[$ \Longrightarrow $] Suppose $f \in \mathscr{R}(B^n)$. Then we have
      \[  \overline{\int} f - \underline{\int} f = 0  \]
      Thus,
      \begin{align*}
        0 = \overline{\int} f - \underline{\int} f
         & = \inf_{P \in \mathscr{P}(B^n)} U(f,P) - \sup_{P \in \mathscr{P}(B^n)} L(f,P) \\
         & = \inf_{P \in \mathscr{P}(B^n)} (U(f,P) - L(f,P))                             \\
      \end{align*}
      Hence for all $\epsilon > 0$, there exists a partition $P \in \mathscr{P}(B^n)$ such that $U(f,P) - L(f,P) < \epsilon$.
    \item[$ \Longleftarrow $] Conversely, assume that for every $\epsilon > 0$, there exists a partition $P \in \mathscr{P}(B^n)$ such that $U(f,P) - L(f,P) < \epsilon$. We want to show that $\displaystyle\overline{\int} f = \underline{\int} f$. Since $\displaystyle U(f,P) \geq \overline{\int} f$ and $\displaystyle L(f,P) \leq \underline{\int} f$, it follows that for all $\epsilon > 0$,
      \[  0 \leq \overline{\int}f - \underline{\int}f < \epsilon  \]
      This implies $\displaystyle\overline{\int} f = \underline{\int} f$, showing that $f \in \mathscr{R}(B^n)$.
  \end{itemize}
\end{proof}

\textbf{Exercise.} Let $ f,g \in \mathscr{R}(B^n) $. Then show that,
\begin{itemize}
  \item $ \abs{f} \in \mathscr{R}(B^n) $ and $ \displaystyle \abs{\int_{B^n} f} \leq \int_{B^n} \abs{f} $
  \item  $ \mathscr{R}(B^n) $ is a $\R-$algebra by showing the following, \begin{enumerate}[label = (\roman*)]
          \item For any $ \alpha, \beta \in \R $,  $ \alpha f + \beta g \in \mathscr{R}(B^n) $
          \item $ fg \in \mathscr{R}(B^n) $
        \end{enumerate}
\end{itemize}

Next, we demonstrate the Riemann integrability of a class of ``nice'' functions (continuous). However, before proceeding, let's introduce the concepts of the diameter of a set and the mesh of a partition.

\begin{Def}{Diameter of a set}{}
  For a set $A \subseteq \R^n$, the diameter $d(A)$ is defined as $d(A) = \sup\{ |x-y| \mid x,y \in A \}$.
\end{Def}

\textbf{Exercise.} Show that $ d(B^n) = \max\left\{ \norm{v_i-v_j} \mid v_i, v_j \text{ are vertices of } B^n \right\} $

\begin{Def}{Mesh of a Partition}{}
  For a partition $P \in \mathscr{P}(B^n)$, the mesh $\norm{P}$ is defined as $\norm{P} = \max\{ d(B_{\alpha}^n) \mid \alpha \in \Lambda(P) \}$.
\end{Def}

\begin{Thm}{Continuous Functions are Riemann Integrable}{}
  The set of all continuous functions over $B^n$ is contained in $\mathscr{R}(B^n)$, i.e.,
  \[  C(B^n) \subset \mathscr{R}(B^n)  \]
\end{Thm}

\begin{proof}
  Let $f \in C(B^n)$. Since $f$ is uniformly continuous, for any $\epsilon > 0$, there exists $\delta > 0$ such that for all $x, y \in B^n$ with $\|x-y\| < \delta$, we have,
  \begin{equation}\label{eq:toweg:uccond}
    \abs{f(x) - f(y)} < \underbrace{\frac{\epsilon}{2\Vol(B^n)}}_{\text{Call it }\tilde{\epsilon}}
  \end{equation}
  Let $P \in \mathcal{P}(B^n)$ be a partition such that $\|P\| < \delta$. For each $\alpha \in \Lambda(P)$, let $a_{\alpha} \in B_{\alpha}^n$. Then $\|x-a_{\alpha}\| < \delta$ for all $x \in B_{\alpha}^n$. It follows from the uniform continuity condition (\ref{eq:toweg:uccond}) that,
  \begin{align}
    \abs{f(x) - f(a_{\alpha})}                         & < \tilde{\epsilon}  \nonumber                               \\
    \implies f(a_{\alpha}) - \tilde{\epsilon} < f(x) < & \; f(a_{\alpha}) - \tilde{\epsilon} \label{eq:toweg:ucimpl}
  \end{align}
  Since, (\ref{eq:toweg:ucimpl}) holds for all $ \alpha \in \Lambda(P) $, $ a_{\alpha} \in B_{\alpha}^n $ and for all $ x \in B_{\alpha}^n $ we have,
  \[  f(a_{\alpha}) - \tilde{\epsilon} \leq m_{\alpha} \leq M_{\alpha} \leq f(a_{\alpha}) + \tilde{\epsilon}  \]
  Multiplying the volumes of $B_{\alpha}^n$ and summing over $\Lambda(P)$, we obtain,
  \[
    \sum_{\alpha \in \Lambda(P)} f(a_{\alpha})\Vol(B_{\alpha}^n) - \frac\epsilon2 \leq L(f,P) \leq U(f,P) \leq \sum_{\alpha \in \Lambda(P)} f(a_{\alpha})\Vol(B_{\alpha}^n) + \frac\epsilon2
  \]
  Thus, $U(f,P) - L(f,P) < \epsilon$, and since $\epsilon$ is arbitrary, we conclude that $f \in \mathscr{R}(B^n)$.
\end{proof}

\

Now, let's consider an important question: Does an analogue of the Fundamental Theorem of Calculus exist in higher dimensions?

In one dimension ($n=1$), we have the useful relationship $ \displaystyle\int_{[a,b]} \dd f = f\Big\vert_{\partial[a,b]} $, which aids in computation.

However, this relationship becomes less practical in higher dimensions. For instance, in $n=1$, the continuous counterpart to a sum $ \displaystyle\sum a_n $ is the one-dimensional integral $ \displaystyle\int_{B^1} f $. Similarly, in $n=2$, the continuous analogue to a double sum $ \displaystyle\sum a_{mn} $ is the two-dimensional integral $ \displaystyle\int_{B^2} f $.

\section{Iterated Integrals}

Before delving deeper into the concept of integrability, let's take a brief detour to discuss the idea of a double sum.

\begin{Def}{Convergence of Double Sequence}{conv:double:seq}
  A double sequence $ \{a_{mn}\} $ converges to $ a $ if for every $ \epsilon > 0 $ there exists a $ \delta > 0 $ such that $ \abs{a_{mn} - a} < \epsilon $ for all $ m,n \geq N $
\end{Def}

Consider the following examples,
\begin{Eg}{}{}
  \begin{itemize}
    \item Let's take the sequence $ \{a_{mn}\} $ defined by $ a_{mn} = \frac 1{m+n} $ for all $ m,n \in \N $. This sequence is bounded, and for $ N > \frac 1{2\epsilon} $, we have $ \abs{a_{mn} - 0} = a_{mn} < \epsilon $ for all $ m,n \geq N $.
    \item Now consider the sequence $ {a_{mn}} $ defined as follows,
          \[  a_{mn} = \begin{cases}
              n            & \text{if } m=1   \\
              \frac 1{m+n} & \text{otherwise}
            \end{cases}  \]
          This sequence is also convergent but not bounded.
  \end{itemize}
\end{Eg}

Recall the relation between total limit and iterated limits in double sequence,
\begin{ThmN}{}{}
  For a double sequence $ \{a_{mn}\} $ if $ \lim_{m,n \to \infty} a_{mn} $ exists and $ \lim_{m \to \infty} a_{mn} $ exists for all $ n $, then
  \[  \lim_{n \to \infty}\left( \lim_{m \to \infty} a_{mn} \right) = \lim_{m,n \to \infty} a_{mn}  \]
\end{ThmN}
An important analogue of the above result is Fubini's Theorem. Computation of the Darboux integral is typically a challenging task. However, Fubini's Theorem offers a valuable approach that simplifies the computation by utilizing iterated integrals.

\subsection*{Visualization}

We look at slice functions along each axis, which enables us to simplify computations and apply Fubini's Theorem for efficient evaluation of multivariable integrals.

\ssk

Consider a function $ f: B^2 \to \R $. For each $ x \in [a_1, b_1] $ we define a slice function $ f_{x}:[a_2, b_2] \to \R $ given by $ f_{x}(y) = f(x,y) $ for all $ y \in [a_2, b_2] $. This function is obtained by fixing $ x $ and slicing along the $ y $-axis at that $ x $-coordinate. Then an iterated integral becomes
\[  \int_{[a_1, b_1]}\left( \int_{[a_2, b_2]} f_x(y) \; \dd y \right)\dd x  \]
The question arise whether this quantity is invariant under the interchange of $ x $ and $ y $, i.e., we may slice $ f $ along $x-$axis at $ y $ to obtain $ f_y:[a_1, b_1] \to \R $ given by $ f_{y}(x) = f(x,y) $ for every $ x \in [a_1, b_1] $ and want to investigate the equality of
\begin{equation}
  \int_{[a_1, b_1]}\left( \int_{[a_2, b_2]} f_x(y) \; \dd y \right)\dd x \overset{?}{=} \int_{[a_2, b_2]}\left( \int_{[a_1, b_1]} f_y(x) \; \dd x \right)\dd y \overset{?}{=} \int_{B^2} f \label{eq:fubini:cond:check}
\end{equation}
In this context, we observe that a partition $P \in \mathscr{P}(B^2)$ can be decomposed into the partitions of the individual coordinates. Specifically, we have $P = P_1 \times P_2$ for the two coordinate intervals $[a_1, b_1]$ and $[a_2, b_2]$, and the corresponding indexing sets satisfy $\Lambda(P) = \Lambda(P_1) \times \Lambda(P_2)$.

\ssk

Consider the example,

\begin{Eg}{A discrepancy: Function integrable, slices not}{}
  Let $ I = [0,1] $ and $ B^2 = I \times I $. Consider the function $ f:B^2 \to \R $ given by,
  \[  f(x,y) = \begin{cases}
      1 & \text{if } x = \frac 12, y \in \Q \cap [0,1] \\
      0 & \text{otherwise}
    \end{cases}  \]
  So the $x-$slice becomes,
  \[  f_x \equiv 0 \text{ for all } x \neq \frac 12 \qquad \text{and} \qquad f_{\frac 12}(y) = \underset{\text{Dirichlet Function}}{\begin{cases}
        1 & \text{if } y \in \Q^c \cap [0,1] \\
        0 & \text{if } y \in \Q \cap [0,1]
      \end{cases}} \]
  and $y-$slice,
  \begin{align*}
     & \text{For } y \in \Q \cap [0,1], \ f_y(x) = \begin{cases}
                                                     1 & \text{if } x = \frac 12 \\
                                                     0 & \text{otherwise}
                                                   \end{cases} \\
     & \text{For } y \in \Q^c \cap [0,1], \ f_y \equiv 0
  \end{align*}
  Clearly, $ f_x \in \mathscr{R}(I) $ for all $ x \in \frac 12 $ but $ f_{\frac 12} \not\in \mathscr{R}(I) $ and $ f_y \in \mathscr{R}(I) $ for every $ y $. So,
  \[  \int_I f_y = 0 \implies \int_I \left( \int_I f_y(x) \dd x \right) \dd y = 0 \]
  But, $ \displaystyle\int_I f_x $ doesn't exist for $ x = \frac 12 $ which means $ \displaystyle x \longmapsto \int_I f_x $ is not a well-defined function on $ [0,1] $. Hence $ \displaystyle\int_I \left( \int_I f_x(y) \dd y \right) \dd x $ doesn't exist. Yet $ f \in \mathscr{R}(B^2) $. To see this, we fix $ \epsilon > 0 $ and consider the partition $ P = P_1 \times P_2 $ where,
  \begin{align*}
    \begin{cases}
      P_1: 0 < \frac 12 - \epsilon < \frac 12 + \epsilon < 1 \\
      P_2: 0 < 1
    \end{cases}
  \end{align*}
  So, $ P = \Bigg\{ \underbrace{\left[ 0, \frac 12 - \epsilon \right] \times I}_{B_{\alpha_1}}, \underbrace{\left[\frac 12 - \epsilon, \frac 12 + \epsilon \right] \times I}_{B_{\alpha_2}}, \underbrace{\left[\frac 12 - \epsilon, 1 \right] \times I}_{B_{\alpha_3}} \Bigg\} $.

  Then $ m_{\alpha_1} = m_{\alpha_{2}} = m_{\alpha_3} = 0 $, $ M_{\alpha_1} = M_{\alpha_3} = 0 $ and $ M_{\alpha_2} = 1 $, which implies $ U(f,P) - L(f,P) = 2\epsilon < 3\epsilon $. This shows that $ f \in \mathscr{R}(B^2) $. Again $ L(f,P) = 0 $ for all $ P \in \mathscr{P}(B^2) $ and hence,
  \[  \int_{B^2} f = 0  \]
\end{Eg}

\textbf{Question.} Under which conditions does (\ref{eq:fubini:cond:check}) hold?

\textit{Answer.} Fubini's Theorem. The conditions for (\ref{eq:fubini:cond:check}) to hold will be discussed in the next lecture.

\end{document}
