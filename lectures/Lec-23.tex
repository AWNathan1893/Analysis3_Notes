\documentclass[../Analysis-3.tex]{subfiles}
\usepackage{style}

\begin{document}
\chapter*{Lecture 23} %Set chapter name
\addcontentsline{toc}{chapter}{Lecture 23} %Set chapter title
\setcounter{chapter}{23} %Set chapter counter
\setcounter{section}{0}
\setcounter{equation}{0}
\setcounter{figure}{0}


We will begin this lecture with few examples.
\begin{Eg}{Path Of a Projectile}{}
  \begin{align*}
    \gamma(t)           & =(\alpha t, \beta t-16t^2)               \\
    \implies \gamma'(t) & = (\alpha, \beta -32t)                   \\
    \text{path length}  & = \int \norm{\gamma'(t)}\dd{t}           \\
                        & = \int \sqrt{\alpha^2 +(\beta-32t)^2} dt
  \end{align*}
\end{Eg}

\begin{Eg}{Perimeter of a Circle}{}
  Parametrization of a circle of radius $r$ is given by $\gamma(t) = (r\cos(t),r\sin(t)), t \in [0,2\pi)$.
  \begin{align*}
    \gamma'(t)    & = (-r\sin(t),r\cos(t))                    \\
    \ell (\gamma) & = \int_{0}^{2\pi} \norm{\gamma'(t)}\dd{t} \\
                  & = \int_{0}^{2\pi} r dt                    \\
                  & = 2\pi r
  \end{align*}
\end{Eg}

\begin{Eg}{Arc Length of graph of functions}{}
  Let, $f : [a,b]\to \R$ be a $C^1$ function. Consider $\gamma(t)=(t, f(t))$. It is a smooth curve.
  \begin{align*}
    \gamma'(t)    & = (1,f'(t))                            \\
    \ell (\gamma) & = \int_{a}^{b} \norm{\gamma'(t)}\dd{t} \\
                  & = \int_{a}^{b} \sqrt{1 + f'(t)^2} dt
  \end{align*}
\end{Eg}

\section{Line Integrals}
To integrate a function over a curve we use \textbf{Line integral}. The function we should integrate maybe a \textbf{Scalar Field} or a \textbf{Vector Field}.(A quick example of a Vector Field: $f : \Op{n} \to \R$ be a differentiable function, then $\grad f$ is a vector field.)

\textbf{Question.} Given a scalar field $f : \Op{n} \to \R$ and $\gamma \equiv \mathcal{C}$ be a curve, we want to define $\displaystyle\int_{\mathcal{C}} f$. But exactly how we can do this?

\textit{Answer.} $\mathcal{C}$ is a curve, so it is bounded subset of $\R^n$. How about thinking of \textbf{Riemann Integration}? For $n \ge 2$, $\mathcal{C}$ is \textbf{content zero} in $\R^n$. This does not make any sense! The right way is as following.

\

Let, $\gamma :[a,b]\to \R^n$ be a \textbf{smooth curve} (or piecewise smooth) and $\mathcal{C} := \ran(\gamma)$ (on other words path of $\gamma$). Let, $f \in \mathscr{B}(\mathcal{C})$. Given $\mathcal{P} \in \mathscr{P}[a,b]$, $\mathcal{P} : a = t_0 < t_1 < \cdots < t_m = b$.

\

\begin{wrapfigure}{r}{0.5\textwidth}
  \centering
  \includegraphics[width=.98\linewidth]{../figures/lec-23.1.png}
  \caption{Curve $\mathcal{C}$}
\end{wrapfigure}

Let, $I_i = [t_{i-1},t_i]$ be the sub-intervals and $\mathcal{C}_i = \gamma(I_i)$. Since, $\gamma$ is smooth there is nice correspondence between $I_i$ and $\mathcal{C}_i$. Also denote $s_i$ by $\norm{\gamma(t_i) - \gamma(t_i)}$. As previous, define $m_i = \inf_{\mathcal{C}_i}f$ and $M_i = \sup_{\mathcal{C}_i}f$.

\begin{align*}
  U(f,\mathcal{P}) & = \sum_{i=1}^{m}M_i\cdot s_i \\
  L(f,\mathcal{P}) & = \sum_{i=1}^{m}m_i\cdot s_i
\end{align*}

The above expressions are same as upper and lower Riemann sum respectively. This opens up ``\textcolor{violet}{The Pandora's box!}''.

\

We can now use all the theory we used for the standard Riemann Integrals. We say $f$ is line integrable over $\gamma$ if,
\[\inf_{\mathcal{P} \in \mathscr{P}[a,b]} U(f,\mathcal{P}) = \sup_{\mathcal{P} \in \mathscr{P}[a,b]} L(f,\mathcal{P})\]

More over we will write the common value of the above equality as $\displaystyle\int_{\mathcal{C}} f$ and call this ``The Line Integral over curve $\mathcal{C}$''.

\begin{notnBox}
  $\mathscr{R}(\mathcal{C}) = $ set of all Riemann integrable functions over $\mathcal{C}$.
\end{notnBox}

Now we can invoke all theory we derived for 1 variable integration!

\begin{Thm}{}{23:1}
  Let, $\gamma$ be a ``Rectifiable'' smooth(or piecewise smooth) and $\mathcal{C} = \ran(\gamma)$ and $f \in \mathscr{B}(\mathcal{C})$. Then,
  \begin{enumerate}
    \item $f \in C \implies f \in \mathscr{R}(\mathcal{C})$
    \item \[ f \in \mathscr{R}(\mathcal{C}) \iff \lim_{||\mathcal{P}||\to 0} \sum_{i=1}^{m} f(\zeta_i)s_i \hspace{0.1cm} \text{exist and equal to}\int_{\mathcal{C}} f.\] Here, $\zeta_i$ is tag of the interval $I_i$.

    \item (This requires smoothness) If $\gamma$ is $C^1$ and smooth, $f \in \mathscr{R}(\mathcal{C})$, then
          \[\int_{\mathcal{C}} f = \int_a^b f(\gamma(t))\norm{\gamma'(t)} \label{eq:1}\]

  \end{enumerate}
\end{Thm}

\textit{Proof.} \textbf{Exercise.}

\

In the above theorem equation in \ref{eq:1} is also independent of choice of Parametrization of $\gamma$. Because any other smooth parametrized curve $\tilde{\gamma} = \gamma \circ \varphi$ where, $\varphi$ is an onto continuous function.

\

\textbf{\underline{Facts:}} $\mathcal{C}$ be a piecewise smooth, parametrized curve $\gamma$. $f,g \in \mathscr{R}(\mathcal{C})$ and $r \in \R$,then
\begin{itemize}
  \item $\displaystyle\int f+rg = \displaystyle\int f + r\displaystyle\int g $
  \item $f \ge g$ over $\mathcal{C}$ then $\displaystyle\int f \ge \displaystyle\int g$
  \item $\displaystyle\int \abs{f} \ge \abs{\displaystyle\int f}$
  \item If $a<d<b$, if $\gamma_1 := \gamma |_{[a,d]}$ and $\gamma_2 := \gamma |_{[d,b]}$ then
        \[\int_{\mathcal{C}} f = \int_{\gamma_1}f + \int_{\gamma_2} f \]
\end{itemize}

We have resolved the problems for Scalar field. \textbf{What about vector fields?}
\begin{figure}[H]
  \centering
  \includegraphics[width=0.5\textwidth]{../figures/lec-23.2.png}
  \caption{Work done by a constant Force}
\end{figure}
Suppose a particle moves a distance $d$ under a constant force $F$, then work done by the force is $Fd\cos{\theta} = \vec{F}\cdot\vec{d}$.


\begin{wrapfigure}{r}{0.5\textwidth}
  \centering
  \includegraphics[width=.98\linewidth]{../figures/lec-23.3.png}
  \caption{$\vec{F}$ is the vector field over the curve $\gamma$}
\end{wrapfigure}

If the force was not constant throughout the path then how can we calculate work done by that force?
Consider the case where $F$ is the vector field (Force in this case) defined over a curve (path)$\gamma$. Here, $\gamma : [a,b] \to \R^n$ and $\mathcal{C} = \ran(\gamma)$. So, work done throughout the whole path will be,
\[\int_{\mathcal{C}} \vec{F}\cdot d\vec{r}\]
Which is equal to, $\displaystyle\int_a^b \vec{F}(\gamma(t))\cdot \grad \gamma(t) dt$

Now we will look into some examples.

\begin{Eg}{}{}
  Find work done by the field $F(x,y,z) = (xy,xz,yz)$ along the curve $\gamma(t) = (t^2,-t^3,t^4),t \in [0,1]$.

  \textit{Answer.} \begin{align*}
    \gamma'(t)
                              & = (2t,-3t^2,4t^3)                                 \\
    \implies \text{Work done} & = \int_0^1 (-t^5,t^6,-t^7)\cdot(2t,-3t^2,4t^3) dt \\
                              & = -\frac{31}{88}
  \end{align*}
\end{Eg}

\subsection{Line Integration of a Vector Field}

$F:\Op{n} \to \R^n$ be a vector field and $\gamma:[a,b] \to \Op{n}$ be a curve. We consider a partition $\mathcal{P}: a =t_0<t_1<\cdots < t_m=b$, Let $\mathcal{C}_i = \gamma|_{[t_{i-1},t_i]}$ and $\gamma_i = \gamma(t_i)$, $\Delta r_i = \gamma_{i} - \gamma_{i-1}$.

\[R(F;\mathcal{P})= \sum_{i=1}^m F(\gamma_i)\cdot \Delta r_i\]

\begin{align*}
  \int_{\mathcal{C}} \vec{F}\cdot d\vec{r} = \lim_{\norm{\mathcal{P}}\to 0} R(F,\mathcal{P}) \tag{if the limit exists}
\end{align*}

Just like the scalar field, if $\gamma$ is $C^1$ and smooth, then
\begin{equation}
  \int_{\mathcal{C}} \vec{F}\cdot d\vec{r} = \int_a^b F(\gamma(t))\cdot \gamma'(t) \dd t \label{eq:3}
\end{equation}


\end{document}
