\documentclass[../Analysis-3.tex]{subfiles}

\begin{document}
\chapter*{Lecture 9} %Set chapter name
\addcontentsline{toc}{chapter}{Lecture 9} %Set chapter title
\setcounter{chapter}{9} %Set chapter counter
\setcounter{section}{0}
\setcounter{equation}{0}
\setcounter{figure}{0}


\section{Directional Derivatives}
We now introduce another extension of one-dimensional derivative, called the Directional Derivative.

\begin{Def}{Directional Derivative}{}
  Let $ u \in \R^n $ (indicates a direction) be a unit vector and $ f:\Op{n} \to \R $ be a scalar-valued function. Take, $ a \in \Op{n} $, the directional derivative of $ f $ at $ a $ in the direction of $ u $ is defined as
  \begin{align*}
    \left( D_uf \right)\left( a \right) := \lim_{t\to 0}\frac{f(a+tu)-f\left( a \right)}{t}
  \end{align*}
  the rate of change of $ f $ at $ a $ in the direction of $ u $, provided the limit exists.
\end{Def}

Note that $ \left( D_{e_i}f \right)\left( a \right) = \pdv{f}{x_i}\left( a \right) $, i.e., the partial derivatives are directional derivatives along the standard basis vectors.
\smallskip

\begin{wrapfigure}[7]{r}{.32\textwidth}
  \centering
  \def\svgwidth{1in}
  \subimport{../figures/directional_derivative}{directional_derivative.pdf_tex}
  \caption{$ t \mapsto a + tu $}
\end{wrapfigure}

Now, consider the following

\[\begin{tikzcd}
    t &&& {a+tu} &&& {f(a+tu)}
    \arrow["\phi", maps to, from=1-1, to=1-4]
    \arrow["f", maps to, from=1-4, to=1-7]
    \arrow["{\eta(t) = f(a+tu)}"', maps to, curve={height=30pt}, from=1-1, to=1-7]
  \end{tikzcd}\]

Take, $ \eta : \left( -\veps, \veps \right) \to \R $ as $ \eta(t) := f(a+tu) $. If $ f $ is differentiable at $ a $ then $ \eta $ is differentiable at 0 (by chain rule) iff $f$ has directional derivative at $ a $ along $ u $ and,
\begin{align*}
  \eta'(0)
   & = \left( D_uf \right)\left( a \right)                                  \\
   & = Df(\phi(0))\cdot \underbrace{(D\phi)(0)}_{u}                         \\
   & = \begin{pmatrix}
         \pdv{f}{x_i}\left( a \right) & \cdots & \pdv{f}{x_n}\left( a \right)
       \end{pmatrix}\begin{pmatrix}
                      u_1 \\ \vdots \\ u_n \end{pmatrix}
\end{align*}
So,
\[
  \boxed{\eta'(0) = \sum_{i=1}^{n}u_i\pdv{f}{x_i} \left( a \right) = \left( D_uf \right)\left( a \right)}
\]


\section{Gradient}

\begin{Def}{Gradient}{}
  Given a scalar-valued function (a.k.a., scalar field) $ f:\Op{n}\to\R $ and $ x\in\Op{n} $, the Gradient of $ f $ at $ x $ is defined as
  \[
    \grad f(x) := \left\langle f_{x_1}(x), \ldots, f_{x_n}(x) \right\rangle
  \]
  the dual of the total derivative, i.e., $ Df(x)^t $, provided that all the partials $ f_{x_i} $ exists at $ x $.
\end{Def}

Observe that, for a differentiable function $ f: \Op{n} \to \R $ at $ a $
\begin{align*}
  \left( D_uf \right)\left( a \right)
   & = \left( \grad f\right)\left( a \right)\cdot u                                                                                                        \\
   & = \norm{\left( \grad f\right)\left( a \right)}\cos{\theta_u} \tag{where, $\theta_u$ is angle between $\left( \grad f\right)\left( a \right)$ and $u$}
\end{align*}
which tells us, The steepest slope is achieved when $ \theta_u \in \{0, \pi\} $, i.e., when $ u $ points along or opposite to the direction of $ \left( \grad f\right)\left( a \right) $
that means $ \max \left( D_uf \right)\left( a \right) $ is attained at $ \underbrace{u = \frac{\left( \grad f\right)\left( a \right)}{\norm{\left( \grad f\right)\left( a \right)}}}_{\text{provided } \norm{\left( \grad f\right)\left( a \right)} \not=0 } $ (direction of the steepest slope).
Hence, we get the following theorem,
\begin{Thm}{}{}
  Let $ f: \Op{n} \to \R $ be a differentiable function at $ a \in \Op{n} $ and suppose $ \left( \grad f\right)\left( a \right) \not= 0 $ then the vector $\left( \grad f\right)\left( a \right)$ points in the direction of the greatest increment of $ f $ at $ a $ with the greatest rate $ \norm{\left( \grad f\right)\left( a \right)} $
\end{Thm}

\section{Examples}

\begin{Eg}{}{}
  Find the directional derivative of $ f(x,y,z) = x^2yz $ along $ \langle 1,1,-1 \rangle $ at $ a = (1,1,0) $.

  \textit{Solution.} We have the unit vector $ u = \frac{\langle 1,1,-1 \rangle}{\norm{\langle 1,1,-1 \rangle}} = \left\langle \frac{1}{\sqrt{3}}, \frac{1}{\sqrt{3}}, -\frac{1}{\sqrt{3}} \right\rangle $. So,
  \begin{align*}
    \left( D_uf \right)\left( a \right)
     & = \left( \grad f\right)\left( a \right)\cdot u                                                                                       \\
     & = \begin{pmatrix} 2xyz & x^2z & x^2y \end{pmatrix}
    \begin{pmatrix} \frac{1}{\sqrt{3}} \\ \frac{1}{\sqrt{3}} \\ -\frac{1}{\sqrt{3}} \end{pmatrix} \\
  \end{align*}
  Therefore, $ \left( D_uf \right)(1,1,0) = -\frac{1}{\sqrt{3}} $ and hence maximum value of $ \left( D_uf \right)(1,1,0) $ is $ \norm{\left( \grad f\right)(1,1,0)} $ along the unit vector $ \langle 0,0,1 \rangle $.
\end{Eg}

\begin{Eg}{}{cts:pde:ndiff}
  Take,
  \[ f(x,y) = \begin{cases}
      \frac{x^2y}{x^2+y^2} & (x,y) = (0,0)     \\
      0                    & (x,y) \not= (0,0)
    \end{cases} \]
  then $ \left| f(x,y) - f(0,0) \right| = \left| \frac{x^2y}{x^2+y^2} \right| \leq |y| \leq \norm{(x,y)} $ implies that $ f $ is continuous at $ (0,0) $. Now, fix $ u = \langle u_1, u_2 \rangle $ with $ \norm{u} = 1 $. We get,
  \begin{align*}
    \left( D_uf \right)(0,0)
     & = \lim_{t\to 0} \frac{f(tu)- 0}{t}                        \\
     & =\lim_{t\to 0} \frac{1}{t}\cdot\frac{tu_1^2u_2}{1}        \\
     & = u_1^2u_2 \not= 0 \tag{Because, $ u $ is an unit vector}
  \end{align*}
  If we assume $ f $ to be differentiable then $ \left( \grad f \right)(0,0) \cdot \begin{pmatrix} u_1 \\ u_2  \end{pmatrix} = 0 \not= u_1^2u_2 $ which is a contradiction!
\end{Eg}

Example \ref{eg:cts:pde:ndiff} shows that, the existence of all partial and directional derivatives at a point fails to imply differentiability at that point.


\end{document}
