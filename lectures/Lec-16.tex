\documentclass[../Analysis-3.tex]{subfiles}

\begin{document}
\chapter*{Lecture 16} %Set chapter name
\addcontentsline{toc}{chapter}{Lecture 16} %Set chapter title
\setcounter{chapter}{16} %Set chapter counter
\setcounter{section}{0}

\section{Riemann-Darboux Integration}


\begin{Def}{Volume of a closed/open ball}{}
  We define the volume of a closed ball $B^n = \prod \limits_{i=1}^n [a_i, b_i]$ as $\Vol(B^n) = \prod \limits_{i=1}^n (b_i - a_i)$. We also define the volume of the open set $O^n = \prod \limits _{i=1}^n (a_i, b_i)$ to be equal to that of $B^n$, i.e., $\Vol(O^n) = \Vol(B^n)$.
\end{Def}

We now introduce some notation. Fix $i \in \{1,2, \dots, n\}$. We define a partition of the $i^{\text{th}}$ interval $[a_i, b_i]$ as
\[  P_i:\, a_i = a_{i,0} < a_{i,1} < \dots < a_{i,n_i} = b_i \]
and the intervals of its partition as
\[  I_{i,t} = [x_{i,t-1}, x_{i,t}], \forall \ 1 \leq t \leq n_i  \]
and set
\[
  B_{(t_1,t_2, \dots, t_n)}^n = B_\alpha = [x_{1,t_1-1}, x_{1,t_1}] \times \dots \times [x_{n,t_n-1}, x_{n,t_n}] = I_{1,t_1} \times \dots \times I_{n,t_n}
\]
where $\alpha$ is chosen from the indexing set $\Lambda (P) = \{\alpha = (t_1, \dots, t_n) \mid 1 \leq t_i \leq n_i, i=1, \dots, n\}$.

\begin{noteBox}
  \begin{enumerate}
    \item $B^n = \bigcup_{\alpha \in \Lambda (P)} B^n_\alpha$
    \item $\Vol(B^n) = \sum_{\alpha \in \Lambda (P)} \Vol(B_\alpha ^n)$
  \end{enumerate}
\end{noteBox}

We call $\mathscr{P}(B) = \{P_1 \times \dots \times P_n \mid P_i \in \mathscr{P}[a_i, b_i]\}$ the set of all partitions of $B^n$.

\begin{Def}{Refinement of Partitions}{}
  Given $\displaystyle P = \prod_{i=1}^n P_i$ and $\displaystyle\widetilde{P} = \prod_{i=1}^n \widetilde{P}_{i}$ with $P, \widetilde{P} \in \mathscr{P}[a,b]$, then $\widetilde{P}$ is called a refinement of $P$ if $\widetilde{P}_{i} \supset P_i \ \forall i = 1, 2, \dots, n$.
\end{Def}

\begin{Thm}{}{}
  Let $f$ be a bounded function over $B^n$. Let $P, \widetilde{P} \in \mathscr{P}(B^n)$ and $\widetilde{P} \supset P$. Then
  \[  L(f,P) \leq L(f, \widetilde{P}) \leq U(f,\widetilde{P}) \leq U(f,P)  \]
\end{Thm}

\begin{proof}
  Note that $L(f, \widetilde{P}) \leq U(f,\widetilde{P})$ follows directly from the fact that $m_\alpha (\widetilde{P}) \leq M_\alpha (\widetilde{P})\ \forall \widetilde{P} \in \mathscr{P}[a,b]$, where $m_\alpha = \inf_{B^n_\alpha} f$ and $M_\alpha = \sup_{B_\alpha ^n} f$.
\end{proof}

\begin{Cor}{Inequality of upper and lower sums}{}
  For all $ P, \widetilde{P} \in \mathscr{P}(B^n) $, the following inequality holds.
  \[  m \times \Vol(B^n) \leq L(f,P) \leq U(f, \widetilde{P}) \leq M \times \Vol(B^n)  \]
\end{Cor}

We denote $ \mathscr{B}(A) = \{f:A \to \R \mid \sup_A \abs{f} < \infty\} $ as the set of all bounded functions over $ A $ for any $ A \subseteq \R^n $.

\begin{Def}{Upper and Lower Darboux Integrals}{}
  For $f \in \mathscr{B}(B^n)$, we define
  \[  \overline{\int}_{B^n} f = \inf_{P \in \mathscr{P}(B^n)} U(f,P) \text{  and  } \underline{\int}_{B^n} f = \sup_{P \in \mathscr{P}(B^n)} L(f,P)  \]
  as the Upper and Lower Darboux Integrals, respectively.
\end{Def}

We know that $L(f,P) \leq U(f, \widetilde{P})$ for any $P, \widetilde{P} \in \mathscr{P}(B^n)$. Thus, we have $\displaystyle\underline{\int}_{B^n} f \leq \overline{\int}_{B^n} f$.

\begin{Def}{Darboux Integral}{}
  Let $f \in \mathscr{B}(B^n)$. Then $f$ is said to be Riemann-Darboux integrable, or simply integrable, if
  \[  \underline{\int}_{B^n} f = \overline{\int}_{B^n} f  \]
  In this case, we write $\displaystyle\int_{B^n} f \, \dd V = \underline{\int}_{B^n} f = \overline{\int}_{B^n} f$.
\end{Def}

We have $L(f, P) \leq U(f, P')$ for all $P, P' \in \mathscr{P}(B^n)$ by taking the common refinement $\widehat{P} = P \cup P'$. Hence,
\[
  \underline{\int}_{B^n} f \leq \overline{\int}_{B^n} f
\]

\begin{Def}{Darboux Integral}{}
  Let \( f \in \mathscr{B}(B^n) \). \( f \) is said to be Riemann-Darboux integrable if
  \[
    \underline{\int}_{B^n} f = \overline{\int}_{B^n} f
  \]
  In this case, we introduce the notation,
  \[
    \int_{B^n} f \, \dd V = \int_{B^n} f(x_1, \dots, x_n) \, \dd x_1 \cdots \dd x_n = \underline{\int}_{B^n} f = \overline{\int}_{B^n} f
  \]
\end{Def}
At this point, the notation $\displaystyle\int_{B^n} f(x_1, \dots, x_n) \, \dd x_1 \cdots \dd x_n$ \textbf{does not} indicate repeated integration, but we will see that it represents repeated integration for ``nice'' functions.

\end{document}
