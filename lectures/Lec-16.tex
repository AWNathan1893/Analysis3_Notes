% \documentclass[a4paper]{article} 
% \usepackage[utf8]{inputenc}%remove
% \usepackage{../style}%remove
\documentclass[../Analysis-3.tex]{subfiles}

\begin{document}
\chapter*{Lecture 16} %Set chapter name
\addcontentsline{toc}{chapter}{Lecture 16} %Set chapter title
\setcounter{chapter}{16} %Set chapter counter
\setcounter{section}{0}

\section{Riemann-Darboux Integration}



\begin{Def}{Volume of a closed/open ball}{}

  We define the volume of a closed ball $B^n = \prod \limits_{i=1}^n [a_i, b_i]$ as $\mathcal{V}(B^n) = \prod \limits_{i=1}^n (b_i - a_i).$ We also define the volume of the open set $O^n = \prod \limits _{i=1}^n (a_i, b_i)$ equal to that of $B^n$, i.e, $\mathcal{V}(O^n) = \mathcal{V}(B^n)$.

\end{Def}

We now introduce some notation. Fix $i \in \{1,2, \dots, n\}$. We define
\begin{align*}
  P_i:\,  & a_i = a_{i,0} < a_{i,1} < \dots < a_{i,n_i} = b_i   \\
  I_{i,t} & = [x_{i,t-1}, x_{i,t}], \forall \ 1 \leq t \leq n_I
\end{align*}
and set
\[
  B_{(t_1,t_2,\dots ,t_n)}^n = B_\alpha = [x_{1,t_1-1}, x_{1,t_1}] \times \dots \times [x_{n,t_n-1}, x_{n,t_n}] = I_{1,t_1} \times \dots \times I_{n,t_n}
\]
where $\alpha$ is chosen from the indexing set $\Lambda (P) = \{\alpha = (t_1,\dots ,t_n) | 1 \leq t_i \leq n_i, i=1, \dots , n\}$.

\begin{noteBox}

  \begin{enumerate}
    \item $B^n = \bigcup \limits _{\alpha \in \Lambda (P)} B^n_\alpha$
    \item $\mathcal{V}(B^n) = \sum \limits _{\alpha \in \Lambda (P)} \mathcal{V}(B_\alpha ^n)$
  \end{enumerate}

\end{noteBox}

We call $\mathcal{P}(B) = \{P_1 \times \dots \times P_n | P_i \in \mathcal{P}[a_i, b_i]\}$ as the set of all partitions of $B^n$.

\begin{Def}{Refinement of Partitions}{}

  Given $P = \prod \limits _{i=1}^n P_i$ and $\widetilde{P} = \prod \limits _{i=1}^n \widetilde{P}_{i}$ with $P, \widetilde{P} \in \mathcal{P}[a,b]$, then $\widetilde{P}$ is called a refinement of $P$ if $\widetilde{P}_{i} \supset P_i \ \forall i = 1,2, \dots, n$.

\end{Def}

\begin{Thm}{}{}

  Let $f$ be a bounded function of $B^n$. Let $P, \widetilde{P} \in \mathcal{P}[B^n]$ and $\widetilde{P} \supset P$. Then
  \[L(f,P) \leq L(f, \widetilde{P}) \leq U(f,\widetilde{P}) \leq U(f,P)\]

\end{Thm}

\begin{proof}
  Note that $L(f, \widetilde{P}) \leq U(f,\widetilde{P})$ follows directly from the fact that $m_\alpha (\widetilde{P}) \leq M_\alpha (\widetilde{P})\ \forall \widetilde{P} \in \mathcal{P}[a,b]$ where $m_\alpha = \text{inf}_{B^n_\alpha} f$ and $M_\alpha = \text{sup}_{B_\alpha ^n} f$.
\end{proof}

\begin{Cor}{}{}
  $m \times \mathcal{V}(B^n) \leq L(f,P) \leq U(f, \widetilde{P}) \leq M \times \mathcal{V}(B^n),\ \forall P, \widetilde{P} \in \mathcal{P}[B^n]$
\end{Cor}

\begin{Def}{Upper and Lower Darboux Integrals}{}
  For $f \in \mathcal{B}[B^n],$ define
  $$\overline{\int \limits _{B^n}} f = \inf_{P \in \mathcal{P}(B^n)} U(f,P) \text{  and }\  \underline{\int \limits _{B^n}} f = \sup_{P \in \mathcal{P}(B^n)} L(f,P)$$
  as the Upper and Lower Darboux Integrals respectively.
\end{Def}

We know that $L(f,P) \leq U(f, \widetilde{P})$ for any $P, \widetilde{P} \in \mathcal{P}[B^n]$. Thus, we get that $\underline{\int \limits _{B^n}} f \leq \overline{\int \limits _{B^n}} f$.

\begin{Def}{Darboux Integral}{}
  Let $f \in \mathcal{B}(B^n)$. Then $f$ is said to be Riemann-Darboux Integrable or just Integrable if
  \[\underline{\int \limits _{B^n}} f = \overline{\int \limits _{B^n}} f\]
  In this case, we write $\int \limits_{B^n}f dV = \underline{\int \limits _{B^n}} f = \overline{\int \limits _{B^n}} f$.
\end{Def}

We have, for all \( P, P' \in \mathcal{P}[B^n] \),
\[
  L(f, P) \leq U(f, P')
\]
by taking the common refinement \( \widehat{P} = P \cup P' \). Hence,
\[
  \underline{\int \limits _{B^n}} f \leq \overline{\int \limits _{B^n}} f
\]

\begin{Def}{Darboux Integral}{}
  Let \( f \in \mathcal{B}[B^n] \). \( f \) is said to be Riemann-Darboux integrable if
  \[
    \underline{\int \limits _{B^n}} f = \overline{\int \limits _{B^n}} f
  \]
  In this case we introduce the notation,
  \[
    \int_{B^n} f \dd V = \int_{B^n} f(x_1, \dots, x_n) \, \dd x_1 \cdots \dd x_n = \underline{\int \limits _{B^n}} f = \overline{\int \limits _{B^n}} f
  \]
\end{Def}
At this point, the notation \( \int_{B^n} f(x_1, \dots, x_n) \, \dd x_1 \cdots \dd x_n \) \textbf{does not} indicate repeated integration, but we will see that it is so for ``nice'' functions.

\end{document}