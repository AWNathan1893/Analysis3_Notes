\documentclass[../Analysis-3]{subfiles}

\begin{document}
\chapter*{Lecture 22} %Set chapter name
\addcontentsline{toc}{chapter}{Lecture 22} %Set chapter title
\setcounter{chapter}{22} %Set chapter counter
\setcounter{section}{0}

\section{Curves and Surfaces}

We now study the basic concepts of curves and surfaces as subsets of $ \R^2 $ or $ \R^3 $ (mainly) with a given parametrization, but also as subsets defined by equations. The connection from equations to parametrization is drawn by means of the Implicit function theorem.

\begin{Def}{Parametrized Curve}{}
    A \textbf{parametrized curve} is a continuous function $\gamma : [a,b] \to \R^n$. We say that parametrized curve is $C^1$ if $t \mapsto \gamma_i(t)$ is $C^1$ for all $i = 1,\dots,n$. A parametrized curve $\gamma : I \to \R^n$ is \textbf{smooth} if $\gamma'(t) \neq \mathbf{0}$ for all $t \in I$. The \textbf{path} of a parametrized curve $\gamma$ is the set
    \[
        \qty{\gamma(t) \mid t \in [a,b]}
    \]
\end{Def}

Let us consider some examples.

\begin{Eg}{}{}
    \begin{enumerate}
        \item Let $\gamma : [0,1] \to \R^2$ be $\gamma(t) = (1-2t,2+t)$. Clearly $\gamma$ is $C^1$ and $\gamma'(t) = (-2,1) \neq \mathbf{0}$ and thus $\gamma$ is smooth.

        \item $\gamma : [0,2\pi] \to \R^2$ given by $\gamma : t \mapsto (r \cos t, r \sin t)$ where $r > 0$ is constant. Then $\gamma'(t) = (-r\sin t, r \cos t) \neq \mathbf{0} \, \forall \, t \in [0,2\pi]$, thus $\gamma$ is smooth.

        \item Fix $r > 0$ and $c \neq 0$, and define
          \begin{align*}
              \gamma : [0, n\pi] & \to \R^3                       \\
              t                  & \mapsto (r\cos t, r\sin t, ct)
          \end{align*}
          Then $\gamma'(t) = ( -r\sin t, r \cos t, c ) \neq \mathbf{0}$ and hence $\gamma$ is smooth. The path of $\gamma$ is called a \textbf{helix}.

        \item $\gamma : [-1,1] \to \R^2$ given by $\gamma(t) = (|t|, t)$, then $\gamma$ is not $C^1$, hence it is not smooth.
        \item $\gamma : [0,1] \to \R^2$ given by $\gamma(t) = (0,t^2)$, then even though $\gamma$ is $C^1$, but it is not smooth as $\gamma'(0) = \mathbf{0}$.
        \item $\gamma : [0,2\pi] \to \R^2$ given by $\gamma : t \mapsto (r\cos t, r\sin t)$, then path of $\gamma$ is given by
          \begin{align*}
              \{ (r\cos t, r\sin t) \mid t \in [0,2\pi] \} & = \{ (x,y) \mid x^2 + y^2 = r^2\} \\
                                                           & = \mbox{path of } \tilde{\gamma}
          \end{align*}
          where $\tilde{\gamma}(t) = (r \cos 2t, r \sin 2t)$.
    \end{enumerate}
\end{Eg}

\begin{Def}{Piecewise Smooth Curve}{}
    A parametrized curve $\gamma : [a,b] \to \R^n$ is called piecewise smooth if there exists a partition $a = t_0 < t_1 < \cdots < t_m = b $ such that
    \[
        \gamma\vert_{[t_{i-1}, t_i]} \mbox{ is smooth } \forall \, i \in [m]
    \]
\end{Def}

\begin{Def}{Equivalent Curves}{}
    Two parametrized curves $\gamma : [a,b] \to \R^n$ and $\tilde{\gamma} : [\tilde{a}, \tilde{b}] \to \R^n$ are equivalent, denoted by $\gamma \sim \tilde{\gamma}$ if there exists a strictly increasing surjective function which is differentiable (even $C^1$), $\varphi : [\tilde{a},\tilde{b}] \to [a,b]$ such that $\tilde{\gamma} = \gamma \circ \varphi$.

    \[\begin{tikzcd}
            {[\tilde{a}, \tilde{b}]} && {\R^n} \\
            & {[a, b]}
            \arrow["\varphi", from=1-1, to=2-2]
            \arrow["\gamma", from=2-2, to=1-3]
            \arrow["{\tilde{\gamma}}", from=1-1, to=1-3]
        \end{tikzcd}\]
\end{Def}

\begin{Def}{}{}
    Let $\gamma : [a,b] \to \R^n$ be a $C^1$ curve, then
    \begin{enumerate}
        \item[(i)] $\| \gamma'(t) \| :=$ speed of $\gamma$ at time $t$.
        \item[(ii)] $\displaystyle\int_a^b \| \gamma'(t) \| \dd t :=$ arc length of $\gamma$.
    \end{enumerate}
\end{Def}

Let's try to look at more natural how equation (ii) in the previous definition gives us the arc length of a curve $\gamma$.

\

Let $\gamma : [a,b] \to \R^n$ and let $\mathcal{P} := a = t_0 < t_1 < \cdots < t_m = b $ be a partition of the interval $[a,b]$. Now define
\[
    l(\gamma,\mathcal{P}) = \sum_{i=1}^m \| \gamma(t_{i-1}) - \gamma(t_i) \|
\]


\begin{Def}{}{}
    A curve $\gamma : [a,b] \to \R^n$ is \textbf{rectifiable} or said to have arc length if
    \[
        \lim_{\overset{\|\mathcal{P}\| \to 0}{\mathcal{P} \in \mathscr{P}[a,b]}} l(\gamma,\mathcal{P}) = l(\gamma) \mbox{ exists}
    \]
    which is equivalent to saying that for all $\varepsilon > 0$, there exists $\delta >0$, such that
    \[
        \norm{l(\gamma,\mathcal{P}) - l(\gamma)} < \varepsilon \quad \forall \, \mathcal{P} \in \mathscr{P}([a,b]) \mbox{ such that } \norm{\mathcal{P}} < \delta
    \]
\end{Def}

\begin{Thm}{}{22:1}
    For a piecewise smooth curve $\gamma : [a,b] \to \R^n$ it is rectifiable and $l(\gamma) = \displaystyle\int_a^b \norm{\gamma'(t)} \dd t $.
\end{Thm}

\textbf{Remark.} Rectifiable curve $\not\Rightarrow$ piecewise smooth, counter example: \href{https://en.wikipedia.org/wiki/Cantor_function}{Cantor's function} (popularly called the \href{https://mathweb.ucsd.edu/~bseward/140b_spring20/Devils-Staircase.pdf}{Devil's staircase}).

\begin{Thm}{}{22:2}
    Let $\gamma : [a,b] \to \R^n$ be a rectifiable parametrized curve and let $\tilde{\gamma} = \gamma \circ \varphi$, where $\varphi$ is a strictly increasing surjective and continuous function, then $\tilde{\gamma}$ is rectifiable and $l(\gamma) = l(\tilde{\gamma})$.
\end{Thm}

\begin{Thm}{}{22:3}
    Let $\gamma : [a,b] \to \R^n$ be a smooth curve, then there exists a parametrization $\varphi$ such that $\| \tilde{\gamma}'(s) \| = 1$ for all $s \in [c,d]$.
\end{Thm}


\end{document}