\documentclass[../Analysis-3]{subfiles}

\begin{document}
\chapter*{Lecture 3} %Set chapter name
\addcontentsline{toc}{chapter}{Lecture 3} %Set chapter title
\setcounter{chapter}{3} %Set chapter counter
\setcounter{section}{0}

\section{Introduction}
We denote the set of limit points of $ S \subseteq \R^n $ by $ S' $. Let $ f: S \to \R^m $ and $ a \in S $. $ f $ is continuous at $ a $ iff for all $ \veps > 0 $, there is $ \delta > 0 $ such that
\begin{align*}
  \norm{f(x) - f(a)} < \veps & \quad \forall \, \norm{x-a} < \delta, x \in S \\
                             & \iff                                          \\
  f(B_\delta(a) \cap S)      & \subseteq B_\veps(f(a))
\end{align*}
For $ a \in S' $, we then get that $ f $ is continuous at $ a $ iff
\[ \lim_{\norm{x-a} \to 0} \norm{f(x) - f(a)} = 0 \iff \lim_{\norm{h} \to 0}\norm{f(a+h) - f(a)} = 0 \]
But $ \norm{h} \to 0 \iff h \to 0 $, and so we have $ f $ is continuous at $ a \in S'$ iff \[ \lim_{h \to 0}\norm{f(a+h) - f(a)} = 0 \]

The proof of the following theorem is left as an exercise.
\begin{Thm}{}{}
  Let $ S \subseteq \R^n, a \in S', b \in \R^m, f : S \to \R^m $. The following are equivalent:
  \begin{enumerate}[label = (\roman*)]
    \item $ \displaystyle \lim_{x \to a}f = b $
    \item $ \forall \, \{x_p\} \subseteq S \setminus \{a\} $ with $ x_p \longrightarrow a $, we have $ f(x_p) \longrightarrow b $
    \item $ \displaystyle \lim_{x \to a}\norm{f(x) - b} = 0 $
  \end{enumerate}
\end{Thm}
\textbf{Note:} If $ a \in S $, we can take $ b = f(a) $ and get analogous results for continuity at $ a $.

\section{Properties of Continuous functions}

\begin{Def}{Continuity on sets}{}
  Let $ S \subseteq \R^n $ and $ f: S \to \R^m $. $ f $ is continuous on $ S $ if it is continuous at all $ a \in S $.
\end{Def}
\textbf{Note:} It is convenient to take $ S $ to be open, as $ f $ is continuous at any isolated points of $ S $ vacuously.
\pagebreak

\begin{Thm}{}{t1}
  Let $ S \subseteq \R^n, f: S \to \R^m $. The following are equivalent:
  \begin{enumerate}[label = (\arabic*)]
    \item $ f $ is continuous on $ S $.
    \item $ \forall \, \{x_p\} \subseteq S $ with $ x_p \longrightarrow a \in S $, we have $ f(x_p) \longrightarrow f(a) $
    \item (Assuming $ S $ is open) $ f^{-1}(\mathcal O) $ is open for all $ \mathcal O \subseteq \R^m $ open.
    \item (Assuming $ S $ is open) $ f^{-1}(C) $ is closed for all $ C \subseteq \R^m $ closed.
  \end{enumerate}
\end{Thm}

\begin{proof} We have the following cases.
  \begin{enumerate}[label = $\bullet$]
    \item $ (3) \iff (4) $\\
          This is true as if $ g: X \to Y $ then for all $ A \subseteq Y $, $ g^{-1}(Y \setminus A) = X \setminus g^{-1}(A) $.
    \item $ (1) \iff (2) $\\
          True by \ref{th:t1}

    \item $ (1) \implies (3) $\\
          Let $ \mathcal O \subseteq \R^m $ be open, and without loss of generality, $ f^{-1}(\mathcal O) \neq \phi $.

          Let $ a \in f^{-1}(\mathcal{O}) $ so that $ f(a) \in \mathcal O $. Hence, there is $ r > 0 $ such that $ B_r(f(a)) \subseteq \mathcal O $. By continuity, there is $ \delta > 0 $ such that
          \begin{align*}
            f(B_\delta(a))       & \subseteq B_r(f(a)) \subseteq \mathcal O \\
            \implies B_\delta(a) & \subseteq f^{-1}(\mathcal{O})
          \end{align*}
          Hence, $ \mathcal O $ is open.

    \item $ (3) \implies (1) $\\
          Fix $ a \in S $ and let $ \veps > 0 $. As $ B_\veps(f(a)) $ is open in $ \R^m $, we have $ f^{-1}(B_\veps(f(a))) $ is open in $ \R^n $. But $ a \in f^{-1}(B_\veps(f(a))) $, and so, there is $ \delta > 0 $ such that
          \begin{align*}
            B_\delta(a)             & \subseteq f^{-1}(B_\veps(f(a))) \\
            \implies f(B_\delta(a)) & \subseteq B_\veps(f(a))
          \end{align*}
          Hence, $ f $ is continuous at $ a $ for all $ a \in S $.
  \end{enumerate}
\end{proof}

This theorem gives us a huge simplification. Recall that $ x \longrightarrow y $ iff $ \Pi_i(x) \longrightarrow \Pi_i(y) $ for all $ i $.

Now consider some $ \{x_p\} $ with $ x_p \longrightarrow a $. We have
\[ f(x_p) \longrightarrow f(a) \iff \Pi_i(f(x_p)) \longrightarrow \Pi_i(f(a)) \, \forall \, i  \]
That is, $ f $ is continuous iff $ f $ is continuous coordinatewise! Hence, for talking about continuity, it is enough to discuss real valued functions rather than $ f: \R^n \to \R^m $ for arbitrary $ m $.
\pagebreak

\section{Examples}
\begin{Eg}{}{}
  Consider the function,
  \begin{align*}
    f: \R^2 \setminus \{(0,0)\} & \to \R                \\
    f(x,y)                      & = \frac{2xy}{x^2+y^2}
  \end{align*}
  Consider the line $ L_1 $ defined by $ y=0 $ and approach $ (0,0) $ from the right ($ x \longrightarrow 0^{+} $). We have,
  \[ f\big|_{L_1} \equiv 0 \implies \lim_{(x,y) \to 0 \text{ along }L_1} f = \lim_{n \to \infty}f\left(0,\frac 1n\right) =  0 \]

  Now consider the line $ L_2 $ defined by $ x=y $. We have,
  \[ f\big|_{L_2} \equiv 1 \implies \lim_{(x,y) \to 0 \text{ along }L_2} f = \lim_{n \to \infty}f\left(\frac 1n,\frac 1n\right) =  1 \]

  Hence, $ \displaystyle \lim_{(x,y)\to(0,0)}f $ does not exist.
\end{Eg}
\textbf{Note:} The approach in the above example is often useful for showing non-existence of limits, or that a function is not continuous.
\msk

\begin{Eg}{}{}
  We wish to compute $ \displaystyle \lim_{(x,y) \to (0,0)}\frac{x^3}{x^2+y^2} $. We have, for all $ (x,y) \neq (0,0) $,
  \[ \abs{\frac{x^3}{x^2+y^2}} \leq \abs{\frac{x^3}{x^2}} = \abs{x} \leq \norm{(x,y)} \]
  and because $ \norm{(x,y)} $ goes to $ 0 $ as $ (x,y) \longrightarrow (0,0) $, we get
  \[ \displaystyle \lim_{(x,y) \to (0,0)}\frac{x^3}{x^2+y^2} = 0 \]
  Hence, the function
  \[ f(x,y) =
    \begin{cases}
      \frac{x^3}{x^2+y^2}, \, (x,y) \neq (0,0) \\
      0, \, (x,y) = (0,0)
    \end{cases} \]
  is continuous at $ (0,0) $.
\end{Eg}

\begin{Eg}{}{}
  Consider $ \displaystyle \lim_{(x,y) \to (0,0)} \frac{\sin (x^2+y^2)}{x^2+y^2} $. We have $ (x^2+y^2) \longrightarrow 0 $ as $ (x,y) \to (0,0) $, and hence, we get
  \[ \lim_{(x,y) \to (0,0)} \frac{\sin (x^2+y^2)}{x^2+y^2} = \lim_{t \to 0}\frac{\sin t}{t} = 1 \]
\end{Eg}
\pagebreak

\textbf{Exercise.} Let $ S \subseteq \R^n, a \in S' $ and $ f,g: S \to \R $. Suppose $ \displaystyle \lim_{x \to a}f = \alpha, \lim_{x \to a}g = \beta $ exist. Show that:
\begin{enumerate}[label = (\roman*)]
  \item $ \displaystyle \lim_{x \to a}(rf+g) = r\alpha + \beta, \, \forall \, r \in \R $
  \item $ \lim_{x \to a}fg = \alpha \beta $
  \item If $ \beta \neq 0 $, $ \displaystyle \lim_{x \to a} \frac{f}{g} = \frac{\alpha}{\beta} $
  \item If $ f \leq h \leq g $ for some $ h: S \to \R $ and $ \alpha = \beta $, then $ \displaystyle \lim_{x \to a}h = \alpha $
\end{enumerate}
\textbf{Note:} Similar results hold for continuity as well, using which we get the next examples.
\msk

\begin{Eg}{Some classes of continuous functions}{}
  \begin{enumerate}[label = (\arabic*)]
    \item The projection maps $ \Pi_i : \R^n \to \R $ are continuous.
    \item $ x_i \in \R[x_1, \dots, x_n] $ is continuous for all $ i $.
    \item $ x_i^2 \in \R[x_1, \dots, x_n] $ is continuous for all $ i $.
    \item All monomials in $ \R[x_1, \dots, x_n] $ are continuous.
    \item Any $ p \in \R[x_1, \dots, x_n] $ is continuous.
    \item $ \frac{p}{q} $ is continuous at $ a \in \R^n $, where $ p,q \in \R[x_1, \dots, x_n] $ and $ q(a) \neq 0 $.
  \end{enumerate}
\end{Eg}

\end{document}