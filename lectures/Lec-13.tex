\documentclass[../Analysis-3]{subfiles}

\begin{document}
\chapter*{Lecture 13} %Set chapter name
\addcontentsline{toc}{chapter}{Lecture 13} %Set chapter title
\setcounter{chapter}{13} %Set chapter counter
\setcounter{section}{0}

\section{Recall (Taylor's Theorem for one variable)}

Let, $ f : \Op{n} \to \R $ be $C^k \ (k \in \N)$

\begin{Def}{Taylor's Polynomial}{}
    \[ p_{a,k}(a+h) = \sum_{n = 0}^{k} \frac{f^{(n)}(a)}{n!} h^{n} \ \forall\ a+h \in \Op{1} \]
    is called the \textbf{Taylor's Polynomial} of $f$ around $a$.
\end{Def}

\textbf{Question.} ``$f(x) \thickapprox  p_{a,k}(x);\ x$ close to $a$''?
\begin{align*}
    p_{a,k}(x) = \sum_{n = 0}^{k} \frac{f^{(n)}(a)}{n!} (x-a)^{n}
\end{align*}
And take, $f(x) - p_{a,k}(x) = r_{a,k}(x)$

\begin{Thm}{Taylor's Theorem}{}
    Let $ f: \Op{1} \to \R $ be $C^{k+1}$
    \[ f(x) = p_{a,k}(x) + r_{a,k}(x) \]
    where, \[r_{a,k} = \frac{f^{k+1}(c)}{(k+1)!} (x-a)^{k+1} \hspace{5mm} \text{for some $c$ in between $a$ and $x \in \Op{1}$ }\]
\end{Thm}

\

\begin{notnBox}{}{}
    
\begin{itemize}
  \item  $\alpha = ( \alpha_1 , \alpha_2 , \cdots , \alpha_n) \in \Z_{+}^{n}$ \\
  \item $ |\alpha| = \sum_{i = 1}^{n} \alpha_i $ \\
  \item $\alpha ! = \alpha_1 ! \alpha_2 ! \cdots \alpha_n !$ \\
  \item $ \partial^{\alpha} = \frac{\partial^{|\alpha|}}{\partial {x_1}^{\alpha_1} \cdots \partial {x_n}^{\alpha_n}}$ \\
  \item $ h^{\alpha} = {h_1}^{\alpha_1} \cdots {h_n}^{\alpha_n}$

\end{itemize}



\end{notnBox}

\begin{Thm}{Taylor's Theorem in Multivariate Case}{}
    Let, $ f: \Op{n} \to \R $ be a $C^{k+1}$ function (Assume, $\Op{n}$ is convex). If $h, a + h \in \Op{n} $ , then 
    \[ f(a+h) = \sum_{| \alpha | \leq k } \frac{1}{\alpha !} ({\partial}^2 f ) (a) h^{\alpha} + r_{a,k} (h)   \] 
    where, \[ r_{a,k} (h) = \sum_{| \alpha | \leq k } \frac{1}{\alpha !} ({\partial}^2 f ) (a + c h) h^{\alpha} \hspace{5mm} \text{for some $c \in (0,1)$} \]
    
\end{Thm}

\begin{proof}

\[\begin{tikzcd}
        {[0,1]} &&& {\mathcal{O}_n} &&& {\mathbb{Z}} \\
        t &&& {a+th} &&& {f(a+th)}
        \arrow[from=1-1, to=1-4]
        \arrow["f", from=1-4, to=1-7]
        \arrow[from=2-1, to=2-4]
        \arrow[from=2-4, to=2-7]
        \arrow["\large{\eta}"', curve={height=30pt}, from=2-1, to=2-7]
\end{tikzcd}\]

$\therefore$ $\eta$ is a $C^{k+1}$ at $0$ .

$\eta (t) = f(a+th) \hspace{3mm} \forall t \in [0,1]$

\begin{align*}
    \therefore \eta'(t) &= \nabla f (a+th) . h \\
     &= (\nabla . h) f(a+th)  \hspace{1cm} \text{where $\nabla . h = \sum_{i=1}^n \frac{\partial}{\partial x_i} h_i$}
\end{align*}

\begin{clmBox}{}
    $\eta^{(m)}(t) = (\nabla . h)^m f (a+th) \hspace{1cm} \forall 0 \leq m \leq k+1 \hspace{1cm}$   
    
    \
    
    where, \[(\nabla . h)^m = \sum_{| \alpha | = m } \frac{m!}{\alpha!} h^{\alpha} \partial^{\alpha}\]
\end{clmBox} 

\begin{align*}
    \eta'(t) &= \nabla f (a+th) . h \\
    &= \sum_{i=1}^n f_{X_i} (a+th) h_i
\end{align*}

\begin{align*}
    \implies \eta''(t) &= \frac{d}{dt}\left(\sum_{i=1}^n f_{X_i} (a+th) h_i\right) \\
    &= \sum_{i=1}^n \left[\frac{d}{dt}\left( f_{X_i} (a+th) h_i\right)\right] \\
    &= \sum_{i=1}^n h_i \sum_{j=1}^n f_{X_i X_j} (a+th) h_j \\
    &= \sum_{i,j=1}^n h_i h_jf_{X_i X_j} (a+th) \\
    &= (\nabla . h)^2 f (a+th)
\end{align*}

So, by induction, $\eta^{(m)}(t) = (\nabla . h)^m f (a+th) \hspace{1cm} 0 \leq m \leq k+1 $

\

where, \[(\nabla . h)^m = \sum_{| \alpha | = m } \frac{m!}{\alpha!} h^{\alpha} \partial^{\alpha}\]

By One variable Taylor's Theorem, 
\[ \eta(1) = p_{0,k}(1) + r_{0,k} (c) \hspace{1cm} \text{for some  } c \in (0,1) \]

Now, \[ p_{0,k}(1) = \eta(0) + \frac{\eta '(0)}{1!} + \ldots + \frac{\eta^{(k)}(0)}{k!}\]
\[ r_{0,k} (c) = \frac{\eta^{(k+1)}(c)}{(k+1)!} \]

By substituting $\eta^{(m)}(t)$ in this , we can prove that :-
\[ f(a+h) = \sum_{| \alpha | \leq k } \frac{1}{\alpha !} ({\partial}^2 f ) (a) h^{\alpha} + r_{a,k} (h) \]
\end{proof}

\begin{Eg}{For $n=2$}{}
 $f : \mathcal{O}_2 \to \R$ be a $C^2$ function.
 \[ f(a+h) = f(a) + \left(\nabla f(a)\right).h + \frac{1}{2}h^t H_f(a+ch)h \label{eq_pj} \tag{*} \]

 where, \[ H_f =  \begin{bmatrix}
    f_{X X}(a) & f_{X Y}(a) \\
    f_{X Y}(a) & f_{Y Y}(a)
 \end{bmatrix} \]
\end{Eg}

\begin{Thm}{Extremum}{}
Let $f : \mathcal{O}_2 \to \R$ be a $C^2$ function on $\mathcal{O}_2$ \\ 
Let $Df(a) = 0$ , write \[ H_f =  \begin{bmatrix}
    f_{X X}(a) & f_{X Y}(a) \\
    f_{X Y}(a) & f_{Y Y}(a)
 \end{bmatrix} \] \\ 
 Then, \begin{enumerate}
    \item $f(a)$ is a local maximum if $f_{X X} (a) < 0$ and $|H_f(a)| > 0$ \\
    \item $f(a)$ is a local minimum if $f_{X X} (a) > 0$ and $|H_f(a)| > 0$ \\
    \item $a$ is a saddle point if $|H_f(a)| < 0$
 \end{enumerate}
\end{Thm}

\begin{proof} \vspace{3mm}
    $\exists r > 0 $ such that $a+h \in B_r(a) \subseteq \mathcal{O}_2$ \\ 
    By $\eqref{eq_pj}$ , \[ f(a+h) - f(a) = \left(\nabla f(a)\right).h + \frac{1}{2}h^t H_f(a+ch)h \] 

\underline{To prove No (2) :-}

\

Assumptions $\implies$ $H_f(a)$ is positive definite

Hence, by the lemma , $H_f(x)$ is positive definite $\forall x \in$ nbd of $a$

$\implies h^t H_f(x) h > 0 \hspace{1cm} \forall h \neq 0$

$\implies f(x) - f(a) >0 \hspace{1cm} \forall x$ in a nbd of $a$

$\implies$ $f(a)$ is a local minimum.

\end{proof}

\begin{Eg}{}{}
    Find the critical Points and discuss the nature of the function \[ f(x,y) = x^3 - 6x^2 - 8y^2 \]

 \textbf{Solution :-}  
 \begin{align*}
    & \nabla f = 0 \\
    \implies & f_X = 3x^2 - 12x = 0 \text{  and  } f_Y = -16y = 0 \\
    \implies x = 0 , 4 \text{  and  } y = 0
 \end{align*}

$\therefore  (0,0) , (4,0)  $ are stationary points.

Also, \[ f_{X X} = 6x - 12 \text{  and  } f_{Y Y} = -16 \]

\[ |H_f(0,0)| =  \begin{vmatrix}
    -12 & 0 \\
    0 & -16
 \end{vmatrix} > 0 \text{  and  } f_{X X}(0,0) = -12 < 0\]

Hence, $(0,0)$ is a local maximum.

\[ |H_f(4,0)| =  \begin{vmatrix}
        12 & 0 \\
        0 & -16
 \end{vmatrix} < 0 \]    

 Hence, $(4,0)$ is a saddle point.
\end{Eg}












\end{document}