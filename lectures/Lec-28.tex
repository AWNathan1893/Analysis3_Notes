\documentclass[../Analysis-3.tex]{subfiles}

\begin{document}
\chapter*{Lecture 28} %Set chapter name
\addcontentsline{toc}{chapter}{Lecture 28} %Set chapter title
\setcounter{chapter}{28} %Set chapter counter
\setcounter{section}{0}
\setcounter{equation}{0}
\setcounter{figure}{0}


\section{Green's Theorem}
\begin{Thm}{$\R^2$ version of Green's Theorem}{gtm}
  Let $\mathcal{R} \subseteq \R^2 $ be a simply connected domain with boundary curve $\mathcal{C}$ where parametrization is taken in anti-clockwise direction. Let $\vec{F} = (P,Q)$ be a $C^1$ vector field on $\mathcal{R}$, then
  \[
    \int_{\mathcal{C}} \vec{F} \cdot \dd r := \int_{\mathcal{C}} P \, \dd x + Q \, \dd y = \int_{\mathcal{R}} \left( \pdv{Q}{x} - \pdv{P}{y}\right) \, \dd A
  \]
\end{Thm}

\begin{proofFig}{\includegraphics[width=.78\linewidth]{../figures/lec-28.1.png}}{A simple region}{\label{R2:green:thm}}{10}{.5\textwidth} (for Simple region)

  Let $\mathcal{R} = \qty{(x,y) \mid a\leq x\leq b, \varphi_1(x) \leq y \leq \varphi_2(x)}$ be a simple region. Here $\mathcal{C} = \mathcal{C}_1 \ \cup \ \mathcal{V}_2 \ \cup \ \mathcal{C}_2 \ \cup \ \mathcal{V}_1$ is the curve bounding the region along anti-clockwise direction (as shown in Figure \ref{R2:green:thm}).

  \

  Now,
  \begin{align*}
    -\int_{\mathcal{R}} \pdv{P}{y} \, \dd A
     & = -\int_{a}^{b} \int_{\varphi_1(x)}^{\varphi_2(x)} \pdv{P}{y} \, \dd y \, \dd x \\
     & = - \int_{a}^{b} (P(x,\varphi_2(x))-P(x,\varphi_1(x))) \, \dd x                 \\
  \end{align*}

  The curves $\mathcal{C}_1,\mathcal{C}_2,\mathcal{V}_1,\mathcal{V}_2$ can be explicitly written as,
  \begin{align*}
    \mathcal{V}_1 & = \qty{(a,t) \mid \varphi_1(a) \leq t \leq \varphi_2(a)} \\
    \mathcal{C}_1 & = \qty{(x,\varphi_1(x)) \mid a \leq x \leq b}            \\
    \mathcal{V}_2 & = \qty{(a,t)\mid \varphi_1(b) \leq t \leq \varphi_2(b)}  \\
    \mathcal{C}_2 & = \qty{(x,\varphi_2(x)) \mid a \leq x \leq b}
  \end{align*}

  We compute the integrals for $ P $ over these curves and obtain,
  \begin{align*}
    \int_{\mathcal{V}_1} P \, \dd x    & = \int_{\mathcal{V}_1} P(x(t),y(t)) \dv{x(t)}{t} \dd t = 0 \\
    \int_{\mathcal{C}_1} P \, \dd x    & = \int_{a}^{b} P(t,\varphi_1(t)) \, \dd t                  \\
    \int_{\mathcal{C}_2} P \, \dd x    & = \int_{a}^{b} P(t,\varphi_2(t)) \, \dd t                  \\
    \implies \int_{\mathcal{C}} P\dd x & = -\int_{\mathcal{R}} \pdv{P}{y} \, \dd A
  \end{align*}

  By similar mechanism we can show $\displaystyle\int_{\mathcal{C}} Q \, \dd y = \int_{\mathcal{R}} \pdv{Q}{y} \, \dd A$. The rest follows from here.
\end{proofFig}

\begin{Eg}{}{}
  Let $\mathcal{C}$ be the boundary of $[0,1]^2$, i.e., $\partial [0,1]\times[0,1] = \mathcal{C}$. Evaluate

  \[\int_{\mathcal{C}} \langle x^2-y^2,2xy \rangle\]
  \textit{Solution.} We can decompose $\mathcal{C} = \mathcal{C}_1 \cup \mathcal{C}_2 \cup \mathcal{C}_3 \cup \mathcal{C}_4$ (as in the following picture)

  \begin{wrapfigure}{r}{0.25\textwidth}
    \centering
    \includegraphics[width=.78\linewidth]{../figures/lec-28.2.png}
    \caption{$\partial( [0,1]^2)$}
  \end{wrapfigure}

  Let $P(x,y) = x^2 - y^2, Q(x,y) = 2xy$. Then the integral,

  \begin{align*}
    \int_{\mathcal{C}} P \, \dd x + Q \, \dd y
     & = \iint_{[0,1]^2} (2y + 2y) \, \dd A   \tag{Green's Theorem} \\
     & = \int_{0}^1 \int_0^1 4y \, \dd y \, \dd x                   \\
     & = 2
  \end{align*}

  If we try to calculate the integral \textbf{directly}, we will end up getting same result.
\end{Eg}

\begin{tcolorbox}
  \textbf{Area of a closed Region.} Let $\mathcal{R}$ (simply connected) be a closed region and $\mathcal{C} = \partial{\mathcal{R}}$ be the curve enclosing the region. Using Green's Theorem we get,

  \[\Area(\mathcal{R}) = \int_{\mathcal{R}} \dd A = \int_{\mathcal{C}} x \dd y = \int_{\mathcal{C}} -y \dd x = \int_{\mathcal{C}} \frac{x \dd y -y \dd x}{2} \]

\end{tcolorbox}

\begin{Eg}{Area inside the ellipse: $\frac{x^2}{a^2} + \frac{y^2}{b^2} = 1$}{}

  \textit{Solution.} Parametrization of ellipse $x = a \cos t, y = b \sin t$ where $t \in [0,2\pi)$. Using the above application of Green's Theorem we can write,
  \[\Area = \int_{\mathcal{C}} x \dd y = ab \int_{0}^{2\pi} \cos^2t \dd t = \pi ab \]
\end{Eg}

\begin{Thm}{Independence of path}{}
  Let $\vec{F}$ be a $C^1$ vector field on $\R^2$ such that $\displaystyle\int_{\mathcal{C}} \vec{F} \cdot \dd \vec{r}$ is independent of path. Then $\vec{F}$ is conservative over an open and simply connected domain.
\end{Thm}

\begin{proof}
  Let $\mathcal{D}$ be an open and connected domain. $\vec{F} = \inp{P}{Q}$ is defined over $\mathcal{D}$. Also let $P_0 = \inp{x_0}{y_0}$ be a fixed point in the domain $\mathcal{D}$ and $P_1 = \inp{x}{y} \in \mathcal{D}$ be a variable point. $\mathcal{C}$ be a smooth curve joining $P_0$ and $P_1$. Define

  \[\varphi(x,y) = \int_{\mathcal{C}} \vec{F} \cdot \dd{\vec{r}}\]

  Since, $\mathcal{D}$ is open set, so we must get an open ball centered at $P_1$ contained in $\mathcal{D}$. Take a point $P_1'= \inp{x_1}{y}$ inside that open ball such that $x_1 < x$. Let $\mathcal{C}_1$ be a  smooth curve from $P_0$ to $P_1$ and $\mathcal{C}_2$ be a line segment from $P_1'$ to $P_1$. So, $\mathcal{C}_1 \cup \mathcal{C}_2$ defines a smooth curve from $P_0$ to $P_1$.

  \[\begin{tikzcd}
      && {P_1'(x_1,y)} && {P_1(x,y)} \\
      \\
      {P_0(x_0,y_0)}
      \arrow["{\mathcal{C}_1}", curve={height=-18pt}, from=3-1, to=1-3]
      \arrow["{\mathcal{C}_2}", from=1-3, to=1-5]
      \arrow["{\mathcal{C}}"', curve={height=18pt}, from=3-1, to=1-5]
    \end{tikzcd}\]

  As $\displaystyle\int_{\mathcal{C}} \vec{F} \cdot \dd{\vec{r}}$ is path independent We can write, \[
    \varphi(x,y) = \int_{\mathcal{C}_1} \vec{F} \cdot \dd{\vec{r}} + \int_{\mathcal{C}_2} \vec{F} \cdot \dd{\vec{r}}
  \]

  Now we take the partial derivative of both sides of this equation with respect to $x$. The first integral does not depend on the variable $x$ since $\mathcal{C}_1$ is the path from $P_0(x_0,y_0,z_0)$ to $P'_1(x_1,y,z)$ and so partial differentiating this line integral with respect to $x$ is zero.

  \begin{align*}
    \pdv{\varphi}{x}
     & =\pdv{x} \left( \int_{\mathcal{C}_1} \vec{F} \cdot \dd{\vec{r}} + \int_{\mathcal{C}_2} \vec{F} \cdot \dd{\vec{r}} \right)                                           \\
     & = \underbrace{\pdv{x} \left( \int_{\mathcal{C}_1} \vec{F} \cdot \dd{\vec{r}}\right)}_{= 0}  + \pdv{x} \left(\int_{\mathcal{C}_2} \vec{F} \cdot \dd{\vec{r}} \right)
  \end{align*}

  Also, $\mathcal{C}_2$ can be parametrized as $r(t) = \inp{t}{y}$ where $t \in [x_1,x]$. So,

  \begin{align*}
    \pdv{x} \left(\int_{\mathcal{C}_2} \vec{F} \cdot \dd{\vec{r}}\right)
     & = \pdv{x} \left( \int_{x_1}^x \inp{P(t,y)}{Q(t,y)} \cdot \inp{1}{0} \dd{t} \right) \\
     & = \pdv{x} \left( \int_{x_1}^x P(t,y) \, \dd{t} \right)                             \\
     & = P(x,y) \qquad \text{[Fundamental Theorem of calculus]}
  \end{align*}

  Similarly, we can show that, $\pdv{\varphi}{y} = Q(x,y)$. And hence, $\grad\varphi = \vec{F}(x,y)$. We can define $\varphi$ as the potential of $\vec{F}$.
\end{proof}

\begin{Thm}{}{}
  Let $\mathcal{D}$ be a simply connected domain in $\R^2$ and $\vec{F}$ is a $C^1$ vector field on $\mathcal{D}$. Then $\vec{F}$ is conservative iff $\curl \vec{F} = 0$ on $\mathcal{D}$.
\end{Thm}

\begin{proof}
  ($\Rightarrow$) This direction is trivial.

  \

  ($\Leftarrow$) From Green's Theorem we can say that $ \displaystyle\int_{\mathcal{C}} \vec{F} \cdot \dd{\vec{r}} = 0$ over all closed curve $\mathcal{C}$. For any two point $p_0,p_1 \in \mathcal{D}$ if $\gamma_1, \gamma_2 : [0,1] \to \mathcal{D}$ are two smooth curves joining $p_0$ and $p_1$. (i.e., $\gamma_1(0) = \gamma_2 (0) = p_0$ and $\gamma_1(1) = \gamma_2(1) = p_1$) then $\gamma_1 \cup \gamma_2(1-t)$ is a closed curve. So, $ \displaystyle\int_{\gamma_1} \vec{F} \cdot \dd{\vec{r}}  = \displaystyle\int_{\gamma_2} \vec{F} \cdot \dd{\vec{r}}$. Which means the integral is path independent. Using the previous theorem we can say, $\vec{F}$ is conservative on $\mathcal{D}$.
\end{proof}

\section{Gauss Divergence Theorem}

\begin{Def}{Divergence of a vector field}{}
  \small  Given a vector field $\vec{F} = (f_1,\cdots, f_n) : \R^n \to \R^n $, the ``Divergence'' of $\vec{F}$ is,

  \[\text{div}(F) = \sum_{i=1}^{n} \pdv{f_i}{x_i} \equiv \div \vec{F}\]
\end{Def}

\begin{Thm}{Gauss Divergence Theorem}{gdt}
  Let $\mathcal{D} \subseteq \R^3$ a solid domain, $\partial{\mathcal{D}}$ be an oriented surface. Let $\vec{F} = \inp{P}{Q,R}$ be a $C^1$ vector field on an open surface containing $\mathcal{D} \cup \partial{\mathcal{D}}$. Then,
  \[
    \underbrace{\int_{\partial{\mathcal{D}} = \mathcal{S}} \vec{F}\cdot \dd{\vec{S}}}_\text{surface integral} = \underbrace{\int_{\mathcal{D}} \div \vec{F} \dd{V}}_\text{volume integral}
  \]
\end{Thm}

Just like FTC, the behavior over a volume is fully determined by the behavior at the boundary. Proof of this theorem is beyond our reach. But we can see the proof for simple cases.

\begin{proof}
  (For a simple case) Consider $\mathcal{D} = \qty{(x,y,z) \mid \varphi_1(x,y) \leq z \leq \varphi_2(x,y), (x,y) \in [a,b]\times [c,d]}$.

  (\textbf{Exercise.}) Complete the proof!
\end{proof}


\begin{Eg}{}{}
  $F (x,y,z) = \inp{x+y}{z^2,x^2}$ and $S$ be the hemisphere $x^2+y^2+z^2 = 1, z>0$. Compute,
  \[\int_{S} \vec{F} \, \dd \vec{S}\]

  \textit{Solution.} Notice that $S$ is open surface. We want to  use Gauss Theorem \ref{th:gdt}. So we need a close surface. Let $S_1$ be the surface $x^2 +y^2 \leq 1$. Then $S \sqcup S_1$ is a closed surface.

  \begin{align*}
    \int_{S \sqcup S_1} \vec{F}\cdot \dd \vec{S}
     & = \int_{x^2+y^2,z^2 \leq 1, z \ge 0} \div \vec{F} \, \dd{V} \\
     & = \int_{x^2+y^2,z^2 \leq 1, z \ge 0}\,\dd{V}                \\
     & = \frac{2\pi}{3}
  \end{align*}

  Parametrization of the surface $S_1 =\qty{(x,y,0) \mid x^2+y^2 = 1}$. So, $r_x \times r_y = \inp{1,0}{0} \times \inp{0,1}{0}$.

  \begin{align*}
    \int_{S_1} \vec{F} \cdot \dd \vec{S}
     & = \int_{x^2+y^2 \leq 1} \inp{x+y}{z^2,x^2} \cdot \inp{0,0}{1} \dd{A} = \int_{x^2+y^2 \leq 1} x^2 \dd{A} \\
     & = \int_{0}^{2\pi} \int_{0}^{1} r^3 \cos^2 \theta \,dr \,d\theta = \frac{\pi}{4}                         \\
    \Rightarrow \int_{S} \vec{F} \cdot \dd \vec{S}
     & = \frac{11\pi}{12}
  \end{align*}
\end{Eg}

\section{Stokes' Theorem}
\begin{Thm}{Stokes' Theorem}{st}
  Let $\mathcal{C}$ be a $C^1$ curve enclosing an oriented surface $\mathcal{S}$ in $\R^3$. Let,  $\vec{F} = \inp{P,Q}{R}$ be a $C^1$ vector field on an open set containing $\mathcal{S}$. Then,

  \small \[\int_{\mathcal{C}} \vec{F} \cdot \dd{\vec{r}} = \int_{\mathcal{S}} \left( \curl \vec{F} \right) \cdot\dd{\vec{S}}  \]

  Here orientation of $\mathcal{S}$ and direction of $\mathcal{C}$ is same.
\end{Thm}

\begin{Eg}{}{}
  Compute $\displaystyle\int_{\mathcal{C}} \vec{F} \cdot \dd{\vec{r}}$, where $\mathcal{C} : x^2 +y^2 = 9, z =4$ and $\vec{F} = \inp{-y,x}{xyz}$.

  \

  \textit{Solution.} $\curl \vec{F} = \inp{xz,-yz,}{2}$. By convention, we should assume direction of $\mathcal{C}$ is along counter-clockwise direction. So,  The normal vector of $\mathcal{S}$ is along negative $z$ axis. So, required integral,

  \begin{align*}
    \int_{\mathcal{C}} \vec{F} \cdot \dd{\vec{r}}
     & = \int_{\mathcal{S}} (\curl \vec{F}) \cdot \dd{\vec{S}}                   \\
     & = \int_{x^2+y^2 \leq 1, z =4 } (\curl \vec{F}) \cdot \inp{0,0}{-1} \dd{A} \\
     & = -2 \int_{x^2+y^2 \leq 1, z =4} \dd{A}                                   \\
     & = -18\pi
  \end{align*}

\end{Eg}


Stoke's Theorem is the $\R^3-$analogue of Green's Theorem \ref{th:gtm}. If we take the third component of $\vec{F}$ to be zero, i.e., $R = 0$, then Stoke's Theorem \ref{th:st} gives us back Green's Theorem \ref{th:gtm}.

\begin{tcolorbox}
  There is a generalized version of Stokes' theorem. Just for information the theorem is stated below.

  $\bullet$ If $\Omega$ is an oriented $n$-manifold (with boundary) and $\omega$ is a differential form ($(n-1)$ form). Then integral of $\omega$ over the boundary $\partial \Omega$ of the manifold $\Omega$ is given by,

  \[\int_{\partial \Omega} \omega = \int_{\Omega} \dd \omega\]

\end{tcolorbox}


\end{document}
