\documentclass[Analysis-3]{subfiles}

\begin{document}
\chapter*{Lecture 20} %Set chapter name
\addcontentsline{toc}{chapter}{Lecture 20} %Set chapter title
\setcounter{chapter}{20} %Set chapter counter

\section{Fubini's Theorem on Elementary Regions}

In the previous lecture we had talked about elementary regions, in this lecture we will see application of Fubini's theorem on elementary regions. 

\begin{Thm}{}{}\label{thm1:20}
    Let $f \in \mathscr{R}(\Omega)$ where $\Omega \subseteq \R^2$ is a bounded elementary domain. 
    \begin{enumerate}
        \item[(1)] If $\Omega = \{ (x,y) \mid a \leq x \leq b, \mbox{ and } \varphi_1(x) \leq y \leq \varphi_2(x) \}$ and if $\displaystyle{\int_{\varphi_1(x)}^{\varphi_2(x)}f(x,y) \, \dd y}$ exists for all $x \in [a,b]$ then 
        \[
            \int_{\Omega} f \, \dd A = \int_a^b \left( \int_{\varphi_1(x)}^{\varphi_2(x)} f(x,y) \, \dd y\right) \dd x    
        \]

        \item[(2)] Similarly we have 
        \[
            \int_{\Omega} f \, \dd A = \int_c^d \left( \int_{\psi_1(y)}^{\psi_2(y)} f(x,y) \, \dd x\right) \dd y    
        \]
        when $\Omega$ is $x$-simple. 
    \end{enumerate}
\end{Thm}
\textit{Proof.} There exists $c,d \in \R$ such that $\Omega \subseteq [a,b] \times [c,d] = B^2$. We know $\tilde{f} \in \mathscr{R}(B^2)$ where 
\[
    \tilde{f}(x,y) = \begin{cases}
        f(x,y) & \mbox{ if } (x,y) \in \Omega \\ 
        0 & \mbox{ if } (x,y) \in B^2 \setminus \Omega
    \end{cases}    
\]
Since $\tilde{f} \in \mathscr{R}(B^2)$, and since $\displaystyle{\int_{\varphi_1(x)}^{\varphi_2(x)} f(x,y) \, \dd y}$ exists for fixed $x$, hence 
\[
    \tilde{f}(x,\cdot)\big\vert_{[\varphi_1(x),\varphi_2(x)]} \quad \mbox{ and } \quad \tilde{f}(x,\cdot)\big\vert_{[c,d]\setminus[\varphi_1(x),\varphi_2(x)]} \equiv 0    
\]
are both Riemann integrable. Thus we get that $\tilde{f}(x,\cdot)\vert_{[c,d]} \in \mathscr{R}([c,d])$ and hence $\displaystyle{\int_c^d \tilde{f}(x,y) \, \dd y}$ exists for all $x \in [a,b]$ and further we have 
\[
    \int_c^d \tilde{f}(x,y) \, \dd y = \int_{\varphi_1(x)}^{\varphi_2(x)} f(x,y) \, \dd y \quad \forall \, x \in [a,b]    
\]
Then we get 
\begin{align*}
    &\Longrightarrow \int_a^b \left( \int_c^d \tilde{f}(x,y) \, \dd y\right)\, \dd x = \int_a^b \left( \int_{varphi_1(x)}^{\varphi_2(x)} f(x,y) \, \dd y\right) \dd x \\ 
    &\overset{(1)}{\Longrightarrow} \int_{B^2} \tilde{f} \, \dd A = \int_a^b \left( \int_{varphi_1(x)}^{\varphi_2(x)} f(x,y) \, \dd y\right) \dd x \\ 
    &\Longrightarrow \int_{\Omega} f \, \dd A = \int_a^b \left( \int_{varphi_1(x)}^{\varphi_2(x)} f(x,y) \, \dd y\right) \dd x
\end{align*}
where $(1)$ follows from Fubini's theorem, and one can do a similar argument for $x$-simple regions. $\hfill \blacksquare$

\

\begin{Eg}{}{}
    Let $f \in \mathscr{C}(\Omega)$ where $\Omega = \left\{ (x,y) \mid 0 \leq x \leq 1 - \frac{y}{2}, \mbox{ and } 0 \leq y \leq 2 \right\}$ then we can write $\Omega$ as a $y$-simple region as 
    \[
        \Omega = \left\{ (x,y) \mid 0 \leq x \leq 1, \mbox{ and } 0 \leq y \leq 2(1-x) \right\}    
    \]
    then using theorem $\ref{thm1:20}$ we get that 
    \begin{align*}
        \int_{\Omega} f \, \dd A &= \int_0^2 \left( \int_0^{1-\frac{y}{2}} f(x,y) \, \dd x\right) \dd y \\ 
        &= \int_0^1 \left( \int_0^{2(1-x)} f(x,y) \, \dd y\right) \dd x
    \end{align*}
\end{Eg}

\begin{Eg}{}{}
    Let $B^2 = [0,\pi] \times [-\frac{\pi}{2}, \frac{\pi}{2}]$, and we want to evaluate the integral $\int_{B^2} \sin (x+y) \, \dd A$. 
    \begin{align*}
        \int_{B^2}\sin(x+y) \, \dd A &= \int_{B^2} \sin x \cos y \, \dd A + \int_{B^2} \sin y \cos x \, \dd A \\ 
        &= \left( \int_{-\frac{\pi}{2}}^{\frac{\pi}{2}} \cos y \, \dd y\right)\left( \int_{0}^{\pi} \sin x \, \dd x\right) + \left( \int_{-\frac{\pi}{2}}^{\frac{\pi}{2}} \sin y \, \dd y\right)\left( \int_{0}^{\pi} \cos x \, \dd x\right) \\ 
        &= 4 
    \end{align*}
\end{Eg}

\begin{Eg}{}{}
    Let $\Omega$ be the region bounded by $y = 1$ and $y = x^2$, and we want to find $\int_{\Omega} x^2y \, \dd V$. We can write $\Omega$ as a $y$-simple region as follows: 
    \[
        \Omega = \{ (x,y) \mid -1 \leq x \leq 1, \mbox{ and } x^2 \leq y \leq 1 \}    
    \]
    Then using theorem $\ref{thm1:20}$ we get that 
    \begin{align*}
        \int_{\Omega} x^2y \, \dd A &= \int_{-1}^1 \left( \int_{x^2}^1 x^2y \, \dd y\right)\dd x \\ 
        &= \int_{-1}^1 x^2 \left( \frac{y^2}{2} \right)\bigg\vert_{x^2}^1 \, \dd x \\ 
        &= \int_{-1}^1 \frac{1}{2} x^2(1-x^2) \, \dd x \\ 
        &= \frac{2}{15}
    \end{align*}
\end{Eg}

\begin{Eg}{}{}
    Compute $\int_{[0,1]^2} f \, \dd A$ where $f : [0,1]^2 \to \R$ is given by
    \[
        f(x,y) = \begin{cases}
            x & \mbox{ if } y \leq x^2 \\ y & \mbox{ if } y > x^2 
        \end{cases} \quad \forall \, (x,y) \in [0,1]^2   
    \]
    and let 
    \[
        \Omega_1 = \{ (x,y) \mid 0 \leq x \leq 1, \, 0 \leq y \leq x^2 \} \quad \mbox{ and } \quad \Omega_2 = \{ (x,y) \mid 0 \leq x \leq 1, \, x^2 \leq y \leq 1 \}    
    \]
    Then $f\vert_{\Omega_1}$ and $f\vert_{\Omega_2}$ are Riemann integrable and $f\vert_{y=x^2}$ is not continuous but the set $\{ (x,x^2) \mid 0 \leq x \leq 1 \}$ is of content zero, and hence $f$ is integrable and we can make sense of writing 
    \[
        \int_{[0,1]^2} f \, \dd A = \int_{\Omega_1} f \, \dd A + \int_{\Omega_2} f \, \dd A     
    \]
    then using theorem $\ref{thm1:20}$ we get that 
    \[
        \int_{\Omega_1} f \, \dd A = \int_0^1 \left( \int_0^{x^2} x \, \dd y \right)\ \dd x = \frac{1}{4}     
    \]
    and 
    \[
        \int_{\Omega_2} f \, \dd A = \int_0^1 \left( \int_{x^2}^1 y \, \dd y\right)\dd x = \frac{2}{5}    
    \]
    and hence
    \[
        \int_{\Omega} f \, \dd A = \frac{13}{20}
    \]
\end{Eg}

\begin{Eg}{}{}
    Compute $\displaystyle{ \int_0^{\frac{\pi}{2}} \int_x^{\frac{\pi}{2}}} \frac{\sin y}{y} \, \dd y \, \dd x$. Define $\Omega = \left\{ (x,y) \mid 0 \leq x \leq \frac{\pi}{2}, \, x \leq y \leq \frac{\pi}{2} \right\}$. Then we can write $\Omega$ as a $x$-simple region as follows
    \[
        \Omega = \left\{ (x,y) \mid 0 \leq y \leq \frac{\pi}{2}, \, 0 \leq x \leq y \right\}    
    \]
    and hence using theorem $\ref{thm1:20}$ we get 
    \begin{align*}
        \int_{\Omega} \frac{\sin y}{y} \, \dd A &= \int_{0}^{\frac{\pi}{2}} \int_x^{\frac{\pi}{2}} \frac{\sin y}{y} \, \dd y \, \dd x \\ 
        &= \int_0^{\frac{\pi}{2}} \int_0^y \frac{\sin y}{y} \, \dd x \, \dd y \\ 
        &= \int_0^{\frac{\pi}{2}} \sin y \, \dd y \\ 
        &= 1
    \end{align*}
\end{Eg}

\section{Change of Variable}

Let us revise the change of variable rule for real valued functions on the real line. Let $\varphi : \mathcal{O} \to \R$ be a $\mathscr{C}^1$ function where $\mathcal{O} \subseteq \R$, assume $\varphi'(x) \neq 0$ for all $x \in \mathcal{O}$, and let $\mathcal{O} \supseteq [a,b]$, then for all $f \in \mathscr{C}(\varphi[a,b])$, we have 
\[
    \int_{\varphi(a)}^{\varphi(b)} f = \int_a^b (f \circ \varphi) \varphi'     
\]

Taking idea from this result we have the following theorem 

\begin{Thm}{Change of Variable}{}\label{thm2:20}
    Let $\mathcal{O}_n \subseteq \R^n$ be a open set, and let $\varphi : \mathcal{O}_n \to \R^n$ be an injective and $\mathscr{C}^1$ function and suppose $\det(J_{\varphi}(x)) \neq 0$ for all $x \in \mathcal{O}_n$. Let $\Omega \subseteq \mathcal{O}_n$, then for $f \in \mathscr{R}(\varphi(\Omega))$ 
    \[
        \int_{\varphi(\Omega)} f \, \dd V = \int_{\Omega} (f \circ \varphi) |\det J_{\varphi}|     
    \]
\end{Thm}
We won't be going over the proof of theorem $\ref{thm2:20}$, but for the proof you refer to page 67 of \textit{Calculus on Manifolds} by Michael Spivak.
\end{document}